\colquestion{Экспоненциальная ФКП. Гиперболическая и тригонометрическая ФКП.}
\begin{plan}
\item Любое дробно-линейное преобразование состоит из последовательного выполнения следующих
  операций:

  параллельный перенос -> инверсия->преобразование подобия -> параллельный перенос
\item Композицию двух дробно-линейных преобразований можно представить в виде умножения матриц с
  коэффициентами.
\end{plan}

\begin{col-answer-preambule}
\end{col-answer-preambule}
Для $z = x + iy; x, y \in \R{}$ экспоненциальная ФКП определяется как
\begin{align*}
  e^z = e^{x + iy} = e^xe^{iy} = e^x(\cos y + i\sin y).
\end{align*}
Используя соответствующие свойства тригонометрических о показательных действительных функций,
нетрудно получить, что ${\forall z_1, z_2 \in \mathbb{C} \Rightarrow e^{z_1}e^{z_2} = e^{z_1 + z_2}}$,
${e^{\frac{z_1}{z_2}} = e^{z_1 - z_2}}$, отсюда, в частности, имеем $e^0 = 1, \dfrac{1}{e^z} = e^{-z}$.
В отличие от действительной экспоненты, комплексная экспонента является уже периодической
функцией с чисто мнимым периодом $T = 2\pi i$, т.к.
\begin{align*}
  \forall k \in \mathbb{Z} \Rightarrow e^{z + 2\pi k i} = e^ze^{2\pi ki} = e^z(\cos 2\pi k +
  i\sin 2 \pi k) = e^z.
\end{align*}
Отсюда получаем, что $e^{2\pi i} = 1$. Непосредственно вычисляя также получаем $e^{\frac{i\pi}{2}} = i$,
$e^{\pi i} = -1$.

На основании комплексной экспоненты вводятся комплексные гиперболические функции
\begin{align*}
  \begin{cases}
    \ch z = \dfrac{e^z + e^{-z}}{2}\\
    \sh z = \dfrac{e^z - e^{-z}}{2}\\
  \end{cases}
\end{align*}

Непосредственно вычисляя для них получаем
\begin{enumerate}
\item $\ch^2 z - \sh^2z = 1$;
\item $\ch^2 z + \sh^2z = \ch 2z$;
\item $\sh 2z = 2\sh z\ch z$;
\end{enumerate}

Из $\sh$ и $\ch$ определяем
\begin{align*}
  &\th z = \dfrac{\sh z}{\ch z} = \dfrac{e^{2z} - 1}{e^{2z} + 1},
  z \neq i(\dfrac{\pi}{2} + 2\pi k), k \in \mathbb{Z},\\
  &\cth z = \dfrac{1}{\th z} = \dfrac{e^{2z} + 1}{e^{2z} - 1},z \neq i2\pi k, k \in \mathbb{Z}.\\
  &1 - \th^2z = \dfrac{1}{\ch^2z}, \quad \cth^2z - 1 = \dfrac{1}{\sh^2z},\\
  &\th 2z = \dfrac{2\th z}{1 + \th^2 z}, \quad \cth 2z = \dfrac{2\cth z}{1 + \cth^2z}
\end{align*}

С помощью комплексной экспоненты вводится также тригонометрическая ФКП
\begin{align*}
  &\cos z = \dfrac{e^{iz} + e^{-iz}}{2} = \ch(iz),\\
  &\sin z = \dfrac{e^{iz} - e^{-iz}}{2i} = -i\sh(iz),\\
  &\tg z = \dfrac{\sin z}{\cos z} = -i\th(iz),\\
  & \ctg z = \dfrac{\cos z}{\sin z} = i\cth(iz).
\end{align*}

В данном случае, если в $z = x + iy$ взять $y = 0, x \in \R{}$, то тригонометрические ФКП совпадают
с действительными тригонометрическими функциями. В связи с этим большинство тригонометрических
формул для действительных тригонометрических функций (кроме связанных с неравенствами) будут
справедливы для тригонометрических ФКП, например
\begin{align*}
  &\cos^2z + \sin^2z = 1,\\
  &1 + \tg^2z = \dfrac{1}{\cos^2z},\\
  &\cos 2z = \cos^2z - \sin^2z,\\
  &\sin 2z = 2\sin z \cos z.\\
\end{align*}
В данном случае тригонометрические ФКП являются неограниченными функциями, в отличие от, например,
действительных $\cos$ и $\sin$.

Например
\begin{align*}
  \cos(i\ln 2) = \ch(-\ln 2) = \dfrac{e^{\ln 2} + e^{-\ln2}}{2} = \dfrac{2 + \frac{1}{2}}{2} =
  \dfrac{5}{4} \notin \interval[{-1; 1}]
\end{align*}
