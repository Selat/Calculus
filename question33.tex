\colquestion{Достаточное условие равномерной сходимости Р.Ф.}

\begin{col-answer-preambule}
\end{col-answer-preambule}

\begin{theorem}[Достаточное условие равномерной сходимости РФ]
  Пусть $2\pi$-периодическая функция $f \in \mathbb{C}(\interval[{-\pi; \pi}]])$ и имеет
    кусочно-непрерывную производную. Тогда РФ для $f(x)$ сходится к этой функции на
    $\interval[{-\pi; \pi}]]$ абсолютно и равномерно.
\end{theorem}
\begin{proof}
  Т.к. $f(x)$ имеет кусочно-непрерывную производную на $\interval[{-\pi; \pi}]$, то этот
  отрезок можно разбить на конечное число частей, внутри каждой из которых $\exists f'(x)$
  непрерывная, а на концах любое из отрезков разбиения у $f(x)$ существуют конечные производные.
  Для простоты будем считать, что это условие выполняется для $\interval[{-\pi; \pi}]$.

  В силу $2\pi$-периодичности имеем
  \begin{align*}
    \forall x \in \R{} \Rightarrow f(x + 2\pi) = f(x), f'(x + 2\pi) = f'(x).
  \end{align*}
  Учитывая, что на концах $f(-\pi) = f(\pi)$, то
  \begin{align*}
    f'(x) \sim \suml_{k = 1}^{\infty}kb_k\cos kx - ka_k\sin kx,
  \end{align*}
  где $a_k, b_k$ - коэффициенты Фурье для $f(x)$ на $\interval[{-\pi; \pi}]$. Используя
  \text{неравенство Бесселя} для $f'(x)$, получаем
  \begin{align*}
    \suml_{k = 1}^n(kb_k)^2 + (-ka_k)^2 \leq \dfrac{1}{\pi}\intl_{-\pi}^{\pi}(f'(x))^2dx.
  \end{align*}
  Имеем
  \begin{align*}
    0 \leq \abs{a_k} + \abs{b_k} = \dfrac{\abs{ka_k}}{k} + \dfrac{\abs{kb_k}}{k} \leq
    \dfrac{1}{2}((kb_k)^2 + \frac{1}{k^2}) + \dfrac{1}{2}((ka_k)^2 + \frac{1}{k^2}) \leq
    \sqcase{
      \dfrac{\alpha + \beta}{2} \geq \sqrt{\alpha\beta}\\
      1) \alpha = kb_k, \beta = \dfrac{1}{k}\\
      2) \alpha = ka_k, \beta = \dfrac{1}{k}} = \dfrac{1}k^2(a_k^2 + b_k^2) + \dfrac{1}{k^2}
  \end{align*}
  Отсюда получаем
  \begin{align*}
    \suml_{k = 1}^n(\abs{a_k} + \abs{b_k}) \leq \dfrac{1}{2}\suml_{k = 1}^nk^2(a_k^2 + b_k^2) +
    \suml_{k = 1}^n\dfrac{1}{k^2} \leq \dfrac{1}{2\pi}\intl_{-\pi}^{\pi}(f'(x))^2dx + \const.
  \end{align*}
  По критерию сходимости положительных рядов получаем, что $\dfrac{\abs{a_0}}{2} +
  \suml_{k = 1}^n(\abs{a_k} + \abs{b_k})$ сходится. А т.к. этот ряд - мажоранта для РФ $f(x)$, т.к.
  \begin{align*}
    \begin{cases}
      \abs{a_k\cos kx} \leq \abs{a_k}, k \in \mathbb{N},\\
      \abs{b_k\sin kx} \leq \abs{b_k}, k \in \mathbb{N},
    \end{cases}
  \end{align*}
  то (эта мажоранта сходящаяся) РФ для $f(x)$ будет сходится по признаку Вейерштрасса.
\end{proof}
