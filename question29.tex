\begin{col-answer-preambule}
\end{col-answer-preambule}

\colquestion{Теорема об ортогональности основной тригонометрической системы, следствие из неё и замечание к ней.}

Система функций
\begin{equation}
  \label{eq:lecture06-03}
  1, \cos x, \sin x, \cos 2x, \sin 2x, \ldots, \cos nx, \sin nx, \ldots -
\end{equation}
\important{основная тригонометрическая система} (ОТС). Функции \eqref{eq:lecture06-03} имеют общий
$\R{}_+$ период, равный $T_0 = 2\pi$.

\begin{theorem}[Об ортогональности ОТС]
  Система функций \eqref{eq:lecture06-03} ортогональна на $\interval[{-\pi; \pi}]$.
\end{theorem}
\begin{proof}
  $\forall k, m \in \mathbb{N}_0, k \neq m$ имеем:
  \begin{enumerate}
  \item $\begin{aligned}[t]
    & <\cos kx, \cos mx> = \intl_{-\pi}^{\pi}\underbrace{\cos kx \cos mx}_{\text{чётная}}dx =
    =2\intl_0^{\pi}\dfrac{1}{2}(\cos(k - m)x + \cos(k + m)x)dx = \\
    &=\sqcase{\dfrac{\sin(k - m)x}{k - m} + \dfrac{\sin(k + m)x}
      {k + m}}_0^{\pi} = \sqcase{\sin \pi n = 0, \forall n \in \mathbb{Z}} = 0,\\
    &\text{Т.е. } \cos kx \bot \cos mx, \forall k \neq m.
  \end{aligned}$
  \item $\begin{aligned}[t]
    & <\cos kx, \sin mx> = \intl_{-\pi}^{\pi}\underbrace{\cos kx \sin mx}_{\text{нечётная}}dx = 0\\
    &\text{Т.е. } \cos kx \bot \sin mx, \forall k \neq m.
  \end{aligned}$
  \item $\begin{aligned}[t]
    & <\sin kx, \sin mx> = \intl_{-\pi}^{\pi}\underbrace{\sin kx \sin mx}_{\text{чётная}}dx
    =2\intl_0^{\pi}\dfrac{1}{2}(\cos(k - m)x - \cos(k + m)x)dx = \\
    &=\sqcase{\dfrac{\sin(k - m)x}{k - m} - \dfrac{\sin(k + m)x}
      {k + m}}_0^{\pi} = \sqcase{\sin \pi n = 0, \forall n \in \mathbb{Z}} = 0,\\
    &\text{Т.е. } \sin kx \bot \sin mx, \forall k \neq m.
  \end{aligned}$
  \end{enumerate}
\end{proof}

\begin{consequence}
  Ортогональной ОТС \eqref{eq:lecture06-03} соответствует ортонормированная тригонометрическая
  система на $\interval[{-\pi; \pi}]$:
  \begin{equation}
    \label{eq:lecture06-04}
    \dfrac{1}{\sqrt{2\pi}}, \dfrac{\cos x}{\sqrt{\pi}}, \dfrac{\sin x}{\sqrt{\pi}},
    \dfrac{\cos 2x}{\sqrt{\pi}}, \dfrac{\sin 2x}{\sqrt{\pi}}, \ldots,
    \dfrac{\cos nx}{\sqrt{\pi}}, \dfrac{\sin nx}{\sqrt{\pi}}, \ldots
  \end{equation}
\end{consequence}
\begin{proof}
  Следует из того, что
  \begin{align*}
    &\norm{1} = \parenthesis{\intl_{-\pi}^{\pi}1^2dx}^{\frac{1}{2}} = \sqrt{2\pi};\\
    &\norm{\cos kx} = \parenthesis{\intl_{-\pi}^{\pi}\cos^2kx^2dx}^{\frac{1}{2}} =
    \parenthesis{2\intl_{0}^{\pi}\dfrac{1 + \cos 2kx}{2}dx}^{\frac{1}{2}} =
    \parenthesis{\sqcase{x + \dfrac{\sin 2kx}{2k}}}^{\frac{1}{2}} = \sqrt{\pi};\\
    &\norm{\sin kx} = \parenthesis{\intl_{-\pi}^{\pi}\sin^2kx^2dx}^{\frac{1}{2}} =
    \parenthesis{2\intl_{0}^{\pi}\dfrac{1 - \cos 2kx}{2}dx}^{\frac{1}{2}} =
    \parenthesis{\sqcase{x - \dfrac{\sin 2kx}{2k}}}^{\frac{1}{2}} = \sqrt{\pi};\\
  \end{align*}
  Поэтому в силу доказанной выше теоремы система \eqref{eq:lecture06-04} будет не только
  ортогональной на $\interval[{-\pi; \pi}]$, но и ортонормированной на $\interval[{-\pi; \pi}]$, т.к.
  норма любой функции из \eqref{eq:lecture06-04} равна 1.
\end{proof}
\begin{notes}
\item Т.к. $T_0 = 2\pi > 0$ - общий период функций \eqref{eq:lecture06-03}, то на основании леммы об
  интеграле от интегрируемой периодической функции, рассмотренном на промежутке длины периода,
  получаем, что ОТС ортогональна на любом отрезке $\interval[{a; a + 2\pi}], \fix a \in \R{}$. В
    доказанной теореме для удобства было взято $a = -\pi$.
  \item Наряду с ортогональной \eqref{eq:lecture06-03} и ортонормированной \eqref{eq:lecture06-04}
    рассмотрим также \important{обобщённую тригонометрическую систему}
    \begin{equation}
      \label{eq:lecture06-05}
      1, \cos \dfrac{\pi x}{l}, \sin \dfrac{\pi x}{l}, \ldots, \cos \dfrac{\pi nx}{l},
      \sin \dfrac{\pi nx}{l}, \ldots
    \end{equation}
    у функций которой общий период $T = 2l > 0$. \eqref{eq:lecture06-05} также будет ортогональной
    на любом отрезке $\interval[{a; a + 2\pi}], \fix a \in \R{}$ и ей будет соответствовать следующая
      \important{обобщённая ортонормированная система}
      \begin{equation}
        \label{eq:lecture06-06}
        \dfrac{1}{\sqrt{2\pi}}, \dfrac{1}{\sqrt{l}}\cos \dfrac{\pi x}{\sqrt{\pi}},
        \dfrac{1}{\sqrt{l}}\sin \dfrac{\pi x}{\sqrt{\pi}}, \ldots
        \dfrac{1}{\sqrt{l}}\cos \dfrac{\pi nx}{\sqrt{\pi}},
        \dfrac{1}{\sqrt{l}}\sin,\dfrac{\pi nx}{\sqrt{\pi}}, \ldots
      \end{equation}
\end{notes}
