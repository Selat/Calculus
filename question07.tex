\colquestion{Теорема о почленном дифференцировании ФР}

\item Берём интеграл с переменным верхним пределом (т.е. на $[a; x]$).
\item Выражаем $S(x)$, дифференцируем по теореме Барроу.
\end{plan}
\begin{theorem}[о почленном дифференцировании ФР]
	Пусть ФР $\sum u_n(x)$ на $X = [a,b]$ удовлетворяет условиям:
	\begin{enumerate}
		\item $\sum u_n(x) \overset{X}{\rightarrow}$,
		\item $\exists \; u_n^{'}(x)$, непрерывная для $\forall \; n \in \mathbb{N}, x \in X$.
	\end{enumerate}
	Тогда, если
    \begin{equation}
        \label{eq:1_34}
        \sum u_n^{'}(x) \overset{X}{\rightrightarrows}
    \end{equation}
    то рассматриваемый ФР $\sum u_n(x)$ можно почленно дифференцировать на $[a,b]$, т.е.
	\begin{equation}
	\label{eq:1_35}
	\exists \left( \sum_{n=1}^{\infty} u_n (x) \right)^{'} = \sum_{k=1}^{\infty} u_k^{'}(x), \text{для }\forall \; x \in X.
	\end{equation}
\end{theorem}
\begin{proof}
	В силу \eqref{eq:1_34}, по условию 2 рассматриваемой теоремы получаем, что по теореме об интегрировании ФР $\sum u_n^{'}(t)$ можно почленно интегрировать на
    	$\forall \; [a,x] \subset [a,b]$, \nolinebreak т.е.

    \begin{equation*}
        \exists \dint\limits_a^x \left(\sum\limits_{n=1}^{\infty} u_n^{'}(t)\right)dt = \sum_{n=1}^{\infty} \dint\limits_a^x u_n^{'}(t)dt = \sum_{n=1}^{\infty} [u_n]^{t = x}_{t = a} = \sum\limits_{n=1}^{\infty}\left(u_n(x) - u_n(a)\right).
    \end{equation*}

	Отсюда в силу условия 1 (поточечная сходимость для $\sum u_n(x)$) получаем, что
    \begin{equation*}
            \exists \; S(x) = \sum\limits_{n=1}^{\infty} u_n(x) = \sum\limits_{n=1}^{\infty}u_n(a) + \dint\limits_a^x \sum\limits_{n=1}^{\infty}u_n^{'}(t)dt.
    \end{equation*}

	Используя далее \important{теорему Барроу} о дифференцировании интеграла с переменным верхним пределом от непрерывной подынтегральной функции, получаем:\\
	\begin{equation*}
	\exists \; S^{'}(x) = (const)^{'} + \left(\dint\limits_a^x \left( \sum\limits_{n=1}^{\infty} u_n^{'} (t) \right)dt \right)^{'}_x = \sum\limits_{n=1}^{\infty}u_n^{'} (x),
	\end{equation*} что соответствует \eqref{eq:1_35}.
\end{proof}
