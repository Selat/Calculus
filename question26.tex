\begin{col-answer-preambule}
\end{col-answer-preambule}

\colquestion{Третья теорема Фруллани.}
\begin{plan}
\item Рассматриваем новую функцию $f_0(t) = f(1/t)$, доопределяем её в нуле.
\item Вычисляем $\Phi(a_0; b_0)$.
\item Показваем, что $\Phi(a_0; b_0) = \Phi(a; b)$
\end{plan}
\begin{statementDotted}{Третья теорема Фруллани}$  $

	Пусть $ f(x) $ непрерывна для $ \forall \; x > 0 $ и $\exists \; f(+\infty) \in \R{}$.

	Тогда, если для  $\forall \; A > 0 \Rightarrow \dintl_0^A \dfrac{f(x)}{x} dx $ сходится, то
	\begin{equation}
	\label{5.08}
	\Phi (a, b) \overset{\eqref{5.04}}{=} - f(+\infty) \ln \frac{b}{a}.
	\end{equation}
\end{statementDotted}
\begin{proof}
	Рассмотрим $ f_0(t) = f\left(\dfrac{1}{t}\right)  $, непрерывную для $ \forall \; t > 0 $.

	Во-первых,
	$ \exists \; f_0(+0) = \lim\limits_{t \to +0} f \left(\dfrac{1}{t}\right) = f (+\infty ) \in \R{}$,
	поэтому $ f_0 $ можно доопределить в точке $ t = 0 $, приняв \newline
	$ f_0(0) = f_0(+0) = f(+\infty) \in \R{} $.

	$  $

	Во-вторых, для полученной непрерывной $f_0(t) $ для 
	$ \forall \; A_0 > 0 \Rightarrow \exists \dintl_{A_0}^{+\infty} \dfrac{f_0(t)}{t} dt =
	\begin{sqcases}
	t = \dfrac{1}{x} \\
	A = \dfrac{1}{A_0} > 0
	\end{sqcases}
	=
	\dintl_0^A \dfrac{f(x)}{x} dx \in \R{} $
	сходится.

	$  $

	Таким образом, в силу второй теоремы Фруллани, имеем:
	\begin{equation*}
	\Phi (a_0, b_0)
	= \begin{sqcases}
	a_0 = \frac{1}{a} > 0 \\
	b_0 = \frac{1}{b} > 0
	\end{sqcases}
	= f_0 (0) \cdot \ln \dfrac{b_0}{a_0}
	= f(+\infty) \ln \left(\dfrac{\parenthesis{\frac{1}{b}}}{\parenthesis{\frac{1}{a}}}\right)
	= - f(+\infty) \ln \frac{b}{a}.
	\end{equation*}

	С другой стороны, получаем:
	\begin{align*}
	& \Phi_0 (a_0, b_0)
	= \dintl_0^{+\infty} \dfrac{f_0(a_0 t)-f_0(b_0 t)}{t} dt
	= \dintl_0^{+ \infty} \left(f_0 \left(\dfrac{a}{t} \right) - f_0 \left(\dfrac{b}{t} \right) \right) \cdot \dfrac{1}{t} \; dt = \sqcase{ t = \dfrac{1}{x} }
	= \ldots
	= \\ &
	= \dintl_0^{+\infty} \dfrac{f(ax)-f(bx)}{x} dx
	\overset{\eqref{5.04}}{=} \Phi (a, b).
	\end{align*}
	Таким образом, $ \Phi (a, b) = - f(+\infty) \; \ln \dfrac{b}{a}. $
\end{proof}
