\begin{col-answer-preambule}
\end{col-answer-preambule}

\colquestion{В-функция Эйлера и её основные свойства.}

\important{Эйлеровым интегралом I рода} или \important{B-функцией Эйлера} называется НИЗОП-2 следующего вида:
\begin{equation}
    \label{6.17}
    B(a; b) = \dintl_0^1 x^{a-1} (1-x)^{b-1} \; dx,
\end{equation}
где $ a, b $ - некоторые константы.

$  $\\

Для исследования \eqref{6.17} на сходимость, отделяя возможные особенности, имеем:
\begin{equation*}
    B(a, b) = \dintl_0^\frac{1}{2} x^{a-1} (1-x)^{b-1} \; dx + \dintl_\frac{1}{2}^1 x^{a-1} (1-x)^{b-1} \; dx.
\end{equation*}

Для подынтегральной функции $ f(x) = x^{a-1} (1-x)^{b-1} $ в точках $ x=0 $ и $ x=1 $ получаем:
\begin{enumerate}
    \item $ x=0 $: $\;\;\;$ $ f(x) \underset{x \to +0}{\sim} x^{a-1} = \dfrac{1}{x^{1-a}}$.

    Отсюда видно, что первое слагаемое будет сходиться при $ 1-a < 1 \Leftrightarrow a > 0$.

    \item $ x=1 $: $\;\;\;$ $ f(x) \underset{x \to 1-0}{\sim} (1-x)^{b-1} = \dfrac{1}{(1-x)^{1-b}}$.

    Аналогичным образом имеем условие сходимости второго слагаемого:
    ${ 1-b < 1 \Leftrightarrow b > 0 }$.
\end{enumerate}

Таким образом, областью сходимости для B-функции Эйлера будет
$ \begin{cases}
    a > 0,\\
    b > 0.
\end{cases} $

$  $\\\\

В дальнейшем нам понадобится представление B-функции не в виде НИЗОП-2, а в виде НИЗОП смешанного типа.
Для этого введём замену
$ x = \dfrac{t}{1+t} \Rightarrow t = \dfrac{x}{1-x} \dvert_0^{+\infty}, $ \linebreak
$ dx = \dfrac{dt}{(1+t)^2} $. Имеем:
\begin{equation}
    \label{6.18}
    B(a,b) \overset{\eqref{6.17}}{=}
    \dintl_0^{+\infty} \left(\dfrac{t}{1+t}\right)^{a-1} \left(1-\dfrac{t}{1+t}\right)^{b-1} \; \dfrac{dt}{(1+t)^2} =
    \dintl_0^{+\infty} \dfrac{t^{a-1}}{(1+t)^{a+b}} \; dt.
\end{equation}

Отсюда, учитывая, что B-функция симметрична относительно своих переменных, т. е.
\begin{equation*}
    B(a, b) \overset{\eqref{6.17}}{=} \sqcase{x = 1-y} =
    \dintl_0^1 (1-y)^{a-1} y^{b-1} \; dy =
    \sqcase{y \leftrightarrow x} =
    \dintl_0^1 x^{b-1} (1-x)^{a-1} \; dx
    = B(b,a) ,
\end{equation*}
получаем НИЗОП смешанного типа:
\begin{equation}
    \label{6.19}
    B(a,b) = \dintlzi \dfrac{t^{b-1} \; dt}{(1+t)^{a+b}}.
\end{equation}

\begin{theorem}[связь между B- и Г- функциями]
    Для любых $ a, b > 0 $ имеем:
    \begin{equation}
        \label{6.20}
        B(a, b) = \dfrac{\Gamma(a) \; \Gamma(b)}{\Gamma (a + b)}.
    \end{equation}
\end{theorem}
\begin{proof}
    Используя определения B- и Г- функций и представление \eqref{6.18}, получаем:
    \begin{align*}
        &
        B(a, b) \cdot \Gamma(a+b)
        = \left( \dintlzi \dfrac{t^{a-1}}{(1+t)^{a+b}} \; dt \right)
        \cdot \left( \dintlzi e^{-x} x^{a+b-1} \; dx \right)
        = \\ &
        = \dintlzi  \dintlzi  \dfrac{e^{-x} \; x^{a+b-1} \; t^{a-1} }{(1+t)^{a+b}} \; dx \; dt
        = \dintlzi  \left(\dintlzi  \dfrac{e^{-x} \; x^{a+b-1} \; t^{a-1} }{(1+t)^{a+b}} \; dx\right) dt
        = \\ &
        = \begin{sqcases}
            t =  \fix\\
            x = (1+t)y \Rightarrow dx = (1+t)dy \\
            y = \dfrac{x}{1+t} \dvert_0^{+\infty}
        \end{sqcases}
        = \dintlzi \left( \dintlzi \dfrac{e^{-(1+t)y} \left((1+t)y\right)^{a+b-1} \; t^{a-1}}{(1+t)^{a+b}} \;  (1+t) \; dy \right) dt
        = \\ &
        = \dintlzi \left( \dintlzi e^{-(1+t)y} \; y^{a+b-1} \; t^{a-1} \; dy \right) dt
        = \dintlzi \left( \dintlzi e^{-(1+t)y} \; y^{a+b-1} \; t^{a-1} \; dt \right) dy
        = \\ &
        = \begin{sqcases}
            y = \fix\\
            t = \dfrac{z}{y} \Rightarrow dt = \dfrac{dz}{y} \\
            z = \sqcase{-y}_0^{+\infty}
        \end{sqcases}
        = \dintlzi \left( \dintlzi e^{-(y+z)} \; y^{b-1} \; z^{a-1} \; dz \right) dy
        = \dintlzi \dintlzi \left(e^{-y} y^{b-1}\right)  \left(e^{-z} z^{a-1} \right)
            dy \; dz
        = \\ &
        = \dintlzi \underbrace{\left( \dintlzi e^{-y} y^{b-1} dy \right)  }_{\Gamma(b)}
            e^{-z} z^{a-1} \;  dz
        = \Gamma(b) \dintlzi e^{-z} z^{a-1} \; dz
        = \Gamma(a) \cdot \Gamma(b) \Rightarrow \eqref{6.20}.
    \end{align*}
\end{proof}

\begin{note}
    Из \eqref{6.20} на основании соответствующих свойств Г-функций получаем аналогичные свойства B-функций:
    \begin{enumerate}
        \item \textit{Симметричность:}
        \begin{equation}
            \label{6.21}
            \begin{cases}
                B(a, b) = \dfrac{\Gamma(a) \Gamma(b)}{\Gamma(a+b)}, \\
                B(b, a) = \dfrac{\Gamma(b) \Gamma(a)}{\Gamma(b+a)}
            \end{cases}
            \Rightarrow
            \begin{cases}
                B(a, b) = B(b, a), \\
                \forall \; a, b > 0
            \end{cases}
        \end{equation}

        \item \textit{Формулы понижения аргумента:}
        \begin{align*}
            & B(a+1, b) = \dfrac{\Gamma(a+1) \Gamma(b)}{\Gamma(a+b+1)} =
            \dfrac{a \; \Gamma(a) \; \Gamma(b)}{(a+b) \; \Gamma(a+b)} =
            \dfrac{a}{a+b} B(a, b), \;\;\; \forall \; a > 0, b > 0,
            \\ &
            B(a, b+1) = \dfrac{b}{a+b} B(a, b),  \;\;\; \forall \; a > 0, b > 0.
        \end{align*}

        \item \textit{Значения B-функции при натуральном значении одного из аргументов:}
        \begin{align*}
            &
            B(n+1, b) = \dfrac{n}{b+n} \; B(n, b) = \dfrac{n(n-1)}{(b+n)(b+n-1)} \; B(n-1, b)   = \ldots =
            \\ &
            = \dfrac{n \cdot (n-1) \cdot \ldots  \cdot 2  \cdot 1 }{(b+n) \cdot (b+n-1) \cdot \ldots \cdot (b+2) \cdot (b+1)} \; B(1, b).
        \end{align*}
        Отсюда, учитывая, что $ B(1, b) \overset{\eqref{6.17}}{=} \dintl_0^1 (1-x)^{b-1} \; dx = \sqcase{- \dfrac{(1-x)^b}{b}}_0^1  \overset{b>0}{=} \dfrac{1}{b}$, получаем:
        \begin{equation}
            \label{6.22}
            B(n+1, b) = \dfrac{n!}{b(b+1) \ldots (b+n)} , \;\; b > 0, \;\; n \in \mathbb{N}.
        \end{equation}
        Аналогично, в силу симметрии, для $ \forall \; m \in \mathbb{N} $:
        \begin{equation}
            \label{6.23}
            B(a, m+1) = \dfrac{m!}{a(a+1) \ldots (a+m)} , \;\; a > 0.
        \end{equation}

        Из \eqref{6.22} и \eqref{6.23}, при $ a = n \in \mathbb{N} $ и $ b = m \in \mathbb{N} $, имеем формулу для вычисления значения B-функции с натуральными аргументами:
        \begin{equation}
            \label{6.24}
            B(n+1, m+1) = \dfrac{n!}{m(m+1) \ldots (m+n)} = \dfrac{m!}{n(n+1) \ldots (n+m)}
            = \dfrac{m! \; n!}{(m+n+1)!} \; .
        \end{equation}

        Непосредственной проверкой убеждаемся, что \eqref{6.24} верно не только в случае, когда $ n, m \in \mathbb{N} $, но и при $ n, m \in \mathbb{N}_0 $.

        \item \textit{Вычисление значения В-функции, когда оба аргумента - полуцелые числа:}
        \begin{equation} \begin{split}
            & \forall \; n, m \in \mathbb{N} \; \Rightarrow \; B\left(n+\dfrac{1}{2}, \; m+ \dfrac{1}{2}\right)
            = \dfrac{\Gamma(n + \dfrac{1}{2})\Gamma(m+ \dfrac{1}{2})}{\Gamma(n+m+1)}
            = \\ &
            = \begin{sqcases}
                \Gamma \left(k+\dfrac{1}{2}\right) = \dfrac{(2k-1)!!}{2^k} \sqrt{\pi} \\
                \Gamma (n+m+1) = (n+m)! \\
                \forall \; k \in \mathbb{N} \;\;\;\;\;\;\;\;\;\;\;\;\;\;\;\;\;\;\;\;\;\;\;\;\;\;
            \end{sqcases}
            = \dfrac{(2n-1)!!(2m-1)!!}{(n+m)! \cdot 2^{n+m}} \cdot \pi.
        \end{split}
        \end{equation}

        \item \textit{Формула дополнения для В-функции:}
        \begin{equation}
            \label{6.26}
            \forall \; a \in ]0;1[ \;\; \Rightarrow \;
            B(a, 1-a) = \dfrac{\Gamma(a)\Gamma(1-a)}{\Gamma(a+1-a)} =
            \dfrac{\;\left(\dfrac{\pi}{\sin \pi a}\right)\;}{\Gamma(1)} =
            \dfrac{\pi}{\sin \pi a}.
        \end{equation}
        Отсюда, в частности, для $ n = \dfrac{1}{2} $ имеем:
        \begin{equation*}
            B\left(\dfrac{1}{2}, \; \dfrac{1}{2}\right) = \dfrac{\pi}{\sin \dfrac{\pi}{2}} = \pi.
        \end{equation*}
    \end{enumerate}
\end{note}
