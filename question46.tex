\colquestion{Логарифмическая ФКП и общая степенная ФКП.}
\begin{plan}
\item $z = x + iy, \omega = u + iv$. Выражаем $u, v$.
\item Расписываем $\Ln\parenthesis{\dfrac{z_1}{z_2}}$ и $\Ln\parenthesis{z_1z_2}$.
\item Общая степенная функция $z^{\alpha} \overset{z \neq 0}{=} e^{\alpha \Ln z}$.
\item Общая показательная функция $a^z = e^{z\Ln a}$.
\end{plan}

\begin{col-answer-preambule}
\end{col-answer-preambule}
Логарифмическая ФКП $\omega = \Ln z$ определяется как решение уравнения $e^{\omega} = z$, для
получения явной формулы для решения этого уравнения запишем
\begin{align*}
  &z = x + iy, \omega = u + iv; x, y, v, u \in \R{}.\\
  &e^{u + iv} = x + iy \Rightarrow e^u(\cos v + i\sin v) = x + iy \Rightarrow\\
  &\Rightarrow
  \begin{cases}
    e^u\cos v = x,\\
    e^u\sin v = y,
  \end{cases}
\end{align*}
\begin{enumerate}
\item $\begin{aligned}[t]
  x^2 + y^2 = e^{2u}(\cos^2v + \sin^2v) = e^{2u} \Rightarrow u = \dfrac{1}{2}\ln(x^2 + y^2) =
  \ln\sqrt{x^2 + y^2} = \ln\abs{z}, z \neq 0.
\end{aligned}$
\item $\begin{aligned}[t]
\begin{cases}
  \cos v = \dfrac{x}{e^u} = \dfrac{x}{\sqrt{x^2 + y^2}},\\
  \sin v = \dfrac{y}{e^u} = \dfrac{y}{\sqrt{x^2 + y^2}},\\
\end{cases}
\end{aligned}$
  поэтому, используя $\phi = \arg z$ и учитывая, что $\cos \phi =
  \dfrac{\operatorname{Re} z}{\abs{z}}$, ${\sin \phi = \dfrac{\operatorname{Im}z}{\abs{z}}}$,
  получаем
  $\begin{aligned}[t]
    \begin{cases}
      \cos v = \cos \phi\\
      \sin v = \sin \phi\\
    \end{cases}
    \Rightarrow v = \phi + 2\pi k, k \in \mathbb{Z}.\\
    \forall z \neq 0 \Rightarrow \Ln z = u + iv = \ln\abs{z} + i(\phi + 2\pi k) =
    \ln\abs{z} + i(\arg z + 2\pi k), k \in \mathbb{Z}.
  \end{aligned}$
\end{enumerate}
Логарифм ФКП - многозначная функция. Для неё ветвь, соответствующая $k = 0$ - главное значение
$\Ln z, z \neq 0$, и обозначается $\ln z = \ln\abs{z} + i\arg z$. Используя форм. действия над
множествами, в силу определения $\Ln z$ имеем
\begin{align*}
  \forall z_1, z_2 \neq 0,
  \Ln(z_1z_2) = \Ln z_1 + \Ln z_2,\\
  \Ln\parenthesis{\dfrac{z_1}{z_2}} = \Ln z_1 - \Ln z_2.\\
\end{align*}
В общем случае имеем
\begin{align*}
  \Ln 1 = 2\pi ki, k \in \mathbb{Z}, k = 0 \Rightarrow \Ln 1 = 0.\\
  \text{ Аналогично } \Ln(-1) = i\pi(2k + 1), k \in \mathbb{Z}.
\end{align*}
Используя экспоненциальную и логарифмическую ФКП, по аналогии с основным логарифмическим тождеством
для действительных функций, в общем случае для $a, b \in \mathbb{C}, a \neq 0$ полагают
$a^b = e^{b\Ln a}$. На основании этого определения общая степенная функция
$z^{\alpha} \overset{z \neq 0}{=} e^{\alpha \Ln z}$ и общая показательная функция $a^z = e^{z\Ln a}$, которая, вообще говоря, многозначна.
