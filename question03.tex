\begin{col-answer-preambule}
	\begin{plan}
		\item Формулировка: из названия (как и Дирихле для рядов).
		\item Доказательство:
		\subitem оценка Абеля, взятая c 2-кой для надёжности.
		\subitem оценить $\abs{b_{n+1}}$ и $\abs{b_{n+m}}$ по $\widetilde{\varepsilon} = \dfrac{\varepsilon}{6 \cdot c}$
		\subitem def равномерной сходимости для $\sum a_n(x) b_n(x)$.
		\item Замечания: как и для рядов ($\sum (-1)^n b_n(x)	\overset{X}{\rightrightarrows}$, $\text{Лейбница} \approx \text{единица}$).
	\end{plan}
\end{col-answer-preambule}
\colquestion{Признак Дирихле равномерной сходимости ФР и следствие из него (признак Лейбница равномерной сходимости ФР)}
\begin{theorem}[Признак Дирихле равномерной сходимости ФР] Пусть для ФП $a_n(x)$ частичные суммы $\sum a_n(x)$ ограничены в совокупности (равномерно на $X$), т.е.
	\begin{equation}
	\label{eq:1_24}
	\text{для }\forall \; x \in X \text{ и для } \forall \; n \in \mathbb{N} \Rightarrow \abs{a_1 (x) + a_2(x) + \ldots + a_n(x)} \leqslant c,
	\end{equation}
	где $c = const > 0$, \important{не зависит ни от $n$, ни от $x$}. Если $\forall \; fix \; x \in X \Rightarrow \left( b_n(x) \right)$ - числовая последовательность является монотонной, то в случае
	\begin{equation}
	\label{eq:1_25}
	\left( b_n(x) \right) \overset{X}{\rightrightarrows} 0,
	\end{equation}
	имеем $\sum a_n(x) b_n(x) \overset{X}{\rightrightarrows}$.
\end{theorem}
\begin{proof}
	Монотонная последовательность $\left( b_n(x) \right) \text{ для } \forall \; fix \; x \in X$ позволяет так же, как и в ЧР, использовать на основе \eqref{eq:1_24} оценку Абеля:
	\begin{equation}
	\label{eq:1_26}
	\abs{\sum_{k = n+1}^{n+m} a_k(x) b_k(x)} \leqslant 2 \text{\important{c}} \left(\abs{b_{n+1}(x)} + 2 \abs{b_{n+m} (x)}\right).
	\end{equation}

	Если выполняется \eqref{eq:1_25}, то тогда имеем:
	\begin{equation*}
	\text{ для } \forall \; \varepsilon > 0 \text{ по } \tilde{\varepsilon} = \dfrac{\varepsilon}{6 c} > 0 \; \exists \; \nu (\varepsilon) \in \mathbb{R} \; | \; \text{для } \forall \; n \geqslant \nu(\varepsilon) \text{ и для } \forall \; m \in \mathbb{N} \text{ и для } \forall \; x \in X \Rightarrow \abs{b_{n+1} (x) } \leqslant \tilde{\varepsilon} \text{ и } \abs{b_{n+m} (x)} \leqslant \tilde{\varepsilon}, 
	\end{equation*}
	поэтому для частичных сумм $S_n(x) = \sum\limits_{k=1}^{n} a_k(x) b_k(x)$ в силу \eqref{eq:1_26} $\text{для } \forall \; n \geqslant \nu(\varepsilon) \text{ и для } \forall \; m \in \mathbb{N} \text{ и для } \forall \; x \in X$ имеем: $\abs{S_{n+m} (x)  - S_n(x)} = \abs{\sum\limits_{k = n+1}^{n+m} a_k(m) b_k(x)} \leqslant 2 \cdot c \cdot(\tilde{\varepsilon} 	+ 2 \tilde{\varepsilon} ) = 6\cdot c \cdot  \tilde{\varepsilon} = \varepsilon$. Отсюда по критерию Коши равномерной сходимости ФР следует, что $\sum\limits a_n(x) b_n(x) \overset{X}{\rightrightarrows}$.
\end{proof}
\begin{consequence}[Признак Лейбница равномерной сходимости ФР]
	Если $\forall \; fix \; x \in X$ последовательность $\left(b_n(x)\right)$ является монотонной, то в случае $b_n(x) \overset{X}{\rightrightarrows}
	0 \Rightarrow 	\sum (-1)^n b_n(x)	\overset{X}{\rightrightarrows}$.
\end{consequence}
\begin{proof}
	Следует из того, что в условии теоремы $a_n = (-1)^n$ не зависит от $x$, причём \newline $\abs{\sum\limits_{k=1}^{n} a_k} \leqslant 1 = const, \text{для } \forall \; n \in \mathbb{N}$.
\end{proof}
