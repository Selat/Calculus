\begin{col-answer-preambule}
	Функцию $\phi(x)$, определённую на $X$ будем называть \important{равномерным частным пределом} $f(x,y)$ при $y \to y_0$, если 
	\begin{equation}
	\label{eq:lecture04-04}
	\text{для } \forall \; \varepsilon > 0 \; \exists \; \delta (\varepsilon) > 0 \; | \; \text{для } \forall \; x \in X \text{ и для } \forall \; y \in Y \text{ из } 0 < |y-y_0| \leqslant \delta \text{ следует } |f(x,y) - \phi(x)| \leqslant \varepsilon.
	\end{equation}
	В этом случае будем писать
	\begin{equation}
	\label{eq:lecture04-05}
	f(x,y) \overset{X}{\underset{y \to y_0}{\rightrightarrows}} \phi(x).
	\end{equation}
\end{col-answer-preambule}

\colquestion{Признак существования равномерного частного предела для непрерывных Ф2П.}
\begin{theorem}[признак равномерной сходимости Ф2П]
	Если функция $f(x,y)$ непрерывна на прямоугольнике $[a,b] \times [c,d]$, являющимся компактом в $\mathbb{R}^2$, и $y_0 \in [c,d]$, то имеем:
	\begin{equation}
	\label{eq:lecture04-08}
	f(x,y) \xrightarrow[y \to y_0, y \in [c,d\text{]}]{[a,b]} f(x, y_0).
	\end{equation}
\end{theorem}
\begin{proof}
	Из \important{теоремы} Кантора для ФНП получаем, что рассматриваемая $f(x,y)$ будет равномерно непрерывна для $\forall \; x \in [a,b]$ и для $\forall \; y \in [c,d]$, т.е.:
	\begin{equation*}
	\begin{split}
	&\text{ для } \forall \; \varepsilon > 0 \; \exists \; \delta = \delta(\varepsilon) > 0 : \text{для }\forall \; \widetilde{x}, \bar{x} \in [a,b] \text{ и для } \forall \; \widetilde{y}, \bar{y} \in Y \\
	& \text{ из } \begin{cases}
	0 < \abs{\bar{x} - \widetilde{x}} \leqslant \delta, \\ 0 < \abs{\bar{y} - \widetilde{y}} \leqslant \delta.
	\end{cases} \Rightarrow \abs{f(\widetilde{x}, \widetilde{y}) - f(\bar{x}, \bar{y})} \leqslant \varepsilon.
	\end{split}
	\end{equation*}
	
	Полагая здесь: $\begin{cases}
	\widetilde{x} = \bar{x} = x \in [a,b],\\ \widetilde{y} = y \in [c,d], \\ \bar{y} = y_0 \in [c,d].
	\end{cases}$, получаем:
	\begin{equation*}
	\begin{split}
	& \text{ для } \forall \; \varepsilon > 0 \; \exists \; \delta = \delta (\varepsilon) > 0: \text{для } \forall \; y \in [c,d] \text{ из } \abs{y - y_0}  \leqslant \delta(\varepsilon),  \text{ для } \forall \; x \in [a,b] \Rightarrow \\
	& \Rightarrow \abs{f(x, y) - f(x, y_0)} \leqslant \varepsilon.
	\end{split}
	\end{equation*}
	Т.к. здесь $\delta = \delta(\varepsilon) > 0$ не зависит от $x \in [a,b]$, то получаем \eqref{eq:lecture04-05}, где $\phi(x) = f(x, y_0)$, что соответствует \eqref{eq:lecture04-08}.
\end{proof}