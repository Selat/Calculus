\begin{col-answer-preambule}
	Обозначение поточечной сходимости ФП:
	\begin{equation}
	\label{eq:lecture01-05}
	f_n(x) \overset{X}{\rightarrow}f(x) \text{ или } f_n(x) \overset{X}{\rightarrow}.
	\end{equation}
	Определение \eqref{eq:lecture01-05} на $(\varepsilon-\delta)$-языке:
	\begin{equation}
	\label{eq:lecture01-06}
	\text{для } \forall \; \varepsilon > 0 \text{ и }
	\text{для } \forall \; fix \; x \in X \; \exists \; \nu = \nu(x, \varepsilon) \in \mathbb{R}\; | \text{ для } \forall \; n \geqslant \nu \Rightarrow \abs{f_n(x) - f(x)} \leqslant \varepsilon.
	\end{equation}
	Обозначение равномерной сходимости ФП:
	\begin{equation}
	\label{eq:lecture01-07}
	f_n(x) \overset{X}{\rightrightarrows}f(x) \text{ или } f_n(x) \overset{X}{\rightrightarrows}.
	\end{equation}
	Определение \eqref{eq:lecture01-07} на $(\varepsilon-\delta)$-языке:
	\begin{equation}
	\label{eq:lecture01-06fixed}
	\text{для } \forall \; \varepsilon > 0 \; \exists \; \nu = \nu(\varepsilon) \in \mathbb{R}\; | \text{ для } \forall \; fix \; x \in X  \text{ и } \text{для } \forall \; n \geqslant \nu\Rightarrow \abs{f_n(x) - f(x)} \leqslant \varepsilon.
	\end{equation}
	\begin{plan}
		\item Формулировка: $+$.
		\item Доказательство:
		\subitem \circled{$\Rightarrow$}: $\abs{f_n(x) - f(x)} \leqslant \varepsilon \Rightarrow r_n = \sup\limits_{x \in X} \abs{f_n(x) - f(x)} \leqslant \varepsilon \Rightarrow 0 \leqslant r_n \leqslant \varepsilon, \text{ т.е. }
		r_n \xrightarrow[n \to \infty]{}0$.
		\subitem \circled{$\Leftarrow$}: написать \eqref{eq:lecture01-06fixed}, вписав $r_n \text{ т.е. } \abs{f_n(x) - f(x)} \leqslant \sup\limits_{x \in X}\abs{f_n(x) - f(x)} = r_n \leqslant
		\varepsilon$.
		\item Замечания: достаточные условия равномерной (неравномерной) сходимости ФП.
		\subitem $\abs{f_n(x) - f(x)} \leqslant a_n$, где $\left(a_n\right)$ - б.м.п
		\subitem $\exists \; x_n \in X \; \,\vert\,\; g_n(x) = \abs{f_n(x) - f(x)} \Rightarrow g_n(x_n) \centernot{
			\xrightarrow[n \to \infty]{}} 0$.
	\end{plan}
\end{col-answer-preambule}

\colquestion{Супремальный критерий равномерной сходимости функциональных последовательностей (ФП) и замечания к нему}
\begin{theorem}[Супремальный критерий равномерной сходимости ФП]
	\begin{equation}
	\label{eq:lecture01-09}
	f_n(x) \overset{X}{\rightrightarrows}f(x) \Leftrightarrow
	r_n = \sup_{x \in X}\abs{f_n(x) - f(x)} \xrightarrow[n \to \infty]{}0.
	\end{equation}
\end{theorem}
\begin{proof}
	\circled{$\Rightarrow$} Если выполнена \eqref{eq:lecture01-07}, то, учитывая, что в \eqref{eq:lecture01-06fixed} используется $\forall \; fix \; x		 \in X$ и $\forall \; n \geqslant \nu(\varepsilon)$, получаем
	\begin{equation*}
	\begin{split}
	&r_n = \sup_{x \in X}\abs{f_n(x) - f(x)} \leqslant \varepsilon, \text{ т.е. }\\
	&\text{для }\forall \; \varepsilon > 0 \; \exists \; \nu = \nu(\varepsilon) \in \mathbb{R} \; \,\vert\, \text{ для }\forall \;
	n \geqslant \nu \Rightarrow 0 \leqslant r_n \leqslant \varepsilon, \text{ т.е. }
	r_n \xrightarrow[n \to \infty]{}0.
	\end{split}
	\end{equation*}\\
	\circled{$\Leftarrow$}
	Пусть выполнена правая часть \eqref{eq:lecture01-09}, тогда
	\begin{equation*}
	\begin{split}
	& \text{для }\forall \; \varepsilon > 0 \; \exists \; \nu = \nu(\varepsilon) \in \mathbb{R} \; \,\vert\, \text{ для }\forall \; n
	\geqslant \nu \text{ и для } \forall \; x \in X \Rightarrow \\
	& \Rightarrow \abs{f_n(x) - f(x)} \leqslant \sup\limits_{x \in X}\abs{f_n(x) - f(x)} = r_n \leqslant
	\varepsilon.
	\end{split}
	\end{equation*}
	Таким образом, имеем \eqref{eq:lecture01-06fixed}, где $\nu$ зависит от $\forall \; \varepsilon > 0$ и
	не зависит от конкретного элемента множества $X$.
\end{proof}

\begin{notes}
	\item Если известно, что для $\forall \; n \in \mathbb{N}$ и для $\forall \; x \in X \Rightarrow
	\abs{f_n(x) - f(x)} \leqslant a_n$, где $\left(a_n\right)$ - б.м.п, то тогда имеем \eqref{eq:lecture01-07}.
	Сформулированное утверждение даёт \important{мажоритарный признак} (достаточное условие)
	равномерной сходимости ФП.
	\item Если
	\begin{equation*}
	\exists \; x_n \in X \; \,\vert\,\; g_n(x) = \abs{f_n(x) - f(x)} \Rightarrow g_n(x_n) \centernot{
		\xrightarrow[n \to \infty]{}} 0,
	\end{equation*}
	то тогда равномерной сходимости нет, т.е. $f_n(x) \centernot{\overset{X}{\rightrightarrows}} f(x)$. Это
	даёт достаточное условие (признак) неравномерной сходимости ФП.
\end{notes}