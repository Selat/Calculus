\begin{col-answer-preambule}
\begin{equation}
\label{eq:lecture01-06}
\forall \fix x \in X, \forall \varepsilon > 0 \Rightarrow \exists \nu = \nu(x, \varepsilon) \in \R{}
\vert \forall n \geqslant \nu \Rightarrow \abs{f_n(x) - f(x)} \leqslant \varepsilon.
\end{equation}
\begin{equation}
\label{eq:lecture01-07}
f_n(x) \xrightarrow[\to]{x}f(x)
\end{equation}
\end{col-answer-preambule}

\colquestion{Супремальный критерий равномерной сходимости функциональных последовательностей (ФП) и замечания к нему}
\begin{theorem}[Супремальный критерий равномерной сходимости ФП]
	\begin{equation}
		\label{eq:lecture01-09}
		f_n(x) \xrightarrow[\to]{x}f(x) \Leftrightarrow
		r_n = \sum\limits_{x \in X}\abs{f_n(x) - f(x)} \xrightarrow[n \to \infty]{}0
	\end{equation}
\end{theorem}
\begin{proof}
	\circled{$\Rightarrow$} Если выполнено \eqref{eq:lecture01-07}, то, учитывая, что в
	\eqref{eq:lecture01-06} используется
	$\forall n \geqslant \nu(\varepsilon)$, $\forall x \in X$, получим
	\begin{equation*}
		\begin{split}
			&r_n = \sup_{x \in \mathbb{R}}\abs{f_n(x) - f(x)} \leqslant \varepsilon, \text{ т.е. }\\
			&\forall \varepsilon > 0, \exists \nu(\varepsilon) \in \mathbb{R} \,\vert\, \forall
			n \geqslant \nu \Rightarrow 0 \leqslant r_n \leqslant \varepsilon, \text{ т.е. }
			r_n \xrightarrow[n \to \infty]{}0.
		\end{split}
	\end{equation*}\\
	\circled{$\Leftarrow$}
	Пусть выполнено \eqref{eq:lecture01-09}, тогда
	\begin{equation*}
		\forall \varepsilon > 0 \exists \nu(\varepsilon) \in \mathbb{R} \,\vert\, \forall n
		\geqslant \nu \land \forall x \in X \Rightarrow
		\abs{f_n(x) - f(x)} \leqslant \sup\limits_{x \in X}\abs{f_n(x) - f(x)} = r_n \leqslant
		\varepsilon.
	\end{equation*}
	Таким образом, имеем \eqref{eq:lecture01-06}, где $\nu$ зависит от $\forall \varepsilon > 0$ и
	не зависит от конкретного элемента множества $X$.
\end{proof}

\begin{notes}
  \item Если известно, что $\forall n \in \mathbb{N}$ и $\forall x \in X \Rightarrow
\abs{f_n(x) - f(x)} \leqslant a_n$, где $(a_n)$ - б.м.п, то тогда имеем \eqref{eq:lecture01-07}.
Сформулированное утверждение даёт \important{мажоритарный признак} (достаточное условие)
равномерной сходимости ФП.
\item Если
  \begin{equation*}
	  \exists x_n \in X \,\vert\, g_n(x) = \abs{f_n(x) - f(x)} \Rightarrow g_n(x) \centernot{
	  \xrightarrow[n \to \infty]{}} 0,
  \end{equation*}
  то тогда равномерной сходимости нет, т.е. $f_n(x) \centernot{\xrightarrow[\to]{x}} f(x)$. Это
  даёт достаточное условие (признак) неравномерной сходимости ФП.
\end{notes}
