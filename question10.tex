\begin{col-answer-preambule}
\end{col-answer-preambule}

\colquestion{Формула Коши для вычисления радиуса сходимости СтР и замечания к ней.}
\begin{plan}
\item Рассмотрим  $x \neq x_0$
\item Применяем теорему Коши для ЧР
\item Рассматриваем два случая: $k < 1$ и $k > 1$.
\end{plan}
\begin{theorem}[формула Коши для вычисления радиуса сходимости СтР]
	Если существует конечный или бесконечный предел
	\begin{equation}
	\label{2.07}
	\limninf \sqrt[n]{\abs{a_n}},
	\end{equation}
	то для радиуса сходимости ряда \eqref{2.01} имеем:
	\begin{equation}
	\label{2.08}
	R = \dfrac{1}{\limninf \sqrt[n]{\abs{a_n}}}.
	\end{equation}
\end{theorem}
\begin{proofUndotted} проведём по той же схеме, что и в предыдущей теореме.

	Т.к. случай $ x=x_0 $ тривиален (в данной точке ряд всегда сходится), то рассмотрим случай $ x \ne x_0 $.

	По признаку Коши сходимости ЧР для \eqref{2.01} получаем:
	\begin{equation*}
	\exists \; k = \limninf \sqrt[n]{\abs{a_n (x-x_0)^n}} =
	\abs{x-x_0} \limninf \sqrt[n]{\abs{a_n}} \overset{\eqref{2.08}}{=} \dfrac{\abs{x-x_0}}{R}.
	\end{equation*}

	Если $ k < 1 $, т. е. $ \abs{x - x_0} < R $, то СтР \eqref{2.01} сходится.

	Если $ k > 1 $, т. е. $ \abs{x - x_0} > R $, то СтР \eqref{2.01} расходится.


	Таким образом, в силу определения, величина \eqref{2.08} будет радиусом сходимости для \eqref{2.01}.
\end{proofUndotted}

\begin{notes}
	\item В силу связи между признаками Даламбера и Коши сходимости ЧР, в случае, когда предел \eqref{2.06} не существует (ни конечный, ни бесконечный),
	предел \eqref{2.08} может существовать, и в этом смысле формула Коши \eqref{2.08} предпочтительнее, чем \eqref{2.06}.

	\item Можно показать, что в случае, когда в \eqref{2.08} нет ни конечного, ни бесконечного предела, радиус сходимости для \eqref{2.01} всегда можно вычислить по
	\textbf{формуле Коши-Адамара}, использующей понятие верхнего предела последовательности:
	\begin{equation}
	\label{2.09}
	R = \dfrac{1}{\;\; \overline{\limninf} \sqrt[n]{\abs{a_n}} \;\;}.
	\end{equation}
	Под верхним пределом последовательности подразумевается верхняя грань (supremum) множества конечных пределов всех сходящихся подпоследовательностей рассматриваемой последовательности.
\end{notes}
