\colquestion{И.Ф. для чётных и нечётных функций. Синус- и косинус-преобразование Фурье.}
%% \begin{plan}
%% \item $z = x + iy, \omega = u + iv$. Выражаем $u, v$.
%% \end{plan}

\begin{col-answer-preambule}
\end{col-answer-preambule}

Пусть $f(x)$ чётна на $\R{}$, тогда
\begin{equation}
\label{eq:lecture27-17}
a(y) = \dfrac{1}{\pi}\intl_{-\infty}^{+\infty}f(t)\cos ytdt =
\dfrac{2}{\pi}\intl_0^{+\infty}f(t)\cos ytdt
\end{equation}
\begin{equation}
b(y) = \dfrac{1}{\pi}\intl_{-\infty}^{+\infty}f(t)\sin ytdt = 0.
\end{equation}

При выполнении соответствующих условий в силу теоремы о поточечной сходимости ИФ и в случае
непрерывности $f(x)$ имеем
\begin{equation}
\label{eq:lecture27-18}
f(x) = \Phi(x) = \intl_0^{+\infty}a(y)\cos xydy.
\end{equation}
В соответствии с \eqref{eq:lecture27-17} - \eqref{eq:lecture27-18} для $f(x)$, определённой для
$x \in \interval]{0; +\infty}[$ её $\cos$-преобразованием Фурье называется
\begin{equation}
\label{eq:lecture27-19}
\Phi_c(y) = \sqrt{\dfrac{2}{\pi}}\intl_0^{+\infty}f(t)\cos ytdt,
\end{equation}
а тогда в силу \eqref{eq:lecture27-18} сама $f(x)$ будет восстанавливаема на $x > 0$ по
своему $\cos$-преобразованию \eqref{eq:lecture27-20} по формуле
\begin{equation}
\label{eq:lecture27-20}
f(x) = \sqrt{\dfrac{2}{\pi}}\intl_0^{+\infty}\Phi_c(y)\cos xydy.
\end{equation}

Аналогично, если $f(x)$ - нечётна на $\R{}$, то
\begin{align}
\label{eq:lecture27-21}
&a(y) = \dfrac{1}{\pi}\intl_{-\infty}^{+\infty}f(t)\cos ytdt = 0,\\
&b(y) = \dfrac{1}{\pi}\intl_{-\infty}^{+\infty}f(t)\sin ytdt =
\dfrac{2}{\pi}\intl_0^{+\infty}f(t)\sin ytdt.\\
\end{align}
Отсюда, при выполнении соответствующего условия на непрерывность $f(x)$ получаем
\begin{equation}
\label{eq:lecture27-22}
f(x) = \Phi(x) = \intl_0^{+\infty}b(y)\sin xydy.
\end{equation}
В соответствии с \eqref{eq:lecture27-21} - \eqref{eq:lecture27-22} для $f(x)$, определённой
для $x > 0$ её $\sin$-преобразованием Фурье называется
\begin{equation}
\label{eq:lecture27-23}
\Phi_x(y) = \sqrt{\dfrac{2}{\pi}}\intl_0^{+\infty}f(t)\sin ytdt,
\end{equation}
а сама $f(x)$ для $x > 0$ в силу \eqref{eq:lecture27-21} - \eqref{eq:lecture27-22}
восстанавливается по формуле
\begin{equation}
\label{eq:lecture27-24}
f(x) = \sqrt{\dfrac{2}{\pi}}\intl_0^{+\infty}\Phi_s(y)\sin xydy,
\end{equation}
В этом случае для $f(x)$, определённой для $x > 0$ строим их чётные (нечётные) продолжения
\begin{align*}
f_{\text{чн}}(x) = f(\abs{x}), x \in \R{},\\
f_{\text{неч}}(x) = f(\abs{x})\sgn x, x \in \R{},
\end{align*}
а полученные для этих функций интегралы Фурье назовём соответствующими интегралами Фурье по
$\cos$ и $\sin$.

Так же, как и общее прФ, $\sin$ и $\cos$ прФ могут использоваться для вычисления интегралов,
решения дифференциальных и интегральных уравнений.
