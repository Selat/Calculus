\begin{col-answer-preambule}
\end{col-answer-preambule}

\colquestion{Теорема о дифференцировании СтР, замечания и следствие из неё.}
\begin{plan}
\item Слагаемые - непрерывно дифференцируемы + имеем поточечную сходимость СтР, поэтому сумма СтР будет непрерывно дифференцируемой.
\item Считаем радиус по обобщённой теореме Коши (формула Коши-Адамара)
\end{plan}
Следствие

\begin{plan}
\item. Просто дифференцируем и замечаем схожесть с рядом Тейлора.
\end{plan}
\begin{theorem}[о дифференцировании СтР]
	Сумма СтР \eqref{2.01} внутри его интервала сходимости является непрерывно дифференцируемой функцией, причём у продифференцированного СтР
	будет тот же радиус (а, значит, и интервал) сходимости, что и у исходного ряда \eqref{2.01}.
\end{theorem}
\begin{proof}
	По теореме о почленном дифференцировании ФР и замечанию к ней достаточно показать, что возможно почленное дифференцирование \eqref{2.01} на $\forall$ отрезке
	$ [a,b] \subset  I = \interval]{ \; x_0-R \;;\; x_0 + R \;}[ $.
	\begin{enumerate}
		\item В \eqref{2.01} слагаемые $ u_n(x) = a_n (x-x_0)^n $, $ n \in \mathbb{N}_0 $ являются непрерывно дифференцируемыми функциями для $ \forall \; x \in [a; b]$
		т.к. $ \exists \; u_n'(x) = n a_n (x-x_0)^{n-1} $ непрерывная на $ [a;b] $.
		\item Так как $\forall$ СтР \eqref{2.01} сходится поточечно внутри своего интервала сходимости, то \linebreak
		$ \sumnzi u_n(x) \xrightarrow[]{\text{для } \forall \; {[a;b]} \subset I} S(x) $.
	\end{enumerate}
	Осталось показать, что продифференцированный СтР
	\begin{equation*}
	\sumnzi u_n'(x) = \sumnzi n a_n (x-x_0)^{n-1} = \sum_{n=1}^{\infty} n a_n (x-x_0)^{n-1}
	= \sumnzi (n+1) a_{n+1} (x-x_0)^n \overset{[a; b]}{\rightrightarrows}.
	\end{equation*}
	Используя \important{формулу Коши-Адамара}, имеем:
	\begin{align*}
	& \overset{\sim}{R} =
	\dfrac{1}{\;\; \overline{\limninf} \sqrt[n]{(n+1)\abs{a_{n+1}}} \;\;} =
	\dfrac{1}{\;\; \overline{\limninf} \left(\sqrt[n]{n+1} \sqrt[n]{\abs{a_{n+1}}}\right) \;\;} =
	\begin{sqcases}
	\sqrt[n]{n+1} \xrightarrow[n \to\infty]{} 1, \\
	\sqrt[n]{\abs{a_{n+1}}} = \left(\sqrt[n+1]{\abs{a_{n+1}}}\;\;\right)^{\frac{n+1}{n}}
	\end{sqcases} = \\
	& =  \dfrac{1}{\;\; \overline{\limninf} \left(\sqrt[n+1]{\abs{a_{n+1}}}\;\; \right)^{\frac{n+1}{n}} \;\;}   =
	\begin{sqcases}
	\frac{n+1}{n} \xrightarrow[n \to \infty]{} 1, \\\\
	\overline{\limninf} \sqrt[n+1]{\abs{a_{n+1}}} = \frac{1}{R}
	\end{sqcases}
	= \dfrac{1}{ \;\; \frac{1}{R} \;\;} = R.
	\end{align*}

	Значит, у исходного и продифференцированного рядов один и тот же радиус, а, значит, и интервал, сходимости.
	Тогда, в силу того, что $\forall$ СтР сходится локально равномерно, получаем, что
	$ \sumnzi u_n'(x) \overset{[a; b]}{\rightrightarrows} S'(x) $.


	Причём, в силу непрерывности слагаемых, S(x) будет непрерывно дифференцируема на $ \forall \; [a; b] \subset I$, а, значит, и для $ \forall \; x \in I $.
\end{proof}

$  $

\begin{notes}
	\item Применяя последовательно дифференцирование к СтР \eqref{2.01}, получим по ММИ, что сумма ряда \eqref{2.01} будет бесконечное число раз дифференцируемой функцией.

	\item Можно показать, что дифференцирование СтР хоть и сохраняет интервал сходимости, но в общем случае \important{не улучшает} его множество сходимости в том смысле, что если, например,
	исходный ряд \eqref{2.01} сходится на каком-то из концов интервала I $ (x=x_0 \pm R) $,
	то продифференцированный ряд уже может расходиться на этом конце.
\end{notes}

\begin{consequence}
	Если на интервале $ I = \interval]{ \; x_0-R \;;\; x_0 + R \;}[ $  бесконечно дифференцируемая функция $ f(x) $ представляется в виде
	$ f(x) = \sumnzi a_n(x-x_0)^n, \text{для }\forall\; x \in I $,
	то для неё СтР \eqref{2.01} будет являться соответствующим рядом Тейлора в окрестности точки $ x_0 $, т. е. для $ \forall \; a_n = \dfrac{f^{(n)} (x_0)}{n!}$,  $n \in \mathbb{N}_0 $.
\end{consequence}
\begin{proof}
	Действительно, дифференцируя почленно $ n $ раз равенство
	\begin{equation*}
	f(x) = a_0 + a_1 (x-x_0) + \ldots + a_n(x-x_0)^n + \ldots
	\end{equation*}
	в силу доказанной теоремы получим:
	\begin{equation*}
	\exists \; f^{(n)}(x) = n! \cdot a_n + (n+1) \cdot n \cdot \ldots \cdot 2 \cdot a_{n+1} (x-x_0) + \ldots
	\end{equation*}
	Отсюда при $ x \to x_0 $ имеем:
	\begin{equation*}
	n! \cdot a_n = \lim\limits_{x \to x_0} f^{(n)} (x) = f^{(n)} (x_0) \;\;\;\;\;
	\Leftrightarrow \;\;\;\;\; a_n =  \dfrac{f^{(n)} (x_0)}{n!},
	\end{equation*}
	т.е. $ \forall \; a_n$ - коэффициент в разложении в ряд Тейлора.
\end{proof}
