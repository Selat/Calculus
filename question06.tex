\colquestion{Теорема о почленном интегрировании равномерно сходящегося ФР}
\begin{plan}
\item Очевидно, что $S(x)$ - непрерывна, поэтому интегрируема
\item Рассмотрим частичные суммы $T_n = \suml_{k=1}^{n}\int u_k(x)dx$.
\item Рассмотрим разницу $\abs{T_n - \intl_a^bS(x)}$ и т.к. $\abs{S(x) - S_n(x)} \leqslant \varepsilon$ получим $\intl_a^b (S(x) - S_n(x)) \leqslant M \varepsilon$
\item Доказываем по М-лемме о сходимости ЧП.
\end{plan}
\begin{theorem}[о почленном интегрировании равномерно сходящихся ФР]
	Если $\forall \; u_n(x) \in C([a,b]), $ \\ для $n \in \mathbb{N}$, то в случае, когда $\sum u_n(x) \overset{[a,b]}{\rightrightarrows}$, возможно почленное интегрирование этого ряда на $[a,b]$, т.е.
	\begin{equation}
	\label{eq:lecture01-32}
	\exists \dint\limits_a^b S(x)dx = \dint\limits_a^b \left(\sum_{n=1}^{\infty}u_n(x)\right)dx = \sum_{n=1}^{\infty} \dint\limits_a^b u_n(x)dx.
	\end{equation}
\end{theorem}
\begin{proof}
	На основании теоремы о непрерывности суммы равномерно сходящихся ФР получим, что сумма ряда $S(x) = \sum\limits_{n=1}^{\infty}u_n(x)$ будет непрерывна на $[a,b]$, а значит, интегрируема на $[a,b]$.

	Используя частичные суммы для $\sum u_n(x)$, рассмотрим частичные суммы $T_n = \dint\limits_a^b S_n(x)dx  =$ \\$= 	\dint\limits_a^b \sum_{k=1}^{n} u_k(x)dx = \sum\limits_{k=1}^{n} \dint\limits_a^b u_k(x)dx$ для ЧР правой части \eqref{eq:lecture01-32}.

	Требуется доказать, что $\lim\limits_{n \to \infty} T_n = \dint\limits_a^b S(x)dx$.

    Из равномерной сходимости $\sum u_n(x)$ на $[a,b]$ получим, что для $\forall \; \varepsilon > 0 \; \exists \; \nu = \nu(\varepsilon) \; | \; \text{для } \forall \; n \geqslant \nu $ и для $ \forall \; x \in [a,b] \Rightarrow$
	\begin{equation}
	\label{eq:1_33}
	 	\abs{S(x) - S_n(x)} = \abs{\sum_{k = n+1}^{\infty} u_k(x)} \leqslant \varepsilon
	\end{equation}

	Отсюда получаем, что $\abs{\dint\limits_a^b S(x)dx - T_n} = \abs{\dint\limits_a^b S(x)dx - \dint\limits_a^b S_n(x)dx} =  \abs{\dint\limits_a^b (S(x) - S_n(x))dx} \leqslant $ \\ $\leqslant \dint\limits_a^b \abs{S(x) - S_n(x)}dx \leqslant \dint\limits_a^b \varepsilon dx = M \varepsilon$, где $M = b - a = const \geqslant 0$.

	Таким образом, для $\forall \; \varepsilon > 0 \; \exists \; \nu = \nu(\varepsilon) \; | \; \text{ для } \forall \; n \geqslant \nu \Rightarrow \abs{\dint\limits_a^b S(x) dx - T_n} \leqslant M \varepsilon$, поэтому по М-лемме сходимости ЧП следует, что
	\begin{equation*}
    	\exists \lim\limits_{n \to \infty} T_n = \dint\limits_a^b S(x)dx = \dint\limits_a^b \left(\sum\limits_{k=1}^{\infty} u_k(x)\right) dx,
	\end{equation*}
	что равносильно \eqref{eq:lecture01-32}.
\end{proof}
