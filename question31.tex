\colquestion{Формула Дирихле для частных сумм ряда Фурье (Р.Ф.) и замечание к ней.}

\begin{col-answer-preambule}
\end{col-answer-preambule}

\begin{equation}
  \label{eq:lecture06-18}
  a_k = \dfrac{1}{\pi}\intl_{-\pi}^{\pi}f(x)\cos kxdx,
\end{equation}
\begin{equation}
  \label{eq:lecture06-19}
  b_k = \dfrac{1}{\pi}\intl_{-\pi}^{\pi}f(x)\sin kxdx,
\end{equation}

\begin{theorem}[Формула Дирихле для частных сумм ряда Фурье]
  Пусть $f(x) - 2\pi$ периодическая и интегрируемая на $\interval[{-\pi; \pi}]$ функция. Тогда для
  $\forall x_0 \in \R{}$ значение частных сумм ряда Фурье
  \begin{align*}
    S_n(x) = \dfrac{a_0}{2} + \suml_{k =1}^na_k\cos kx + b_k\sin kx, n \in \mathbb{N}_0
  \end{align*}
  для $f(x)$ вычисляются по формуле Дирихле\
  \begin{equation}
    \label{eq:lecture06-24}
    S_n(x_0) = \dfrac{1}{\pi}\intl_{0}^{\pi}(f(x_0 - t) + f(x_0 + t))\dfrac
    {\sin\parenthesis{n + \frac{1}{2}}}{2\sin\frac{t}{2}}dt
  \end{equation}
\end{theorem}
\begin{proof}
  Используя \eqref{eq:lecture06-18}-\eqref{eq:lecture06-19}, имеем
  \begin{align*}
    &S_n(x_0) = \dfrac{1}{2\pi}\intl_{-\pi}^{\pi}f(x)dx + \dfrac{1}{\pi}\suml_{k=1}^n\parenthesis{
      \parenthesis{\intl_{-\pi}^{\pi}f(x)\cos kxdx}\cos kx_0 +
      \parenthesis{\intl_{-\pi}^{\pi}f(x)\sin kxdx}\sin kx_0 } =\\
    &=\dfrac{1}{\pi}\intl_{-\pi}^{\pi}\parenthesis{\dfrac{1}{2} + \suml_{k = 1}^n
      (\cos kx \cos kx_0 + \sin kx \sin kx_0)}f(x)dx =
    \dfrac{1}{\pi}\intl_{-\pi}^{\pi}\parenthesis{\dfrac{1}{2} + \suml_{k = 1}^n
      \cos k(x - x_0)}f(x)dx =\\
    &=\sqcase{
      \dfrac{1}{2} + \suml_{k =1}^n\cos k\phi = \dfrac{\frac{\sin \phi}{2} + \suml_{k =1}^n
        2\sin \frac{\phi}{2}\cos k\phi}{2\sin \frac{\phi}{2}} =
      \dfrac{\frac{\sin \phi}{2} + \suml_{k =1}^n\parenthesis{
          -\sin(k - \frac{1}{2})\phi + \sin(k + \frac{1}{2})\phi}}{2\sin \frac{\phi}{2}} =
      \dfrac{\sin(n + \frac{1}{2})\phi}{2\sin \frac{\phi}{2}},\\
      \phi = x - x_0\vert_{-\pi - x_0}^{\pi - x_0}, dx = d\phi
    }=\\
    &=\dfrac{1}{\pi}\intl_{-\pi-x_0}^{\pi - x_0}f(x_0 + \phi)\parenthesis{\dfrac{
        \sin(n + \frac{1}{2})\phi}{2\sin\frac{\phi}{2}}}d\phi=
    \sqcase{f - 2\pi \text{ периодическая},\\
      T = (\pi - x_0) - (-\pi - x_0) = 2\pi}=
    \dfrac{1}{\pi}\parenthesis{\intl_{-\pi}^0 + \intl_0^{\pi}}=\\
    &=\sqcase{1) t = -\phi\vert_{\pi}^0, d\phi = -dt\\
      2) t = \phi\vert_{0}^{\pi}, d\phi = dt} =
    \dfrac{1}{\pi}\parenthesis{
      -\intl_{\pi}^{0}f(x_0 - t)\parenthesis{\dfrac{
          \sin(n + \frac{1}{2})(-t)}{2\sin\frac{-t}{2}}}dt
      +
      \intl_{0}^{\pi}f(x_0 + t)\parenthesis{\dfrac{
          \sin(n + \frac{1}{2})t}{2\sin\frac{t}{2}}}dt =
    }=\\
    &=\dfrac{1}{\pi}\intl_{0}^{\pi}(f(x_0 - t) + f(x_0 + t))\parenthesis{\dfrac{
        \sin(n + \frac{1}{2})t}{2\sin\frac{t}{2}}}dt.
  \end{align*}
\end{proof}
\begin{notes}
\item Применим \eqref{eq:lecture06-24} к $f(x) = 1, \forall x \in \R{}$. Учитывая, что
  \begin{align*}
    &a_0 = \dfrac{1}{\pi}\intl_{-\pi}^{\pi}dx = 2,
    \forall a_k = \dfrac{1}{\pi}\intl_{-\pi}^{\pi}\cos kxdx = 0,\\
    &\forall b_k = \dfrac{1}{\pi}\intl_{-\pi}^{\pi}\sin kxdx = 0,\text{ получаем}\\
    &\forall S_n(x) = 1, n \in \mathbb{N}_0, 1 = \dfrac{1}{\pi}\intl_{0}^{\pi}2
    \parenthesis{\dfrac{\sin(n + \frac{1}{2})t}{2\sin\frac{t}{2}}}dt, \text{т.е.}
  \end{align*}
  \begin{equation}
    \label{eq:lecture06-25}
    \intl_{0}^{\pi}\dfrac{\sin(n + \frac{1}{2})t}{2\sin\frac{t}{2}}dt = \dfrac{\pi}{2}.
  \end{equation}
\item $\begin{aligned}[t]
  D_n(t) =\dfrac{\sin(n + \frac{1}{2})t}{2\sin\frac{t}{2}}, n \in \mathbb{N}_0 -
  \text{ ядро Дирихле.}
\end{aligned}$
  Для неё в точке $t = 2\pi m, m \in \mathbb{Z}$, где $\sin \frac{t}{2}$ - устранимый разрыв, т.к.
  \begin{align*}
    \exists \liml_{t \to 2\pi m}D_n(t) = \sqcase{\dfrac{0}{0}} = \liml_{t \to 2\pi m}
    \dfrac{(n + \frac{1}{2})\cos(n + \frac{1}{2})t}{\cos \frac{t}{2}} =
    \dfrac{(n + \frac{1}{2})\cos(m(2n + 1)\pi}{\cos \pi m} = \ldots = (n + \frac{1}{2}) \in \R{},
  \end{align*}
  поэтому в LHS формулы \eqref{eq:lecture06-25} имеем СИЗОП, зависящий от параметра
  $n \in \mathbb{N}_0$, а сама формула \eqref{eq:lecture06-25} аналогична интегралу Дирихле.
\end{notes}
