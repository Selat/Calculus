\colquestion{Обобщённое равенство Парсеваля и следствие из него (о почленном интегрировании Р.Ф.).}

\begin{col-answer-preambule}
\end{col-answer-preambule}

\begin{theorem}[Обобщённое равенство Парсеваля]
  Пусть $2\pi$ периодические функции $f, g \in \R{}(\interval[{-\pi; \pi}])$ имеют РФ
  \begin{align}
    f(x) \sim \dfrac{a_0}{2} + \suml_{k = 1}^{\infty}(a_k\cos kx + b_k\sin kx),\\
    g(x) \sim \dfrac{\widetilde{a_0}}{2} + \suml_{k = 1}^{\infty}(\widetilde{a_k}\cos kx
    + \widetilde{b_k}\sin kx),
  \end{align}
  Тогда выполняется обобщённое равнство Парсеваля
  \begin{equation}
    \label{eq:lecture06-33}
    \dfrac{1}{\pi}\intl_{-\pi}^{\pi}f(x)g(x)dx = \dfrac{a_0\widetilde{a_0}}{2} +
    \suml_{k = 1}^{\infty}(a_k\widetilde{a_k} + b_k\widetilde{b_k})
  \end{equation}
\end{theorem}
\begin{proof}
  Учитывая, что $\forall f, g$ выполняется равнствво Парсеваля, то
  \begin{equation}
    \dfrac{\widetilde{a_0}^2}{2} + \suml_{k = 1}^{\infty}(\widetilde{a_k}^2 + \widetilde{b_k}^2) =
    \dfrac{1}{\pi}\intl_{-\pi}^{\pi}g^2(x)dx,
  \end{equation}
  и, учитывая, что
  \begin{align*}
    f(x) \pm g(x) \sim \dfrac{a_0 \pm \widetilde{a_0}}{2} + \suml_{k = 1}^{\infty}
    ((a_k \pm \widetilde{a_k})\cos kx + (b_k \pm \widetilde{b_k})\sin kx),
  \end{align*}
  приходим к равенствам Парсеваля
  \begin{align*}
    \dfrac{1}{\pi}\intl_{-\pi}^{\pi}(f(x) \pm g(x))^2dx =
    \dfrac{(a_0 \pm \widetilde{a_0})^2}{2} + \suml_{k = 1}^{\infty}
    ((a_k \pm \widetilde{a_k})^2 + (b_k \pm \widetilde{b_k})^2),
  \end{align*}
  Беря отдельно уравнения со знаком ``+'' и ``-'' и вычитая из почленно, получим
  \begin{align*}
    &\dfrac{1}{\pi}\intl_{-\pi}^{\pi}f(x)g(x)dx =
    \dfrac{1}{4\pi}\parenthesis{\intl_{-\pi}^{\pi}(f(x) + g(x))^2dx -
      \intl_{-\pi}^{\pi}(f(x) - g(x))^2dx} =\\
    &=\dfrac{1}{4}
    \parenthesis{\dfrac{(a_0 + \widetilde{a_0})^2 - (a_0 - \widetilde{a_0})^2}{2}
      + \suml_{k = 1}^{\infty}\parenthesis{(a_k + \widetilde{a_k})^2 - (a_k - \widetilde{a_k})^2
        + (b_k + \widetilde{b_k})^2 - (b_k - \widetilde{b_k})^2}} =\\&= \ldots =
    \dfrac{a_0\widetilde{a_0}}{2} +
    \suml_{k = 1}^{\infty}(a_k\widetilde{a_k} + b_k\widetilde{b_k})
  \end{align*}
\end{proof}

\begin{consequence}[О почленном интегрировании РФ]
  РФ $2\pi$-периодической функции $f \in \R{}(\interval[{-\pi; \pi}])$ можно почленно интегрировать
  по любому промежутку $\interval[{x_0; x}] \subset \interval[{-\pi; \pi}]$, и при этом будет
  выполнятся равенство
  \begin{equation}
    \label{eq:lecture06-35}
    \intl_{x_0}^xf(t)dt = \dfrac{a_0}{2}\intl_{x_0}^xdx + \suml_{k = 1}^{\infty}
    (a_k\intl_{x_0}^x\cos ktdt + b_k\intl_{x_0}^x \sin ktdt),
  \end{equation}
  независимо от характера сходимости РФ для $f(x)$.
\end{consequence}
\begin{proof}
  Рассмотрим функцию
  \begin{align*}
    g(t) = \begin{cases}
      1, t \in \interval[{x_0; x}] \subset \interval[{-\pi; \pi}],\\
      0, t \in \interval[{-\pi; x_0}[ \cup \interval]{x; \pi}],
    \end{cases}
  \end{align*}
  Для коэффициентов её РФ имеем
  \begin{align*}
    \widetilde{a_0} = \dfrac{1}{\pi}\intl_{-\pi}^{\pi}g(t)dt = \dfrac{1}{\pi}\intl_{x_0}^xdt.
  \end{align*}
  Аналогично
  \begin{align*}
    \widetilde{a_k} = \dfrac{1}{\pi}\intl_{-\pi}^{\pi}g(t)\cos ktdt = \ldots =
    \dfrac{1}{\pi}\intl_{x_0}^x\cos ktdt, \forall k \in \mathbb{N}.
  \end{align*}
  Точно так же
  \begin{align*}
    \widetilde{b_k} = \dfrac{1}{\pi}\intl_{-\pi}^{\pi}g(t)\sin ktdt = \ldots =
    \dfrac{1}{\pi}\intl_{x_0}^x\sin ktdt, \forall k \in \mathbb{N},
  \end{align*}
  поэтому в силу обобщённого равенства Парсеваля \eqref{eq:lecture06-33} получаем
  \begin{align*}
    \intl_{x_0}^xf(t)dt = \ldots = \intl_{-\pi}^{\pi}f(t)g(t)dt =
    \dfrac{a_0\widetilde{a_0}}{2} +
    \suml_{k = 1}^{\infty}(a_k\widetilde{a_k} + b_k\widetilde{b_k}) =
    \dfrac{a_0}{2}\intl_{x_0}^xdt + \suml_{k = 1}^{\infty}(a_k\intl_{x_0}^x\cos kt dt
    + b_k\intl_{x_0}^x\sin kt dt),
  \end{align*}
  что соответствует почленному интегралу РФ для $f(x)$ на $\interval[{x_0; x}] \subset
  \interval[{-\pi; \pi}]$, при этом сам РФ для $f \in \R{}(\interval[{-\pi; \pi}])$ может как
  сходится, так и расходится, а проинтегрированный ряд \eqref{eq:lecture06-35} будет уже
  всегда сходится.
\end{proof}
