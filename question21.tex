\begin{col-answer-preambule}
\end{col-answer-preambule}

\colquestion{Теорема о почленном дифференцировании НИЗОП и замечание к ней.}
    \begin{theorem}[О почленном дифференцировании НИЗОП]
    	Пусть $f(x, y)$ - непрерывна на $\interval[{a; +\infty}[\times\interval[{c; d}]$,
    	$\exists \; f'_y(x, y)$ - непрерывная на $\interval[{a; +\infty}[\times\interval[{c; d}]$.
    	Тогда если
    	\begin{enumerate}
    		\item $\dintl_a^{+\infty}f(x, y)dx \xrightarrow[]{\interval[{a; +\infty}[}$
    		\item $\dintl_a^{+\infty}f'_y(x, y)dx \overset{\interval[{a; +\infty}[}
    		{\rightrightarrows}$,
    	\end{enumerate}
    	то тогда НИЗОП \eqref{eq:lecture04-18} - функция почленно дифференцируема на $\interval[{a; +\infty}[$, и её производная
    	вычисляется по правилу Лейбница:
    	\begin{equation*}
    	\label{eq:lecture04-28}
    	\exists \; F'(y) \overset{\eqref{eq:lecture04-18}}{=}
    	\parenthesis{\intl_a^{+\infty}f(x, y)dx}'_y = \intl_a^{+\infty}f'_y(x, y)dx
    	\end{equation*}
    \end{theorem}
    \begin{proof}
    	Для $\forall \; \fix y \in \interval[{c; d}]$ корректно определяем СИЗОП
    	\begin{equation*}
    	\Phi(y) = \intl_c^y\parenthesis{\intl_a^{+\infty}f'_y(x, t)dx}dt
    	\end{equation*}
    	В силу выполнения всех условий почленного интегрирования СИЗОП можем изменить порядок
    	интегрирования
    	\begin{align*}
    	&\Phi(y) = \intl_a^{+\infty}\parenthesis{\intl_c^yf'_y(x, t)}dx =
    	\intl_a^{+\infty}\sqcase{f(x, t)}_{t = c}^{t = y}dx =
    	\intl_a^{+\infty}\parenthesis{\intl_c^yf(x, y) - f(x, c)}dx =\\
    	&=\intl_a^{+\infty}f(x, y)dx - \intl_a^{+\infty}f(x, c)dx = F(y) - F(c)
    	\end{align*}
    	Отсюда, используя теорему Барроу о дифференцировании интеграла с переменным верхним пределом
    	имеем
    	\begin{align*}
    	&\exists F'(y) = \parenthesis{\Phi(y) + F(c)}'_y =
    	\parenthesis{\intl_c^y\parenthesis{\intl_a^{+\infty}f'_y(x, t)dt}dx}'_y =\\
    	&=\sqcase{\intl_0^{+\infty}f'_y(x, t)dx}_{t = y} = \intl_a^{+\infty}f'_y(x, y)dx \Leftrightarrow
    	\eqref{eq:lecture04-28}
    	\end{align*}
    \end{proof}
    \begin{note}
    	Так же, как и в условии непрерывности НИЗОП в доказательстве теоремы о почленном
    	дифференцировании вместо равномерной сходимости рассмотрим НИЗОП используя локальную
    	равномерную сходимость соответствующего НИЗОП.
    \end{note}