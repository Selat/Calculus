\begin{col-answer-preambule}
\end{col-answer-preambule}

\colquestion{Вторая теорема Фруллани.}
\begin{plan}
\item Раскладываем на два интеграла по лемме Фруллани.
\item По теореме о среднем для ОИ получаем подобие формулы из условия.
\item Переходим к пределу в формуле.
\end{plan}
\begin{statementDotted}{Вторая теорема Фруллани}$  $

	Пусть $ f(x) $ непрерывна для $ \forall \; x \geq 0 $ и $\forall \; A > 0 \Rightarrow$
	$ \exists \dintl_A^{+\infty} \dfrac{f(x)}{x} dx \in \mathbb{R} $ - сходится.

	Тогда:
	\begin{equation}
	\label{5.07}
	\Phi (a, b) \overset{\eqref{5.04}}{=} f(0) \ln \frac{b}{a}.
	\end{equation}
\end{statementDotted}
\begin{proof}
	Действуя как в предыдущей теореме, получим:
	\begin{align*}
	& \Phi (a, b) \overset{\eqref{5.04}}{=} \limlim{\alpha \to + 0}{\beta \to + \infty} \dintl_\alpha^\beta \dfrac{f(ax)-f(bx)}{x} \; dx
	= \ldots
	= \lim\limits_{\alpha \to +0} \;  \dintl_{\alpha a}^{\alpha b} \dfrac{f(t)}{t} dt -
	\lim\limits_{\beta \to + \infty} \; \dintl_{\beta a}^{\beta b} \dfrac{f(t)}{t} dt
	= \\ &
	= \begin{sqcases}
	1) \; \exists \; c \in [\alpha a; \alpha b] \Rightarrow \dintl_{\alpha
		a}^{\alpha b} \dfrac{f(\alpha t)}{t} dt
	= f(c) \dintl_{\alpha a}^{\alpha b} \dfrac{dt}{t} = f(c) \ln \frac{b}{a}
	\;\;\;\;\;\;\;\;\;\;\;\;\;\;\;\;\;\;\;\;\;\;\;\;\;\;\;\;\;\;\;\; \\
	2) \; \dintl_{\beta a}^{\beta b} \dfrac{f(t)}{t} dt
	= \dintl_{A>0}^{\beta b} \dfrac{f(t)}{t} dt \;  -
	\dintl_{A>0}^{\beta a} \dfrac{f(t)}{t} dt
	\xrightarrow[\beta \to + \infty]{}
	\underbrace{\dintl_{A>0}^{+\infty} \dfrac{f(t)}{t} dt}_{\text{сходится}} -
	\underbrace{\dintl_{A>0}^{+\infty} \dfrac{f(t)}{t} dt}_{\text{сходится}}
	= 0
	\end{sqcases}
	= \\ &
	= \lim\limits_{\alpha \to +0} f(c) \ln\dfrac{b}{a}
	= \begin{sqcases}
	\alpha a \leq c \leq \beta b \\
	\alpha \to +0, \;
	c \to 0
	\end{sqcases}
	= f(0) \ln \dfrac{b}{a}.
	\end{align*}
\end{proof}
