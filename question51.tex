\colquestion{Интеграл от аналитической ФКП. Интегральная теорема Коши и замечание к ней.}
%% \begin{plan}
%% \item $z = x + iy, \omega = u + iv$. Выражаем $u, v$.
%% \end{plan}

\begin{col-answer-preambule}
\end{col-answer-preambule}
ФКП $f(z)$ называется аналитической в точке $z_0 \in D(f)$, если
\begin{align*}
  \exists B(z_0) \subset D(f) \vert \forall z \in B(z_0) \Rightarrow f'(z) \in \mathbb{C},
\end{align*}
т.е. $f(z)$ дифференцируема в некоторой окрестности точки $z_0$. Функцию $f(z)$, аналитическую в
любой точке множества $G \subset D(f)$ будем называть аналитической в $G$.

\begin{theorem}[Интегральная теорема Коши]
  Пусть $f(z)$ - аналитическая в односвязной области $G \subset D(f)$, причём $f'(z)$ непрерывна в
  $G$, тогда для произвольного кусочно-непрерывного замкнутого контура
  \begin{equation}
    l \subset G \Rightarrow \oint\limits_lf(z)dz = 0.
  \end{equation}
\end{theorem}
\begin{proof}
  Пусть $u = \operatorname{Re}f(z), v = \operatorname{Im}f(z)$, тогда по формуле вычисления интеграла
  ФКП через КРИ-2 для $I = \oint\limits_lf(z)dz$ имеем: $I = I_1 + iI_2$, где
  \begin{align*}
    &I_1 = \oint\limits_ludx -vdy,\\
    &I_2 = \oint\limits_lvdx + udy.
  \end{align*}
  Из существования $f'(z)$ непрерывной в $G$ следует, что $u = u(x, y)$ и $v = v(x, y)$ непрерывно
  дифференцируемы. Воспользуемся теоремой о независимости КРИ-2 от пути интегрирования:
  \begin{enumerate}
  \item $P = u, Q = -v$. В силу условия Коши-Римана имеем
    \begin{align*}
      P'_y = u'_y = -v'_x = Q'_x \Rightarrow I_1 = \oint\limits_lPdx + Qdy = 0.
    \end{align*}
  \item Аналогично для $P = v, Q = u$ в силу условия Коши-Римана получаем
    \begin{align*}
      P'_y = v'_y = u'_x = Q'_x \Rightarrow I_2 = \oint\limits_lPdx + Qdy = 0.
    \end{align*}
  \end{enumerate}
  Отсюда следует $I = I_1 + iI_2 = 0$.
\end{proof}
\begin{notes}
\item Можно показать, что интегральная формула Коши верна и при менее ограничивающих условиях на
  $f(z)$ - достаточно потребовать лишь дифференцируемости $f(z)$ на $G$, но при этом строгое
  доказательство значительно усложняется.
\item Если $f(z)$ аналитическая в $G$ и непрерывная в $\overline{G} = G \cup \sigma G$, то в случае,
  когда $l = \sigma G$ - кусочно-гладкий контур, интегральная теорема Коши верна и для этого контура,
  т.е. $\displaystyle{\oint\limits_{\sigma G}}f(z)dz = 0$.
\item Интегральная теорема Коши естественным образом обобщается на случай многосвязной области $G$,
  но при этом под границей $l$ для такой многосвязной области $G$ подразумевается её полная граница,
  соответствующим образом ориентированная.
\end{notes}
