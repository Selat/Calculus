\begin{col-answer-preambule}
\end{col-answer-preambule}

\colquestion{Первая теорема Фруллани.}
\begin{plan}
\item Раскладываем на два интеграла по лемме Фруллани.
\item По теореме о среднем для ОИ получаем подобие формулы из условия.
\item Переходим к пределу в формуле.
\end{plan}
\begin{statementDotted}{Первая теорема Фруллани}
	Если $ f(x) $ непрерывна для $ \forall \; x \geq 0 $ и $ \exists f(\infty) \in \mathbb{R} $, то
	\begin{equation}
	\label{5.06}
	\Phi (a, b) = \left(\nullFrac f(0) - f(+\infty) \nullFrac\right) \ln \left(\dfrac{b}{a}\right).
	\end{equation}
\end{statementDotted}
\begin{proof}
	В силу леммы Фруллани для \eqref{5.04}, имеем:
	\begin{align*}
	&\Phi (a, b) = \limlim{\alpha \to + 0}{\beta \to + \infty} \; \dintl_\alpha^\beta
	\dfrac{f(ax)-f(bx)}{x} \; dx \overset{\eqref{5.05}}{=}
	\limlim{\alpha \to +0}{\beta \to + \infty} \; \dintl_a^b
	\dfrac{f(\alpha x)-f(\beta x)}{x} \; dx =
	\\ & =
	\lim\limits_{\alpha \to +0} \; \dintl_a^b \dfrac{f(\alpha x)}{x} \; dx -
	\lim\limits_{\beta \to + \infty} \dintl_a^b  \dfrac{f(\beta x)}{x} \; dx =
	\\ & =
	\begin{sqcases}
	\text{По теореме о среднем для ОИ:} \\
	\nullFrac
	1) \; \exists \; c_1 \in [a;b] \Rightarrow \dintl_a^b \dfrac{f(\alpha x)}{x} dx
	= f(\alpha c_1) \dintl_a^b \dfrac{dx}{x} = f(\alpha c_1) \ln \frac{b}{a}
	\nullFrac \\
	2) \; \exists \; c_2 \in [a;b] \Rightarrow \dintl_a^b \dfrac{f(\beta x)}{x} dx
	= f(\beta c_2) \dintl_a^b \dfrac{dx}{x} = f(\beta c_2) \ln \frac{b}{a}
	\end{sqcases} =
	\limlim{\alpha \to + 0}{\beta \to + \infty} \; \left(\nullFrac f(\alpha c_1) - f(\beta c_2) \nullFrac\right) \ln \dfrac{b}{a} =
	\\ & =
	\begin{sqcases}
	1) \; \alpha a \leq \alpha c_1 \leq \alpha b \Rightarrow
	\sqcase{\alpha \to +0, \; \alpha c_1 \to 0} \Rightarrow f(\alpha c_1) \xrightarrow[\alpha \to +0]{}  f(0)
	\;\;\;\;\;\;\;\;\; \\
	2) \; \beta a \leq \beta c_2 \leq \beta b \Rightarrow
	\sqcase{\beta \to +\infty, \; \beta c_2 \to \infty} \Rightarrow f(\beta c_2) \xrightarrow[\beta \to + \infty]{} f(+\infty)
	\end{sqcases} =
	\left(\nullFrac f(0) - f(+\infty) \nullFrac\right) \ln \dfrac{b}{a}.
	\end{align*}
\end{proof}
