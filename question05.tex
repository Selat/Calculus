\begin{col-answer-preambule}
Пусть $x_0$ - предельная точка множества сходимости $X \subset \mathbb{R}$ для ФР $\sum u_n(x)$.
Будем говорить, что в $\sum u_n(x)$ \important{возможен почленный предельный переход} $x \to x_0$, если
\begin{equation}
\label{eq:lecture01-28}
\exists \; \lim\limits_{x \to x_0}\sum_{n = 1}^{\infty} u_n(x) = \sum_{n = 1}^{\infty} \lim_{x \to x_0} u_n(x),
\end{equation}
причём получившийся в левой части \eqref{eq:lecture01-28} ЧР является сходящимся.

В частности, если $x_0 \in X$ и $\forall \; u_n(x)$ непрерывен в некоторой окрестности точки $x_0$, и значит, \newline для $\forall \; n \in \mathbb{N} \; \exists \lim\limits_{x \to x_0} u_n(x) = u_n(x_0)$, то в случае выполнения \eqref{eq:lecture01-28} для суммы $S(x)$ ФР $\sum u_n(x)$ при $x \to x_0$ имеем:
\begin{equation}
\label{eq:lecture01-29}
\exists \; \lim\limits_{x \to x_0} S(x) = \lim\limits_{x \to x_0} \sum_{n=1}^{\infty} u_n(x) = \sum_{n=1}^{\infty} \lim\limits_{x \to x_0} u_n(x) = \sum_{n=1}^{\infty} u_n(x_0) = S(x_0),
\end{equation}
что соответствует непрерывности $S(x)$ в точке $x_0 \in X$.
\end{col-answer-preambule}

\colquestion{Теорема о непрерывности суммы равномерно сходящегося ФР и замечания к ней}
\begin{col-answer-preambule}
  \begin{plan}
  	\item Формулировка: по названию + каждый член ряда — непрерывная функция.
  	\item Доказательство (\textcolor{magenta}{З}ейдель $= 3 \cdot \varepsilon$):
    \subitem Пишем, что нужно обосновать для $\forall \; x_0 \in X$, при этом нужно использовать односторонние пределы для концевых значений.
    \subitem Рассматриваем приращение суммы $\Delta S(x_0)$.
    \subitem Рассматриваем три разности частичной суммы и полной суммы (с $x_0$ и $x_0 + \Delta x$).
    \subitem Подставляем $3$ разности $(1 + 3 - 2)$ и получаем непрерывность по M-лемме.
  \end{plan}
\end{col-answer-preambule}
\begin{theorem}[о непрерывности суммы равномерно сходящегося ФР]
	Если все члены $u_n(x), n \in \mathbb{N}$, ФР $\sum u_n(x)$ непрерывны на $X = [a,b]$, то в случае равномерной сходимости этого ряда на $[a,b]$ его сумма $S(x)$ будет непрерывной функцией на $[a,b]$.
\end{theorem}
\begin{proof}
	Требуется обосновать \eqref{eq:lecture01-29} для $\forall \; x_0 \in [a,b]$, причём в случае концевых значений $x_0 = a, \; x_0 = b$ будем использовать соответствующие односторонние пределы, т.е. рассматривать одностороннюю непрерывность.

	Для $fix \; x_0 \in [a,b]$ придадим произвольные приращения $\Delta x \in \mathbb{R} \; | \; (x_0 + \Delta x) \in [a,b]$ и рассмотрим соответствующие приращения суммы ФР $\sum u_n(x)$:
    \begin{equation*}
        \Delta S(x_0) = S(x_0 + \Delta x) - S(x_0).
    \end{equation*}

    Из равномерной сходимости ФР $\sum u_n(x)$ на
    $X = [a,b] \Rightarrow \text{для } \forall \; \varepsilon > 0 $
    ${ \exists \; \nu = \nu(\varepsilon) \in \mathbb{R} \; | \; \text{для } \forall \; n \geqslant \nu }$,
    $ \text{ и для } \forall \; x \in [a,b]$ для частичных сумм $S_n(x) = u_1(x) + u_2(x) + \ldots + u_n(x)$ ряда $\sum u_n(x)$ имеем: $\abs{S_n(x) - S(x)} \leqslant \varepsilon$.

	Отсюда, в частности, для $x = x_0 \in X $ и $ x = x_0 + \Delta x \in X \Rightarrow$
	\begin{equation}
	\label{eq:1_30}
	\begin{cases}
	\abs{S_n(x_0) - S(x_0)} \leqslant \varepsilon, \\
	\abs{S_n(x_0 + \Delta x) - S(x_0 + \Delta x)} \leqslant \varepsilon.
	\end{cases}
	\end{equation}

	Далее из непрерывности $\forall \; u_n(x)$ в $x_0 \in [a,b]$ следует непрерывность частичных сумм в $x_0$ (как конечных сумм непрерывных функций).

	В силу этого, для $ \forall \; \varepsilon \; \exists \; \delta > 0 \; : \; \text{для } \forall \; \abs{\Delta x} \leq \delta \Rightarrow$
	\begin{equation}
    	\label{eq:1_31}
    	\Rightarrow \abs{S_n(x_0 + \Delta x) - S_n(x_0)} \leqslant \varepsilon.
	\end{equation}

	Таким образом,  в силу \eqref{eq:1_30}, \eqref{eq:1_31} имеем: для $\forall \varepsilon > 0$, выбирая $n \geqslant \nu$ и рассматривая $\forall \abs{\Delta x}	 \leqslant \delta$, имеем:

	$\abs{\Delta S(x_0)} = \abs{S_n(x_0) - S(x_0) + S_n(x_0 + \Delta x) - S_n(x_0) + S (x_0 + \Delta x) - S_n(x_0 + \Delta x)} \leqslant \\  \leqslant \abs{S_n(x_0) - S(x_0)} + \abs{S_n(x_0 + \Delta x) - S_n(x_0)} + \abs{S(x_0 + \Delta x	) - S_n(x_0 + \Delta x)} \leqslant \varepsilon + \varepsilon + \varepsilon = 3 \cdot \varepsilon$.

    Поэтому получаем:
	для	$\forall \; \varepsilon \; \exists \; \delta > 0 : \; \text{для } \forall \; \abs{\Delta x} \leqslant \delta \Rightarrow \abs{\Delta S(x_0)} \leqslant M \cdot \varepsilon, M = const = 3 > 0$.

	Отсюда по М-лемме для Ф1П следует, что $\Delta S(x_0) \underset{\Delta x \to 0}{\to} 0$, что на языке приращений равносильно \eqref{eq:lecture01-29}. При этом, т.к. из равномерной сходимости следует поточечная сходимость ЧР в правой части \eqref{eq:lecture01-29} будет сходящимся.
\end{proof}

\begin{notes}
	\item Доказанную теорему часто называют теоремой Стокса-Зейделя или теоремой Стокса-Зайделя.
	\item В условии доказанной теоремы равномерную сходимость можно заменить для произвольного множества $ X \subset \mathbb{R}$ на локальную равномерную сходимость.
\end{notes}
