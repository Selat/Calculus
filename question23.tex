\begin{col-answer-preambule}
\important{Интегралами Фруллани} будем называть интегралы вида
\begin{equation}
\label{5.04}
\Phi(a; b)  = \dintlzi \dfrac{f(ax) - f(bx)}{x} dx,
\end{equation}
где $ a, b = \const > 0 $.

В зависимости от свойств подынтегральной функции в \eqref{5.04}, рассмотрим три основные формулы для вычисления интеграла Фруллани. 
Для этого нам понадобится с следующая 
\end{col-answer-preambule}

\colquestion{Лемма Фруллани.}
\begin{lemmaNamed}{Фруллани}
	Если для функции $ f(x) $, определённой для $ \forall \; x > 0 $, функция $ \dfrac{f(x)}{x} $ интегрируема на любом конечном промежутке из $ \interval]{\; 0; +\infty \; }[ $, то
	тогда для $ \forall \; a , b, \alpha, \beta = \const > 0$ верно равенство
	\begin{equation}
	\label{5.05}
	\dintl_\alpha^\beta \dfrac{f(ax)-f(bx)}{x} dx =
	\dintl_a^b \dfrac{f(\alpha x)-f(\beta x)}{x} dx
	\end{equation}    
\end{lemmaNamed}
\begin{proof}$  $
	
	Используя аддитивность интеграла Римана, после соответствующей замены имеем:
	\begin{align*}
	&
	\dintl_\alpha^\beta \dfrac{f(ax)-f(bx)}{x} dx =
	\dintl_\alpha^\beta \dfrac{f(ax)}{x} dx - \dintl_\alpha^\beta \dfrac{f(bx)}{x} dx =    
	\begin{sqcases}
	1) \; t = ax \vert_{\alpha a}^{\beta a} \;  \\
	2) \; t = bx \vert_{\alpha b}^{\beta b} \;
	\end{sqcases} =
	\dintl_{\alpha a}^{\beta a} \dfrac{f(t)}{\frac{t}{a}} \cdot \dfrac{dt}{a} -
	\dintl_{\alpha b}^{\beta b} \dfrac{f(t)}{\frac{t}{b}} \cdot \dfrac{dt}{b} = 
	\\ & =     
	\dintl_{\alpha a}^{\beta a} \dfrac{f(t)}{t} dt -
	\dintl_{\alpha b}^{\beta b} \dfrac{f(t)}{t} dt = 
	\left( \nullFrac \dintl_{\alpha a}^{\alpha b} + \dintl_{\alpha b}^{\beta a} \nullFrac \right) -
	\left( \nullFrac \dintl_{\alpha b}^{\beta a} + \dintl_{\beta a}^{\beta b} \nullFrac \right)         
	= \dintl_{\alpha a}^{\alpha b} \dfrac{f(t)}{t} dt 
	- \dintl_{\beta a}^{\beta b} \dfrac{f(t)}{t} dt =
	\\ & =
	\sqcase{\text{Some comments here :)}} =
	\dintl_a^b \dfrac{f(\alpha x)}{\alpha x} \; \alpha \; dx - 
	\dintl_a^b \dfrac{f(\beta x)}{\beta x} \; \beta \; dx =
	\dintl_a^b \dfrac{f(\alpha x)-f(\beta x)}{x} dx.
	\end{align*}    
\end{proof}
