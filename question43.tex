\colquestion{Дробно-линейная ФКП и её свойства.}
\begin{plan}
\item Любое дробно-линейное преобразование состоит из последовательного выполнения следующих
  операций:

  параллельный перенос -> инверсия->преобразование подобия -> параллельный перенос
\item Композицию двух дробно-линейных преобразований можно представить в виде умножения матриц с
  коэффициентами.
\end{plan}

\begin{col-answer-preambule}
\end{col-answer-preambule}
Рассмотрим отображение $\omega = f(z) = \dfrac{az + b}{cz + d}; a, b, c, d = \const \in \mathbb{C}$,
причём $c \neq 0$ или $d \neq 0$. Если у нас $c = 0, d \neq 0$, то
$\omega = \dfrac{a}{d}z + \dfrac{b}{d}$ - уже рассмотренная линейная ФКП.

Вычислим
\begin{align*}
  \Delta =
  \begin{vmatrix}
    a & b\\
    c & d\\
  \end{vmatrix}
  = ad - bc.
\end{align*}
В случае $\Delta = 0$ получаем пропорциональность строк рассматриваемого определителя, а из этого
следует пропорциональность числителя и знаменателя дробно-линейной функции, в силу чего
$\omega = \const \in \mathbb{C}$, что также было рассмотрено.

Пусть теперь $c \neq 0$ и $\Delta \neq 0$. После элементарных преобразование имеем
\begin{align*}
  \omega = \dfrac{a}{c} - \dfrac{\Delta}{c^2\parenthesis{z + \frac{d}{c}}}.
\end{align*}
В связи в этим дробно-линейную ФКП можно рассматривать как последовательность (композицию)
простых дробно-линейных ФКП:
\begin{enumerate}
\item $\omega_1 = z$ - идентичное отображение.
\item $\omega_2 = \omega_1 + z_0 = z + z_0, z_0 = \dfrac{d}{c}$ - параллельный перенос на
  радиус-вектор, соответствующий $z_0 \in \mathbb{C}$.
\item $\omega_3 = \dfrac{1}{\omega_2} = \dfrac{1}{z + \frac{d}{c}}$ - инверсия (симметрия
  относительно единичной окружности).
\item $\omega_4 = p\omega_3 = -\dfrac{\Delta}{^2(z + \frac{d}{c})}$,
  ${p = -\dfrac{\Delta}{с^2} \in \mathbb{C}}$ - преобразование подобия с помощью растяжения
  (сжатия) и поворота.
\item $\omega_5 = \omega_4 + \omega_0 = \dfrac{a}{c} - \dfrac{\Delta}{c^2(z + \frac{d}{c})}$,
  $\omega_0 = \dfrac{a}{c}$ - параллельный перенос на радиус-вектор, соответствующий числу
  $\omega_0$.
\end{enumerate}
Из указанных преобразований новым является лишь инверсия относительно окружности.

Говорят, что точки $z$ и $\omega$ находятся в инверсии между собой относительно окружности
$\abs{z - z_0} = R$, если $\abs{OZ}\abs{OW} = R^2$. Инверсия также называется симметричной
относительно рассмотренной окружности радиуса $ > 0$. Для нашего преобразования
$\omega_3 = \dfrac{1}{\omega_2}$ имеем
\begin{align*}
  \omega_3\omega_2 = 1 \Rightarrow \abs{\omega_3}\abs{\omega_2} = 1,
\end{align*}
т.е. точки $\omega_2$ и $\omega_3$ симметричны относительно единичной окружности с центром в начале
координат, т.е. находятся в инверсии.

Таким образом, любое дробно-линейное преобразование состоит из последовательного выполнения
параллельного переноса, растяжения (сжатия), поворота и инверсии, при условии, что $c \neq 0$ и
$\Delta \neq 0$. При этих ограничениях можно показать, что обратная к дробно-линейной функция
также будет соответствующей дробно-линейной функцией. Кроме того, при $\Delta \neq 0$ множество
всех дробно-линейных преобразований (в том числе линейных) относительно композиции преобразований
изоморфно множеству невырожденных матриц второго порядка, с комбинацией коэффициентов относительно
их умножения, т.е. для
\begin{align*}
  &\widetilde{\omega} = \dfrac{a_1z + b_1}{c_1z + d_1},
  &\overline{\omega} = \dfrac{a_2z + b_2}{c_2z + d_2}\\
  &\Delta_1 =
  \begin{vmatrix}
    a_1 & b_1\\
    c_1 & d_1\\
  \end{vmatrix} \neq 0,
  &\Delta_2 =
  \begin{vmatrix}
    a_2 & b_2\\
    c_2 & d_2\\
  \end{vmatrix} \neq 0,\\
  &\widetilde{\omega}(\overline{\omega}) = \dfrac{a_1\overline{\omega} + b_1}
                {c_1\overline{\omega} + d_1} = \dfrac{a_0z + b_0}{c_0z + d_0}, &\text{где }\quad
                \begin{bmatrix}
                   a_0 & b_0\\
                   c_0 & d_0\\
                 \end{bmatrix} =
                \begin{bmatrix}
                  a_1 & b_1\\
                  c_1 & d_1\\
                \end{bmatrix}
                \begin{bmatrix}
                  a_2 & b_2\\
                  c_2 & d_2\\
                \end{bmatrix}
\end{align*}

Кроме того, можно показать, что любое дробно-линейное (а следовательно и линейное) преобразование
обладает круговым свойством. При этих преобразованиях окружность в широком смысле слова (т.е. либо
окружность конечного радиуса, либо прямая как окружность бесконечного радиуса) переходит в
окружность в широком смысле слова. При этом при линейном отображении всегда обычная окружность
переходит в обычную окружность, а обычная прямая - в обычную прямую, а при обобщённом
дробно-линейном преобразовании возможны все варианты. Обоснование основывается на том, что
существует единственная дробно-линейная функция, которая 3 различные заданные точки $z_1, z_2, z_3$
плоскости \circled{$z$} переводит в указанном порядке в 3 заданные точки $\omega_1, \omega_2$, $\omega_3$ плоскости \circled{$\omega$}.

Это дробно-линейное преобразование можно найти из соотношения
\begin{equation}
  \label{eq:lecture10-08}
  \dfrac{\omega - \omega_1}{\omega - \omega_2}\cdot\dfrac{\omega_3 - \omega_2}{\omega_3 - \omega_1} =
  \dfrac{z - z_1}{z - z_2}\cdot\dfrac{z_3 - z_2}{z_3 - z_1}
\end{equation}
В \eqref{eq:lecture10-08} точки могут быть как конечными, так и бесконечными, при этом все разности
с бесконечными точками заменяются на 1.
