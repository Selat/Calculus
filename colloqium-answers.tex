\documentclass[a4paper]{article}
\usepackage[utf8]{inputenc}
\usepackage[russian]{babel}
\usepackage{mathtools}
\usepackage{amssymb}
\usepackage{amsthm}
\usepackage{tikz}
\usepackage{multicol}
\usepackage{xparse}
\usepackage{enumitem}
\usepackage{centernot}
\usepackage{comment}

\usepackage[left=1cm,right=1cm,top=1cm,bottom=1cm]{geometry}

\usepackage{setspace}
%\doublespace
\usepackage{epstopdf}
\usepackage{graphicx}
\usepackage{titlesec}

\newtheorem*{theorem}{Теорема}
\newtheorem*{lemma}{Лемма}

\newcommand{\norma}{}
\usepackage{mainstyle}
\renewcommand{\theenumii}{\asbuk{enumii}}

% For cyrillic symbols in "enumerate" environment.
\renewcommand{\theenumii}{\asbuk{enumii}}
\AddEnumerateCounter{\asbuk}{\@asbuk}{ы}


%====================================================================================
%   ENVIORONMENTS

\newenvironment{definition}
{\begin{statement}{Определение}}
    {\end{statement}}

\newenvironment{characteristics}
{\begin{statementItemed}{Свойства}}
    {\end{statementItemed}}

\newenvironment{note}
{\begin{statementDotted}{Замечание}

        }
    {\end{statementDotted}}

\newenvironment{notes}
{\begin{statementItemed}{Замечания}}
    {\end{statementItemed}}

\newenvironment{proofUndotted}
{{\raggedright \textit{Доказательство}}}
{\begin{flushright}
       \boxed{}
 \end{flushright}
}

\newenvironment{noteo}{}{}
\RenewDocumentEnvironment{noteo}{o}
{{\raggedright \textbf{Замечание}\IfValueTF{#1}{ (\textit{#1})}{}.}$  $

}
{}

\newenvironment{consequence}{}{}
\RenewDocumentEnvironment{consequence}{o}
{{\raggedright \textbf{Следствие}\IfValueTF{#1}{ (\textit{#1})}{}.}$  $

    }
{}

\NewDocumentEnvironment{consequences}{o}
{\raggedright \textbf{Следствия}\IfValueTF{#1}{(\textit{#1})}{}:\begin{enumerate}}
    {\end{enumerate}}


\RenewDocumentEnvironment{theorem}{o}
{{\raggedright
  \textbf{Теорема}\IfValueTF{#1}{ (\textit{#1})}{}}.$  $

}

\newenvironment{theoremNamed}{}{}
\RenewDocumentEnvironment{theoremNamed}{m o}
{{\raggedright
  \textbf{Теорема #1}\IfValueTF{#2}{ (\textit{#2})}{}}.$  $

}


\RenewDocumentEnvironment{lemma}{o}
{{\raggedright
        \textbf{Лемма}\IfValueTF{#1}{ (\textit{#1})}{}}.$  $

}

\newenvironment{lemmaNamed}{}{}
\RenewDocumentEnvironment{lemmaNamed}{m o}
{{\raggedright
        \textbf{Лемма #1}\IfValueTF{#2}{ (\textit{#2})}{}}.$  $

}


\newenvironment{exercise}
{\begin{statementDotted}{Упражнение}}
    {\end{statementDotted}}

\newenvironment{example}
{\begin{statementDotted}{Пример}}
    {\end{statementDotted}}

\newenvironment{examples}
{\begin{statementItemed}{Примеры}}
    {\end{statementItemed}}

% =================================
% math functions
\newcommand{\abs}[1]{
    \left\lvert #1 \right\rvert
}

\newcommand{\maxf}[1]{
    \max \left\{ #1 \right\}
}

\newcommand{\suchthat}{
    \;\ifnum\currentgrouptype=16 \middle\fi|\;
}

\newcommand{\arc}[1]{
    \buildrel\,\,\frown\over{#1}
}

\DeclareMathOperator{\const}{\text{const}}

\DeclareMathOperator{\fix}{\text{fix }}
\DeclareMathOperator{\diam}{diam\,}
\DeclareMathOperator{\mes}{mes}
\DeclareMathOperator{\divergence}{div}

\DeclareMathOperator{\arcsh}{arcsh}
\DeclareMathOperator{\arth}{{arth}}
\DeclareMathOperator{\arcth}{{arcth}}
\DeclareMathOperator{\sgn}{sgn}

\newcommand{\parenthesis}[1]{%
    \left( #1 \right)
}

\newcommand{\plot}[1]{$ \text{Г}_{#1} $}

\newcommand{\limlim}[2]
{ \lim\limits_{ \substack{ #1 \\ #2 } } }

\newcommand{\limitslimits}[2]
{ \limits_{ \substack{ #1 \\ #2 } } }

\newcommand{\limlimlim}[3]
{ \lim\limits_{ \substack{ #1 \\ #2 \\ #3 } } }

\newcommand{\nullFrac}{\dfrac{ }{}}

\renewcommand{\emptyset}{\varnothing}

\newcommand{\diint}{\displaystyle\iint}

\newcommand{\liml}{\lim\limits}
\newcommand{\intl}{\int\limits}
\newcommand{\iintl}{\iint\limits}
\newcommand{\diintl}{\diint\limits}
\newcommand{\dintl}{\dint\limits}

\newcommand{\suml}{\sum\limits}
\newcommand{\sumnzi}{\sum\limits_{n=0}^{\infty}}
\newcommand{\limninf}{\lim\limits_{n\to\infty}}
\newcommand{\dintlzi}{\dintl_0^{+\infty}}


%========================================================
% text style

\newcommand{\important}[1]{\textit{#1}}

\newcommand{\dint}{\displaystyle\int}
\newcommand{\dsum}{\displaystyle\sum}

\newcommand{\oiint}[2]{
    \begin{tikzpicture}[baseline=(C.base)]
        \node(C) {$ \displaystyle \iintl_{#1}^{#2} $};
        \draw (0,0.15) circle (0.25);
        %\node[draw,circle,inner sep=1pt](C) ++ (0, 0.1) {$ \;\;\;\; $};
    \end{tikzpicture}
}

\newcommand{\circled}[1]{
    \begin{tikzpicture}[baseline=(C.base)]
    \node[draw,circle,inner sep=1pt](C) {#1};
    \end{tikzpicture}
}

\newcommand{\eqlhopital}{%
    \overset{\circled{Л}}{=}}

\newcommand{\neqlhopital}{%
    \overset{\circled{Л}}{\neq}}

\newcommand{\sqcase}[1]{%
    \left[\begin{matrix}#1\end{matrix}\right]
}

\newcommand{\dvert}{\left.\nullFrac\right\vert}

\renewcommand{\r}[1]{$\overset{\text{ }_\bullet\text{ }}{\text{#1}}$}

\renewcommand{\norma}[1]{\left\lvert\left\lvert#1\right\rvert\right\rvert}

\newcommand{\RN}{\mathbb{R}^n}
\newcommand{\R}[1]{\mathbb{R}^{#1}}

% =================================
% sets utilities

\newcommand{\defineset}[2]{
    \left\{ #1 \, \middle\vert \, #2 \right\}
}

\newcommand{\set}[1]{
    \left\{ #1 \right\}
}

\newcommand{\colquestion}[1]{\section{#1}}
\newenvironment{col-answer-preambule}
               {\ignorespaces}
               {\ignorespacesafterend}

\begin{document}
\begin{center}
  \LARGE\underline{\textbf{Ответы к коллоквиуму по курсу}}\\
  \LARGE\underline{\textbf{ ``Математический анализ''}}\\
  \Large\textbf{(1-ый семестр 2015/2016 учебного года, специальность ``Информатикa'')}
\end{center}
\begin{col-answer-preambule}
	Обозначение поточечной сходимости:
	\begin{equation}
	\label{eq:lecture01-05}
	f_n(x) \overset{X}{\rightarrow}f(x)
	\end{equation}
	Определение \eqref{eq:lecture01-05} на $(\varepsilon-\delta)$-языке:
	\begin{equation}
	\label{eq:lecture01-06}
	\text{для } \forall \; \varepsilon > 0 \text{ и }
	\text{для } \forall \; fix \; x \in X \; \exists \; \nu = \nu(x, \varepsilon) \in \mathbb{R}\; | \text{ для } \forall \; n \geqslant \nu \Rightarrow \abs{f_n(x) - f(x)} \leqslant \varepsilon.
	\end{equation}
	Обозначение равномерной сходимости:
	\begin{equation}
	\label{eq:lecture01-07}
	f_n(x) \overset{X}{\rightrightarrows}f(x)
	\end{equation}
	Определение \eqref{eq:lecture01-07} на $(\varepsilon-\delta)$-языке:
	\begin{equation}
	\label{eq:lecture01-06fixed}
	\text{для } \forall \; \varepsilon > 0 \; \exists \; \nu = \nu(\varepsilon) \in \mathbb{R}\; | \text{ для } \forall \; fix \; x \in X  \text{ и } \text{для } \forall \; n \geqslant \nu\Rightarrow \abs{f_n(x) - f(x)} \leqslant \varepsilon.
	\end{equation}
\end{col-answer-preambule}

\colquestion{Супремальный критерий равномерной сходимости функциональных последовательностей (ФП) и замечания к нему}
\begin{theorem}[Супремальный критерий равномерной сходимости ФП]
	\begin{equation}
	\label{eq:lecture01-09}
	f_n(x) \overset{X}{\rightrightarrows}f(x) \Leftrightarrow
	r_n = \sup_{x \in X}\abs{f_n(x) - f(x)} \xrightarrow[n \to \infty]{}0.
	\end{equation}
\end{theorem}
\begin{proof}
	\circled{$\Rightarrow$} Если выполнена \eqref{eq:lecture01-07}, то, учитывая, что в \eqref{eq:lecture01-06fixed} используется $\forall \; fix \; x		 \in X$ и $\forall \; n \geqslant \nu(\varepsilon)$, получаем
	\begin{equation*}
	\begin{split}
	&r_n = \sup_{x \in X}\abs{f_n(x) - f(x)} \leqslant \varepsilon, \text{ т.е. }\\
	&\text{для }\forall \; \varepsilon > 0 \; \exists \; \nu = \nu(\varepsilon) \in \mathbb{R} \; \,\vert\, \text{ для }\forall \;
	n \geqslant \nu \Rightarrow 0 \leqslant r_n \leqslant \varepsilon, \text{ т.е. }
	r_n \xrightarrow[n \to \infty]{}0.
	\end{split}
	\end{equation*}\\
	\circled{$\Leftarrow$}
	Пусть выполнена \eqref{eq:lecture01-09}, тогда
	\begin{equation*}
	\begin{split}
	& \text{для }\forall \; \varepsilon > 0 \; \exists \; \nu = \nu(\varepsilon) \in \mathbb{R} \; \,\vert\, \text{ для }\forall \; n
	\geqslant \nu \text{ и для } \forall \; x \in X \Rightarrow \\
	& \Rightarrow \abs{f_n(x) - f(x)} \leqslant \sup\limits_{x \in X}\abs{f_n(x) - f(x)} = r_n \leqslant
	\varepsilon.
	\end{split}
	\end{equation*}
	Таким образом, имеем \eqref{eq:lecture01-06}, где $\nu$ зависит от $\forall \; \varepsilon > 0$ и
	не зависит от конкретного элемента множества $X$.
\end{proof}

\begin{notes}
	\item Если известно, что для $\forall \; n \in \mathbb{N}$ и для $\forall \; x \in X \Rightarrow
	\abs{f_n(x) - f(x)} \leqslant a_n$, где $\left(a_n\right)$ - б.м.п, то тогда имеем \eqref{eq:lecture01-07}.
	Сформулированное утверждение даёт \important{мажоритарный признак} (достаточное условие)
	равномерной сходимости ФП.
	\item Если
	\begin{equation*}
	\exists \; x_n \in X \; \,\vert\,\; g_n(x) = \abs{f_n(x) - f(x)} \Rightarrow g_n(x) \centernot{
		\xrightarrow[n \to \infty]{}} 0,
	\end{equation*}
	то тогда равномерной сходимости нет, т.е. $f_n(x) \centernot{\overset{X}{\rightrightarrows}} f(x)$. Это
	даёт достаточное условие (признак) неравномерной сходимости ФП.
\end{notes}
\newpage
\begin{col-answer-preambule}
	Обозначение равномерной сходимости ФР:
	\begin{equation}
	\label{eq:lecture01-20}
	\sum u_n(x) \overset{X}{\rightrightarrows}.
	\end{equation}
	Критерий Коши равномерной сходимости ФР:
	\begin{equation}
	\label{eq:lecture01-21}
	\eqref{eq:lecture01-20} \Leftrightarrow \text{для } \forall \; \varepsilon > 0 \; \exists \; \nu = \nu (\varepsilon) \in \mathbb{R} \; |
	\; \text{для } \forall \; x \in X \text{ и для } \forall \; n \geqslant \nu \; \text{и для }\forall \; m \in \mathbb{N} \Rightarrow 
	\abs{S_{n+m}(x) - S_n(x)} = \abs{ \sum\limits_{k = n + 1}^{n + m} u_k(x) } \leqslant \varepsilon.
	\end{equation}
	Критерий Коши сходимости ЧР:
	\begin{equation}
	\label{eq:lecture01-temp}
	\sum a_n \text{ сходится} \Leftrightarrow \text{для } \forall \;  \varepsilon>0 \ \exists \; \nu \in \mathbb{R} : \text{для }\forall \; n \geqslant \nu \; \text{ и} \;  \text{для }\forall\; m \in \mathbb{N} \Rightarrow \abs{S_{n+m} - S_n} = \abs{\sum\limits_{k = n + 1}^{n + m} a_k} \leqslant \varepsilon
	\end{equation}
	ЧП $\left(a_n\right)$ является \important{сходящейся числовой мажорантой} для ФР $\sum u_n (x)$, если:
	\begin{equation}
	\label{eq:lecture01-mazh-01}
		\text{1.  ЧР} \sum a_n \text{ сходится},
	\end{equation}
	\begin{equation}
	\label{eq:lecture01-mazh-02}
	\text{2. для }\forall \; n \in \mathbb{N}  \text{ и для } \forall \; x \in X \Rightarrow |u_n(x)| \leqslant a_n.
	\end{equation}
\end{col-answer-preambule}

\colquestion{Мажорантный признак Вейерштрасса равномерной сходимости функционального ряда (ФР) и замечания к нему}
\begin{theorem}[мажорантный признак Вейерштрасса равномерной сходимости ФР] Если ФР имеет на $X$ сходяющуюся числовую мажоранту, то он равномерно сходится на $X$.
\end{theorem}
\begin{plan}
	\item По def \important{сходящейся числовой мажоранты}.
	\subitem Расписать \eqref{eq:lecture01-mazh-01} по \eqref{eq:lecture01-temp}: $\abs{\sum\limits_{k = n+1}^{n+m} a_k} \leqslant \varepsilon$.
	\subitem Подставить \eqref{eq:lecture01-mazh-02} в \eqref{eq:lecture01-21}.
\end{plan}
\begin{proof}
	Доказательство с использованием критерия Коши сходимости ЧП \eqref{eq:lecture01-temp} и критерия Коши равномерной сходимости ФР \eqref{eq:lecture01-21}:

	Т.к. $\sum a_n$ сходится, то
	\begin{equation}
	\label{eq:1_23}
	\text{для } \forall \; \varepsilon > 0 \; \exists \; \nu(\varepsilon) \in \mathbb{R} \; | \; \text{для } \forall \;	n \geqslant \nu \text{ и для } \forall \; m \in \mathbb{N} \Rightarrow \abs{\sum_{k = n+1}^{n+m} a_k} \leqslant \varepsilon.
	\end{equation}

	Если $\text{для } \forall \; n \in \mathbb{N} \text{ и для } \forall \; x \in X \Rightarrow \abs{u_n(x)} \leqslant a_n$, то для частичных сумм ФР $\sum u_n(x)$ имеем: \newline $\abs{S_{m+n} (x) - S_n (x)} = \abs{\sum\limits_{k = n+1}^{n+m} u_k(x)} \leqslant \sum\limits_{k = n + 1}^{n+m} \abs{u_k (x)} \leqslant \sum\limits_{k = n+1}^{n+m} a_k = \abs{\sum\limits_{k = n+1}^{n+m} a_k} \leqslant \varepsilon$, это для $\forall \; n \geqslant \nu = \nu(\varepsilon) \text{ и для } \forall \; m \in \mathbb{N}$, что в силу \eqref{eq:lecture01-21} даёт \eqref{eq:lecture01-20}.
\end{proof}

\begin{notes}
	\item Признак Вейерштрасса является лишь достаточным условием равномерной сходимости ФР. На практике сходящуюся числовую мажоранту $\left( a_n \right)$ либо находят с помощью соответствующих оценок $\abs{u_n(x)}$ сверху, либо берут $a_n = \underset{x \in X}{sup} \abs{u_n(x)}$. В последнем случае получаем наиболее точную мажоранту, но в случае расходимости $\sum a_n$ даже для этой самой точной мажоранты ничего о равномерной сходимости ФР сказать нельзя, т.е. требуются дополнительные исследования.
	\item Обобщая признак Вейерштрасса, где используется сходимость числовой мажоранты - признак равомерной сходимости ФР, используют функцию мажоранты, а именно получаем:
	\begin{equation*}
	 \text{если } \exists \; v_n(x) \geqslant 0 \; : \; \abs{u_n(x)} \leqslant v_n(x) \; \text{для } \forall \; n \in \mathbb{N} \text{ и для } \forall \; x \in X \text{ и } \sum v_n(x)	 \overset{X}{\rightrightarrows},
	 \end{equation*}
	 то тогда для ФР $\sum u_n(x)$ имеем \eqref{eq:lecture01-20}.
\end{notes}

\newpage
\begin{col-answer-preambule}
	\begin{plan}
		\item Формулировка: из названия (как и Дирихле для рядов).
		\item Доказательство:
		\subitem оценка Абеля, взятая c 2-кой для надёжности.
		\subitem оценить $\abs{b_{n+1}}$ и $\abs{b_{n+m}}$ по $\widetilde{\varepsilon} = \dfrac{\varepsilon}{6 \cdot c}$
		\subitem def равномерной сходимости для $\sum a_n(x) b_n(x)$.
		\item Замечания: как и для рядов ($\sum (-1)^n b_n(x)	\overset{X}{\rightrightarrows}$, $\text{Лейбница} \approx \text{единица}$).
	\end{plan}
\end{col-answer-preambule}
\colquestion{Признак Дирихле равномерной сходимости ФР и следствие из него (признак Лейбница равномерной сходимости ФР)}
\begin{theorem}[Признак Дирихле равномерной сходимости ФР] Пусть для ФП $a_n(x)$ частичные суммы $\sum a_n(x)$ ограничены в совокупности (равномерно на $X$), т.е.
	\begin{equation}
	\label{eq:1_24}
	\text{для }\forall \; x \in X \text{ и для } \forall \; n \in \mathbb{N} \Rightarrow \abs{a_1 (x) + a_2(x) + \ldots + a_n(x)} \leqslant c,
	\end{equation}
	где $c = const > 0$, \important{не зависит ни от $n$, ни от $x$}. Если $\forall \; fix \; x \in X \Rightarrow \left( b_n(x) \right)$ - функциональная последовательность является монотонной, то в случае
	\begin{equation}
	\label{eq:1_25}
	b_n(x) \overset{X}{\rightrightarrows} 0,
	\end{equation}
	имеем $\sum a_n(x) b_n(x) \overset{X}{\rightrightarrows}$.
\end{theorem}
\begin{proof}
	Монотонная последовательность $\left( b_n(x) \right) \text{ для } \forall \; fix \; x \in X$ позволяет так же, как и в ЧР, использовать на основе \eqref{eq:1_24} оценку Абеля:
	\begin{equation}
	\label{eq:1_26}
	\abs{\sum_{k = n+1}^{n+m} a_k(x) b_k(x)} \leqslant 2 \text{\important{c}} \left(\abs{b_{n+1}(x)} + 2 \abs{b_{n+m} (x)}\right).
	\end{equation}

	Если выполняется \eqref{eq:1_25}, то тогда имеем:
	\begin{equation*}
	\text{ для } \forall \; \varepsilon > 0 \text{ по числу } \tilde{\varepsilon} = \dfrac{\varepsilon}{6 c} > 0 \; \exists \; \nu (\varepsilon) \in \mathbb{R} \; | \; \text{для } \forall \; n \geqslant \nu(\varepsilon) \text{ и для } \forall \; m \in \mathbb{N} \text{ и для } \forall \; x \in X \Rightarrow \abs{b_{n+1} (x) } \leqslant \tilde{\varepsilon} \text{ и } \abs{b_{n+m} (x)} \leqslant \tilde{\varepsilon}, 
	\end{equation*}
	поэтому для частичных сумм $S_n(x) = \sum\limits_{k=1}^{n} a_k(x) b_k(x)$ в силу \eqref{eq:1_26} $\text{для } \forall \; n \geqslant \nu(\varepsilon) \text{ и для } \forall \; m \in \mathbb{N} \text{ и для } \forall \; x \in X$ имеем: $\abs{S_{n+m} (x)  - S_n(x)} = \abs{\sum\limits_{k = n+1}^{n+m} a_k(m) b_k(x)} \leqslant 2 \cdot c \cdot(\tilde{\varepsilon} 	+ 2 \tilde{\varepsilon} ) = 6\cdot c \cdot  \tilde{\varepsilon} = \varepsilon$. Отсюда по критерию Коши равномерной сходимости ФР следует, что $\sum\limits a_n(x) b_n(x) \overset{X}{\rightrightarrows}$.
\end{proof}
\begin{consequence}[Признак Лейбница равномерной сходимости ФР]
	Если $\forall \; fix \; x \in X$ последовательность $\left(b_n(x)\right)$ является монотонной, то в случае $b_n(x) \overset{X}{\rightrightarrows}
	0 \Rightarrow 	\sum (-1)^n b_n(x)	\overset{X}{\rightrightarrows}$.
\end{consequence}
\begin{proof}
	Следует из того, что в условии теоремы $a_n = (-1)^n$ не зависит от $x$, причём \newline $\abs{\sum\limits_{k=1}^{n} a_k} \leqslant 1 = const, \text{для } \forall \; n \in \mathbb{N}$.
\end{proof}

\newpage
\begin{col-answer-preambule}
	Для обозначения поточечной сходимости ФР $\sum u_n(x)$ на X будем использовать запись:
	\begin{equation}
	\label{eq:lecture01-14}
	\sum u_n(x) \overset{X}{\rightarrow}.
	\end{equation}
	\begin{plan}
		\item Формулировка: \important{оДин}и - один знак, Ди\important{ни} - непрерывны, Ди\important{ни} - непрерывны.
		\item Доказательство:
		\subitem 3 свойства остатка ряда $R_n(x) = S(x) - S_n(x)$: Fun UFO (\textcolor{magenta}{F}u\textcolor{magenta}{n} UFO — функция непрерывна, Fu\textcolor{magenta}{n} \textcolor{magenta}{U}FO — функциональная последовательность убывает, Fun U\textcolor{magenta}{FO} — функция к 0).
		\subitem дм у пво (\textcolor{magenta}{д}е \textcolor{magenta}{М}орган, \textcolor{magenta}{у}прощение, \textcolor{magenta}{п}ринцип \textcolor{magenta}{в}ыбора, $x_{\textcolor{magenta}{0}}$)
		\subitem противоречие с последним свойством остатка.
		\subitem $R_{m} (x_{nk}) \geqslant R_{nk} (x_{nk}) > \varepsilon_0 \Rightarrow [\text{ переходя к пределу }] \Rightarrow R_m (x_0) = \lim\limits_{n_k \to \infty} R_m(x_{nk}) \geqslant \varepsilon_0$, что противоречит последнему из свойств остатка.
		\item Теорема: то же самое, только вместо сохранения одного знака члены ФП будут монотонны.
			\subitem по доказанному признаку, задав ФР как $u_n(x) = f_n(x) - f_{n-1}(x)$.
	\end{plan}
\end{col-answer-preambule}
\colquestion{Признак Дини равномерной сходимости ФР и следствие из него (теорема Дини для ФП)}
\begin{theorem}[Признак Дини равномерной сходимости ФР]
	Пусть
	\begin{enumerate}
		\item Члены ФР $\sum u_n(x)$ непрерывны и сохраняют один и тот же знак на $X = [a, b], \text{ для } \forall \; n \in \mathbb{N}$.
		\item $\sum u_n(x) \overset{X}{\to} S(x)$.
	\end{enumerate}
	Тогда, если $S(x) = \sum\limits_{n=1}^{\infty} u_n (x)$ - непрерывная функция на $[a, b]$, т.е. $S(x) \in C([a, b])$, то $\sum u_n(x) \overset{X}{\rightrightarrows}$.
\end{theorem}
\begin{proof}
	Рассмотрим на $X = [a,b]$ остатки ряда $R_n(x) = u_{n+1}(x) + \ldots = S(x) - S_n(x)$.	Нетрудно видеть, что выполняются следующие свойства:
	\begin{enumerate}
		\item для $\forall \; fix \; n \in \mathbb{N} \Rightarrow R_n (x)$ - непрерывная функция на $[a,b]$ как разность двух непрерывных функций.
		\item для $\forall \; fix \; x \in X \Rightarrow$ $\text{ФП}$ $(R_n(x))$ убывает в случае, когда $\forall \; u_n (x) > 0$, т.к. \newline $R_n(x) = u_n(x) + R_{n+1}(x) \geqslant R_{n+1}(x), \text{ для } \forall \; n \in \mathbb{N}$.
		\item Т.к. имеет место \eqref{eq:lecture01-14}, то для $\forall \; fix \; x \in X \Rightarrow R_n(x) \overset{X}{\to} 0$.
	\end{enumerate}
	Докажем от противного. Предположим, что рассматриваемая положительная поточечная сходимость на $X$ ФР не является равномерной сходимостью на $X$.

	Тогда по правилу де Моргана имеем: $\exists \; \varepsilon_0 > 0 \; | \; \text{для } \forall \; \nu \in \mathbb{R} \; \exists \; n (\nu) \geqslant 0 \text{ и } \exists \; x (\nu) \in X \; | \; R_{n \nu} (x_\nu) > \varepsilon_0$. Для простоты будем считать, что $\exists \; x_n \in X \; | \; R_n (x_n) > \varepsilon_0$. По принципу выбора из ограниченной последовательности $x_n$ можно выбрать сходящуюся подпоследовательность, т.е. $x_{n_k} \underset{n_k \to \infty}{\longrightarrow} x_0$, при этом в силу использования $X = [a,b]$ - компакт, получаем, что $x_0 \in X$. Если зафиксируем $m \in \mathbb{N}$, то для $\forall \; n_k \geqslant m \Rightarrow R_{n_k} (x_{n_k}) > \varepsilon_0$, по свойствам остаткам будем иметь, что $R_{m} (x_{n_k}) \geqslant R_{n_k} (x_{n_k}) > \varepsilon_0$. В неравенстве $R_m (x_{n_k}) > \varepsilon_0$, переходя к пределу при $n_k \to \infty \; \text{для } \forall \; m \in \mathbb{N}$, получаем в силу непрерывности $R_n(x): R_m (x_0) = \lim\limits_{n_k \to \infty} R_m(x_{n_k}) \geqslant \varepsilon_0$, что противоречит последнему из свойств остатка, а именно $R_m(x_0) \overset{X}{\longrightarrow} 0$ при $m \to \infty$, поэтому из нашего предположения следует, что выполняется $R_m(x_0) \not\to 0$, противоречие, т.е. выполняется $\sum u_n(x) \overset{X}{\rightrightarrows}$.
\end{proof}
\begin{consequence}[Теорема Дини для ФП]
	Если для ФП $f_n(x), n \in \mathbb{N}$ на $X = [a,b]$ выполняются свойства:
	\begin{enumerate}
		\item для $\forall \; f_n(x) \in C([a,b])$ и для	 $\forall \; fix \; x \in X \Rightarrow f_n(x)$ монотонна.
		\item $f_n(x) 	\overset{X}{\longrightarrow}f(x)$. Тогда, если $f(x) \in C([a,b])$, то $f_n(x) \overset{X}{\rightrightarrows}$.
	\end{enumerate}
\end{consequence}
\begin{proof}
	следует из того, что члены рассматриваемой ФП $f_n(x)$ можно рассматривать как частичные суммы соответствующего ФР с общим членом
	\begin{equation}
	\label{eq:1_27}
	\begin{cases}
	u_n(x) = f_n(x) - f_{n-1}(x),\\
	f_0(x) = 0.
	\end{cases}
	\end{equation}
	Действительно, $S_n(x) = f_n(x) - f_0(x) = f_n(x), \text{ для } \forall \; n \in \mathbb{N}$.

	А далее к соответствующему ФР применима теорема Дини равномерной сходимости ФР.
\end{proof}

\newpage
\begin{col-answer-preambule}
Пусть $x_0$ - предельная точка множества сходимости $X \subset \mathbb{R}$ для ФР $\sum u_n(x)$.
Будем говорить, что в $\sum u_n(x)$ \important{возможен почленный предельный переход} $x \to x_0$, если
\begin{equation}
\label{eq:lecture01-28}
\exists \; lim \sum_{n = 1}^{\infty} u_n(x) = \sum_{n = 1}^{\infty} lim u_n(x),
\end{equation}
причём получившийся в левой части \eqref{eq:lecture01-28} ЧР является сходящимся.

В частности, если $x_0 \in X$ и $\forall \; u_n(x)$ непрерывен в некоторой окрестности точки $x_0$, и значит, для $\forall \; n \in \mathbb{N} \; \exists \lim\limits_{x \to x_0} u_n(x) = u_n(x_0)$, то в случае выполнения \eqref{eq:lecture01-28} для суммы $S(x)$ ФР $\sum u_n(x)$ при $x \to x_0$ имеем:
\begin{equation}
\label{eq:lecture01-29}
\exists \; \lim\limits_{x \to x_0} S(x) = \lim\limits_{x \to x_0} \sum_{n=1}^{\infty} u_n(x) = \sum_{n=1}^{\infty} \lim\limits_{x \to x_0} u_n(x) = \sum_{n=1}^{\infty} u_n(x_0) = S(x),
\end{equation}
что соответствует непрерывности $S(x)$ в точке $x_0 \in X$.
\end{col-answer-preambule}

\colquestion{Теорема о непрерывности суммы равномерно сходящегося ФР и замечания к ней}
\begin{theorem}[о непрерывности суммы равномерно сходящегося ФР]
	Если все члены $u_n(x), n \in \mathbb{N}$, ФР $\sum u_n(x)$ непрерывны на $X = [a,b]$, то в случае равномерной сходимости этого ряда на $[a,b]$ его сумма $S(x)$ будет непрерывной функцией на $[a,b]$.
\end{theorem}
\begin{proof}
	Требуется обосновать \eqref{eq:lecture01-29} для $\forall \; x_0 \in [a,b]$, причём в случае концевых значений $x_0 = a, \; x_0 = b$ будем использовать соответствующие односторонние пределы, т.е. рассматривать одностороннюю непрерывность.

	Для $fix \; x_0 \in [a,b]$ придадим произвольные приращения $\Delta x \in \mathbb{R} \; | \; (x_0 + \Delta x) \in [a,b]$ и рассмотрим соответствующие приращения суммы ФР $\sum u_n(x)$:
    \begin{equation*}
        \Delta S(x_0) = S(x_0 + \Delta x) - S(x_0).
    \end{equation*}

    Из равномерной сходимости ФР $\sum u_n(x)$ на
    $X = [a,b] \Rightarrow \text{для } \forall \; \varepsilon > 0 $
    ${ \exists \; \nu = \nu(\varepsilon) \in \mathbb{R} \; | \; \text{для } \forall \; n \geqslant \nu }$,
    $ \text{ и для } \forall \; x \in [a,b]$ для частичных сумм $S_n(x) = u_1(x) + u_2(x) + \ldots + u_n(x)$ ряда $\sum u_n(x)$ имеем: $\abs{S_n(x) - S(x)} \leqslant \varepsilon$.

	Отсюда, в частности, для $x = x_0 \in X $ и $ x = x_0 + \Delta x \in X \Rightarrow$
	\begin{equation}
	\label{eq:1_30}
	\begin{cases}
	\abs{S_n(x_0) - S(x_0)} \leqslant \varepsilon, \\
	\abs{S_n(x_0 + \Delta x) - S(x_0 + \Delta x)} \leqslant \varepsilon.
	\end{cases}
	\end{equation}

	Далее из непрерывности $\forall \; u_n(x)$ в $x_0 \in [a,b]$ следует непрерывность частичных сумм в $x_0$ (как конечных сумм непрерывных функций).

	В силу этого, для $ \forall \; \varepsilon \; \exists \; \delta > 0 \; : \; \text{для } \forall \; \abs{\Delta x} \leq \delta \Rightarrow$
	\begin{equation}
    	\label{eq:1_31}
    	\Rightarrow \abs{S_n(x_0 + \Delta x) - S_n(x_0)} \leqslant \varepsilon.
	\end{equation}

	Таким образом,  в силу \eqref{eq:1_30}, \eqref{eq:1_31} имеем: для $\forall \varepsilon > 0$, выбирая $n \geqslant \nu$ и рассматривая $\forall \abs{\Delta x}	 \leqslant \delta$, имеем:

	$\abs{\Delta S(x_0)} = \abs{S_n(x_0) - S(x_0) + S_n(x_0 + \Delta x) - S_n(x_0) + S (x_0 + \Delta x) - S_n(x_0 + \Delta x)} \leqslant \\  \leqslant \abs{S_n(x_0) - S(x_0)} + \abs{S_n(x_0 + \Delta x) - S_n(x_0)} + \abs{S(x_0 + \Delta x	) - S_n(x_0 + \Delta x)} \leqslant \varepsilon + \varepsilon + \varepsilon = 3 \cdot \varepsilon$.

    Поэтому получаем:
	для	$\forall \; \varepsilon \; \exists \; \delta > 0 : \; \text{для } \forall \; \abs{\Delta x} \leqslant \delta \Rightarrow \abs{\Delta S(x_0)} \leqslant M \cdot \varepsilon, M = const = 3 > 0$.

	Отсюда по М-лемме для Ф1П следует, что $\Delta S(x_0) \underset{\Delta x \to 0}{\to} 0$, что на языке приращений равносильно \eqref{eq:lecture01-29}. При этом, т.к. из равномерной сходимости следует поточечная сходимость ЧР в правой части \eqref{eq:lecture01-29} будет сходящимся.
\end{proof}

\begin{notes}
	\item Доказанную теорему часто называют теоремой Стокса-Зейделя или теоремой Стокса-Зайделя.
	\item В условии доказанной теоремы равномерную сходимость можно заменить для произвольного множества $ X \subset \mathbb{R}$ на локальную равномерную сходимость. Будем говорить, что ФР $\sum u_n(x)$ сходится локально равномерно на $X \subset \mathbb{R}$, если для $\forall \; [a,b] \subset X \Rightarrow \sum u_n(x) (x) \overset{[a,b]}{\rightrightarrows}$. У $\sum u_n(x)$ может быть локальная равномерная сходимость на $X$, но может не быть полной (???) равномерной сходимости на $X$. В случае локальной равномерной сходимости $\sum u_n(x)$ на $X$ берём $\forall x_0 \in X$ и заключаём её в некоторый отрезок $x_0 \in [a,b] \subset X$. Т.к. есть равномерная сходимость для $\sum u_n(x)$ на этом отрезке, то по доказанной теореме сумма $S(x)$ в случае непрерывности $\forall \; u_n(x)$ на $X$ будет непрерывна на $[a,b] \subset X$ и, в частности, непрерывна в $x_0 \in X$, а т.к. это можно сделать для $\forall \; x_0 \in X$, то тем самым получаем непрерывность $S(x)$ на $X \subset \mathbb{R}$ даже в случае, когда нет равномерной сходимости ФР на $X$.
\end{notes}

\newpage
\colquestion{Теорема о почленном интегрировании равномерно сходящегося ФР}
\begin{plan}
\item Очевидно, что $S(x)$ - непрерывна, поэтому интегрируема
\item Рассмотрим частичные суммы $T_n = \suml_{k=1}^{n}\int u_k(x)dx$.
\item Рассмотрим разницу $\abs{T_n - \intl_a^bS(x)}$ и т.к. $\abs{S(x) - S_n(x)} \leqslant \varepsilon$ получим $\intl_a^b (S(x) - S_n(x)) \leqslant M \varepsilon$
\item Доказываем по М-лемме о сходимости ЧП.
\end{plan}
\begin{theorem}[о почленном интегрировании равномерно сходящихся ФР]
	Если $\forall \; u_n(x) \in C([a,b]), $ \\ для $n \in \mathbb{N}$, то в случае, когда $\sum u_n(x) \overset{[a,b]}{\rightrightarrows}$, возможно почленное интегрирование этого ряда на $[a,b]$, т.е.
	\begin{equation}
	\label{eq:lecture01-32}
	\exists \dint\limits_a^b S(x)dx = \dint\limits_a^b \left(\sum_{n=1}^{\infty}u_n(x)\right)dx = \sum_{n=1}^{\infty} \dint\limits_a^b u_n(x)dx.
	\end{equation}
\end{theorem}
\begin{proof}
	На основании теоремы о непрерывности суммы равномерно сходящихся ФР получим, что сумма ряда $S(x) = \sum\limits_{n=1}^{\infty}u_n(x)$ будет непрерывна на $[a,b]$, а значит, интегрируема на $[a,b]$.

	Используя частичные суммы для $\sum u_n(x)$, рассмотрим частичные суммы $T_n = \dint\limits_a^b S_n(x)dx  =$ \\$= 	\dint\limits_a^b \sum_{k=1}^{n} u_k(x)dx = \sum\limits_{k=1}^{n} \dint\limits_a^b u_k(x)dx$ для ЧР правой части \eqref{eq:lecture01-32}.

	Требуется доказать, что $\lim\limits_{n \to \infty} T_n = \dint\limits_a^b S(x)dx$.

    Из равномерной сходимости $\sum u_n(x)$ на $[a,b]$ получим, что для $\forall \; \varepsilon > 0 \; \exists \; \nu = \nu(\varepsilon) \; | \; \text{для } \forall \; n \geqslant \nu $ и для $ \forall \; x \in [a,b] \Rightarrow$
	\begin{equation}
	\label{eq:1_33}
	 	\abs{S(x) - S_n(x)} = \abs{\sum_{k = n+1}^{\infty} u_k(x)} \leqslant \varepsilon
	\end{equation}

	Отсюда получаем, что $\abs{\dint\limits_a^b S(x)dx - T_n} = \abs{\dint\limits_a^b S(x)dx - \dint\limits_a^b S_n(x)dx} =  \abs{\dint\limits_a^b (S(x) - S_n(x))dx} \leqslant $ \\ $\leqslant \dint\limits_a^b \abs{S(x) - S_n(x)}dx \leqslant \dint\limits_a^b \varepsilon dx = M \varepsilon$, где $M = b - a = const \geqslant 0$.

	Таким образом, для $\forall \; \varepsilon > 0 \; \exists \; \nu = \nu(\varepsilon) \; | \; \text{ для } \forall \; n \geqslant \nu \Rightarrow \abs{\dint\limits_a^b S(x) dx - T_n} \leqslant M \varepsilon$, поэтому по М-лемме сходимости ЧП следует, что
	\begin{equation*}
    	\exists \lim\limits_{n \to \infty} T_n = \dint\limits_a^b S(x)dx = \dint\limits_a^b \left(\sum\limits_{k=1}^{\infty} u_k(x)\right) dx,
	\end{equation*}
	что равносильно \eqref{eq:lecture01-32}.
\end{proof}

\newpage
\colquestion{Теорема о почленном дифференцировании ФР}

\begin{theorem}[о почленном дифференцировании ФР]
	Пусть ФР $\sum u_n(x)$ на $X = [a,b]$ удовлетворяет условиям:
	\begin{enumerate}
		\item $\sum u_n(x) \overset{X}{\rightarrow}$,
		\item $\exists \; u_n^{'}(x)$, непрерывная для $\forall \; n \in \mathbb{N}, x \in X$.
	\end{enumerate}
	Тогда, если
    \begin{equation}
        \label{eq:1_34}
        \sum u_n^{'}(x) \overset{X}{\rightrightarrows}
    \end{equation}
    то рассматриваемый ФР $\sum u_n(x)$ можно почленно дифференцировать на $[a,b]$, т.е.
	\begin{equation}
	\label{eq:1_35}
	\exists \left( \sum_{n=1}^{\infty} u_n (x) \right)^{'} = \sum_{k=1}^{\infty} u_k^{'}(x), \text{для }\forall \; x \in X.
	\end{equation}
\end{theorem}
\begin{proof}
	В силу \eqref{eq:1_34}, по условию 2 рассматриваемой теоремы получаем, что по теореме об интегрировании ФР $\sum u_n^{'}(t)$ можно почленно интегрировать на
    	$\forall \; [a,x] \subset [a,b]$, \nolinebreak т.е.

    \begin{equation*}
        \exists \dint\limits_a^x \left(\sum\limits_{n=1}^{\infty} u_n^{'}(t)\right)dt = \sum_{n=1}^{\infty} \dint\limits_a^x u_n^{'}(t)dt = \sum_{n=1}^{\infty} [u_n]^{t = x}_{t = a} = \sum\limits_{n=1}^{\infty}\left(u_n(x) - u_n(a)\right).
    \end{equation*}

	Отсюда в силу условия 1 (поточечная сходимость для $\sum u_n(x)$) получаем, что
    \begin{equation*}
            \exists \; S(x) = \sum\limits_{n=1}^{\infty} u_n(x) = \sum\limits_{n=1}^{\infty}u_n(a) + \dint\limits_a^x \sum\limits_{n=1}^{\infty}u_n^{'}(t)dt.
    \end{equation*}

	Используя далее \important{теорему Барроу} о дифференцировании интеграла с переменным верхним пределом от непрерывной подынтегральной функции, получаем:\\
	\begin{equation*}
	\exists \; S^{'}(x) = (const)^{'} + \left(\dint\limits_a^x \left( \sum\limits_{n=1}^{\infty} u_n^{'} (t) \right)dt \right)^{'}_x = \sum\limits_{n=1}^{\infty}u_n^{'} (x),
	\end{equation*} что соответствует \eqref{eq:1_35}.
\end{proof}

\newpage
\begin{col-answer-preambule}
Под \important{степенным рядом} будем подразумевать ФР вида
\begin{equation}
\label{2.01}
a_0 + a_1(x-x_0) + a_2(x-x_0)^2 + \ldots +  a_n(x-x_0)^n + \ldots
= \sum_{n=0}^{\infty} a_n(x-x_0)^n,
\end{equation}
где $ \fix x_0 \in \R{} $ - центр для СтР, а $ \forall \;	 a_n \in \R{} $ - соответствующая числовая последовательность (\important{коэффициенты СтР}).
\end{col-answer-preambule}

\colquestion{Теорема Абеля о сходимости степенного ряда (СтР) и замечание к ней.}
\begin{plan}
\item \textcolor{magenta}{А}беля - сходится \textcolor{magenta}{а}бсолютно, признак сравнения ЧР.
\item Сходящаяся ЧП является ограниченной (т.е. ограничен каждый её член)
\item Рассматриваем это условие для $x_1$, получаем верхнюю границу для $a_n$.
\item Затем аналогично рассматриваем условие для $x$, ограничивая сверху $M q^n$.
\end{plan}
\begin{statementDotted}{Теорема Абеля}[о сходимости степенных рядов]

	Если СтР \eqref{2.01} сходится при $ x = x_1 \neq x_0 $, то он будет сходится абсолютно для любого $ x $, где
	\begin{equation}
	\label{2.02}
	\abs{x-x_0} < \abs{x_1 - x_0}.
	\end{equation}

\end{statementDotted}
\begin{proof}
	Из сходимости при $x = x_1$, т.е. ряда $ \sumnzi a_n(x_1-x_0)^n $ следует в силу необходимого условия сходимости ЧР, что $ a_n(x_1-x_0)^n \xrightarrow[n \to \infty]{} 0$ ,
	а т.к. $\forall$ сходящаяся ЧП является ограниченной, то \newline
	$ \exists \; M  = \const > 0 :
	\abs{a_n(x_1-x_0)^n} \leq M, \text{ для } \forall \; n \in \mathbb{N}$, т. е.
	\begin{equation}
	\label{2.03}
	\abs{a_n} \leq \dfrac{M}{\abs{x_1-x_0}^n}.
	\end{equation}

	Для $\forall \; x $, удовлетворяющего \eqref{2.02}, в силу \eqref{2.03} получаем:
	\begin{equation*}
	\abs{a_n (x - x_0)^n} = \abs{a_n} \abs{x - x_0}^n \overset{\eqref{2.03}}{\leq}
	\dfrac{M \abs{x - x_0}^n}{\abs{x_1 - x_0}^n} = Mq^n,
	\text{ где } q = \dfrac{\abs{x - x_0}}{\abs{x_1 - x_0}} \in [0;1[.
	\end{equation*}

	Таким образом, мы получили сходящуюся мажоранту, ибо ряд $ \sumnzi Mq^n = M \sumnzi q^n $ сходится при $ q \in [0;1[ $.

	По признаку сравнения сходимости ЧР имеем, что для $ \forall \; x $, удовлетворяющего \eqref{2.02}, ряд \eqref{2.01} будет сходиться.
\end{proof}
\begin{note}
	Из полученных выше результатов следует, что если рассмотреть множество $ X_0 $ всех $ x $, удовлетворяющих \eqref{2.02}, то имеем, что $ X_0 \subset X $, т.е. $X_0$ - некоторое подмножество множества $X$ сходимости для \eqref{2.01}.
\end{note}

\newpage
\begin{col-answer-preambule}
\end{col-answer-preambule}

\colquestion{Формула Даламбера для вычисления радиуса сходимости СтР.}
\begin{plan}
\item Рассматриваем $x \in \interval]{-R + x_0; x_0 + R}[ (x \neq x_0)$ .
  \item Подставляем в теорему Даламбера для ЧР ($a_{n+1} / a_n$).
    \item Рассматриваем два случая: $d < 1$ и $d > 1$.
\end{plan}
\begin{theorem}[формула Даламбера для вычисления радиуса сходимости СтР]
	Если существует конечный или бесконечный предел
	\begin{equation}
	\label{2.05}
	\limninf \abs{\dfrac{a_n}{a_{n+1}}},
	\end{equation}
	то для радиуса сходимости ряда \eqref{2.01} имеем:
	\begin{equation}
	\label{2.06}
	R = \limninf \abs{\dfrac{a_n}{a_{n+1}}}.
	\end{equation}
\end{theorem}
\begin{proof}$  $

	Без ограничения общности будем считать, что в \eqref{2.01} $ \forall \; a_n \neq 0 $.
	Т.к. СтР \eqref{2.01} сходится при $ x=x_0 $, то рассмотрим случай $ x \ne x_0 $.

	Если $ x \in I = \; \interval]{ \nullFrac x_0-R \;;\; x_0 + R \nullFrac}[ $, где $ R \geq 0 $, то по признаку Даламбера сходимости ЧР для \eqref{2.01} имеем:
	\begin{equation*}
	\exists \; d = \limninf \dfrac{\abs{a_{n+1} (x-x_0)^{n+1}}}{\abs{a_{n} (x-x_0)^{n}}} =
	\limninf \abs{\dfrac{a_{n+1}}{a_n}} \abs{x-x_0} \overset{\eqref{2.06}}{=} \dfrac{\abs{x-x_0}}{R}.
	\end{equation*}

	В силу того, что $ x \in I $ и, значит, $ \abs{x-x_0} < R$, получаем, что   $ d < 1 $ и СтР \eqref{2.01} будет сходящимся.
	Если $ d > 1 $, т.е. $ \abs{x-x_0} > R $, то \eqref{2.01} расходится.
	Таким образом, \eqref{2.06} будет радиусом сходимости для \eqref{2.01}.
\end{proof}

\newpage
\begin{col-answer-preambule}
\end{col-answer-preambule}

\colquestion{Формула Коши для вычисления радиуса сходимости СтР и замечания к ней.}
\begin{plan}
\item Рассмотрим  $x \neq x_0$
\item Применяем теорему Коши для ЧР
\item Рассматриваем два случая: $k < 1$ и $k > 1$.
\end{plan}
\begin{theorem}[формула Коши для вычисления радиуса сходимости СтР]
	Если существует конечный или бесконечный предел
	\begin{equation}
	\label{2.07}
	\limninf \sqrt[n]{\abs{a_n}},
	\end{equation}
	то для радиуса сходимости ряда \eqref{2.01} имеем:
	\begin{equation}
	\label{2.08}
	R = \dfrac{1}{\limninf \sqrt[n]{\abs{a_n}}}.
	\end{equation}
\end{theorem}
\begin{proofUndotted} проведём по той же схеме, что и в предыдущей теореме.

	Т.к. случай $ x=x_0 $ тривиален (в данной точке ряд всегда сходится), то рассмотрим случай $ x \ne x_0 $.

	По признаку Коши сходимости ЧР для \eqref{2.01} получаем:
	\begin{equation*}
	\exists \; k = \limninf \sqrt[n]{\abs{a_n (x-x_0)^n}} =
	\abs{x-x_0} \limninf \sqrt[n]{\abs{a_n}} \overset{\eqref{2.08}}{=} \dfrac{\abs{x-x_0}}{R}.
	\end{equation*}

	Если $ k < 1 $, т. е. $ \abs{x - x_0} < R $, то СтР \eqref{2.01} сходится.

	Если $ k > 1 $, т. е. $ \abs{x - x_0} > R $, то СтР \eqref{2.01} расходится.


	Таким образом, в силу определения, величина \eqref{2.08} будет радиусом сходимости для \eqref{2.01}.
\end{proofUndotted}

\begin{notes}
	\item В силу связи между признаками Даламбера и Коши сходимости ЧР, в случае, когда предел \eqref{2.06} не существует (ни конечный, ни бесконечный),
	предел \eqref{2.08} может существовать, и в этом смысле формула Коши \eqref{2.08} предпочтительнее, чем \eqref{2.06}.

	\item Можно показать, что в случае, когда в \eqref{2.08} нет ни конечного, ни бесконечного предела, радиус сходимости для \eqref{2.01} всегда можно вычислить по
	\textbf{формуле Коши-Адамара}, использующей понятие верхнего предела последовательности:
	\begin{equation}
	\label{2.09}
	R = \dfrac{1}{\;\; \overline{\limninf} \sqrt[n]{\abs{a_n}} \;\;}.
	\end{equation}
	Под верхним пределом последовательности подразумевается верхняя грань (supremum) множества конечных пределов всех сходящихся подпоследовательностей рассматриваемой последовательности.
\end{notes}

\colquestion{Теорема о локальной равномерной сходимости СтР, замечания к ней и следствие из неё (о равенстве степенных рядов).}
\colquestion{Теорема о дифференцировании СтР, замечания и следствие из неё.}
\colquestion{Теорема о замене переменной в несобственных интегралах (НИ) и замечание к ней.}
\colquestion{Формула двойной подстановки для НИ и интегрирование по частям в НИ.}
\colquestion{Признак существования равномерного частного предела для непрерывных Ф2П.}
\colquestion{Критерий Гейне равномерной сходимости Ф2П и замечания к нему.}
\colquestion{Теорема о предельном переходе в собственных интегралах, зависящих от параметра (СИЗОП) и замечания к ней.}
\colquestion{Теорема о почленном дифференцировании СИЗОП.}
\colquestion{Теорема о предельном переходе в несобственных интегралах, зависящих от параметра (НИЗОП), следствие из неё и замечание к ней.}
\colquestion{Теорема об интегрировании НИЗОП и замечания к ней.}
\colquestion{Теорема о почленном дифференцировании НИЗОП и замечание к ней.}
\colquestion{Вычисление интеграла Дирихле и его обобщения.}
\colquestion{Лемма Фруллани.}
\colquestion{Первая теорема Фруллани.}
\colquestion{Вторая теорема Фруллани.}
\colquestion{Третья теорема Фруллани.}
\end{document}
