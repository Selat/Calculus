\documentclass[a4paper]{article}
\usepackage[utf8]{inputenc}
\usepackage[russian]{babel}
\usepackage{mathtools}
\usepackage{amssymb}
\usepackage{amsthm}
\usepackage{tikz}
\usepackage{multicol}
\usepackage{xparse}
\usepackage{enumitem}
\usepackage{centernot}
\usepackage{comment}

\usepackage[left=1cm,right=1cm,top=1cm,bottom=1cm]{geometry}

\usepackage{setspace}
%\doublespace
\usepackage{epstopdf}
\usepackage{graphicx}
\usepackage{titlesec}

\newtheorem*{theorem}{Теорема}
\newtheorem*{lemma}{Лемма}

\newcommand{\norma}{}
\usepackage{mainstyle}
\renewcommand{\theenumii}{\asbuk{enumii}}

% For cyrillic symbols in "enumerate" environment.
\renewcommand{\theenumii}{\asbuk{enumii}}
\AddEnumerateCounter{\asbuk}{\@asbuk}{ы}


%====================================================================================
%   ENVIORONMENTS

\newenvironment{definition}
{\begin{statement}{Определение}}
    {\end{statement}}

\newenvironment{characteristics}
{\begin{statementItemed}{Свойства}}
    {\end{statementItemed}}

\newenvironment{note}
{\begin{statementDotted}{Замечание}

        }
    {\end{statementDotted}}

\newenvironment{notes}
{\begin{statementItemed}{Замечания}}
    {\end{statementItemed}}

\newenvironment{proofUndotted}
{{\raggedright \textit{Доказательство}}}
{\begin{flushright}
       \boxed{}
 \end{flushright}
}

\newenvironment{noteo}{}{}
\RenewDocumentEnvironment{noteo}{o}
{{\raggedright \textbf{Замечание}\IfValueTF{#1}{ (\textit{#1})}{}.}$  $

}
{}

\newenvironment{consequence}{}{}
\RenewDocumentEnvironment{consequence}{o}
{{\raggedright \textbf{Следствие}\IfValueTF{#1}{ (\textit{#1})}{}.}$  $

    }
{}

\NewDocumentEnvironment{consequences}{o}
{\raggedright \textbf{Следствия}\IfValueTF{#1}{(\textit{#1})}{}:\begin{enumerate}}
    {\end{enumerate}}


\RenewDocumentEnvironment{theorem}{o}
{{\raggedright
  \textbf{Теорема}\IfValueTF{#1}{ (\textit{#1})}{}}.$  $

}

\newenvironment{theoremNamed}{}{}
\RenewDocumentEnvironment{theoremNamed}{m o}
{{\raggedright
  \textbf{Теорема #1}\IfValueTF{#2}{ (\textit{#2})}{}}.$  $

}


\RenewDocumentEnvironment{lemma}{o}
{{\raggedright
        \textbf{Лемма}\IfValueTF{#1}{ (\textit{#1})}{}}.$  $

}

\newenvironment{lemmaNamed}{}{}
\RenewDocumentEnvironment{lemmaNamed}{m o}
{{\raggedright
        \textbf{Лемма #1}\IfValueTF{#2}{ (\textit{#2})}{}}.$  $

}


\newenvironment{exercise}
{\begin{statementDotted}{Упражнение}}
    {\end{statementDotted}}

\newenvironment{example}
{\begin{statementDotted}{Пример}}
    {\end{statementDotted}}

\newenvironment{examples}
{\begin{statementItemed}{Примеры}}
    {\end{statementItemed}}

% =================================
% math functions
\newcommand{\abs}[1]{
    \left\lvert #1 \right\rvert
}

\newcommand{\maxf}[1]{
    \max \left\{ #1 \right\}
}

\newcommand{\suchthat}{
    \;\ifnum\currentgrouptype=16 \middle\fi|\;
}

\newcommand{\arc}[1]{
    \buildrel\,\,\frown\over{#1}
}

\DeclareMathOperator{\const}{\text{const}}

\DeclareMathOperator{\fix}{\text{fix }}
\DeclareMathOperator{\diam}{diam\,}
\DeclareMathOperator{\mes}{mes}
\DeclareMathOperator{\divergence}{div}

\DeclareMathOperator{\arcsh}{arcsh}
\DeclareMathOperator{\arth}{{arth}}
\DeclareMathOperator{\arcth}{{arcth}}
\DeclareMathOperator{\sgn}{sgn}

\newcommand{\parenthesis}[1]{%
    \left( #1 \right)
}

\newcommand{\plot}[1]{$ \text{Г}_{#1} $}

\newcommand{\limlim}[2]
{ \lim\limits_{ \substack{ #1 \\ #2 } } }

\newcommand{\limitslimits}[2]
{ \limits_{ \substack{ #1 \\ #2 } } }

\newcommand{\limlimlim}[3]
{ \lim\limits_{ \substack{ #1 \\ #2 \\ #3 } } }

\newcommand{\nullFrac}{\dfrac{ }{}}

\renewcommand{\emptyset}{\varnothing}

\newcommand{\diint}{\displaystyle\iint}

\newcommand{\liml}{\lim\limits}
\newcommand{\intl}{\int\limits}
\newcommand{\iintl}{\iint\limits}
\newcommand{\diintl}{\diint\limits}
\newcommand{\dintl}{\dint\limits}

\newcommand{\suml}{\sum\limits}
\newcommand{\sumnzi}{\sum\limits_{n=0}^{\infty}}
\newcommand{\limninf}{\lim\limits_{n\to\infty}}
\newcommand{\dintlzi}{\dintl_0^{+\infty}}


%========================================================
% text style

\newcommand{\important}[1]{\textit{#1}}

\newcommand{\dint}{\displaystyle\int}
\newcommand{\dsum}{\displaystyle\sum}

\newcommand{\oiint}[2]{
    \begin{tikzpicture}[baseline=(C.base)]
        \node(C) {$ \displaystyle \iintl_{#1}^{#2} $};
        \draw (0,0.15) circle (0.25);
        %\node[draw,circle,inner sep=1pt](C) ++ (0, 0.1) {$ \;\;\;\; $};
    \end{tikzpicture}
}

\newcommand{\circled}[1]{
    \begin{tikzpicture}[baseline=(C.base)]
    \node[draw,circle,inner sep=1pt](C) {#1};
    \end{tikzpicture}
}

\newcommand{\eqlhopital}{%
    \overset{\circled{Л}}{=}}

\newcommand{\neqlhopital}{%
    \overset{\circled{Л}}{\neq}}

\newcommand{\sqcase}[1]{%
    \left[\begin{matrix}#1\end{matrix}\right]
}

\newcommand{\dvert}{\left.\nullFrac\right\vert}

\renewcommand{\r}[1]{$\overset{\text{ }_\bullet\text{ }}{\text{#1}}$}

\renewcommand{\norma}[1]{\left\lvert\left\lvert#1\right\rvert\right\rvert}

\newcommand{\RN}{\mathbb{R}^n}
\newcommand{\R}[1]{\mathbb{R}^{#1}}

% =================================
% sets utilities

\newcommand{\defineset}[2]{
    \left\{ #1 \, \middle\vert \, #2 \right\}
}

\newcommand{\set}[1]{
    \left\{ #1 \right\}
}

\newcommand{\colquestion}[1]{\section{#1}}
\newenvironment{col-answer-preambule}
               {\ignorespaces}
               {\ignorespacesafterend}

\begin{document}
\begin{center}
  \LARGE\underline{\textbf{Ответы к коллоквиуму по курсу}}\\
  \LARGE\underline{\textbf{ ``Математический анализ''}}\\
  \Large\textbf{(1-ый семестр 2015/2016 учебного года, специальность ``Информатикa'')}
\end{center}
\begin{col-answer-preambule}
	Обозначение поточечной сходимости:
	\begin{equation}
	\label{eq:lecture01-05}
	f_n(x) \overset{X}{\rightarrow}f(x)
	\end{equation}
	Определение \eqref{eq:lecture01-05} на $(\varepsilon-\delta)$-языке:
	\begin{equation}
	\label{eq:lecture01-06}
	\text{для } \forall \; \varepsilon > 0 \text{ и }
	\text{для } \forall \; fix \; x \in X \; \exists \; \nu = \nu(x, \varepsilon) \in \mathbb{R}\; | \text{ для } \forall \; n \geqslant \nu \Rightarrow \abs{f_n(x) - f(x)} \leqslant \varepsilon.
	\end{equation}
	Обозначение равномерной сходимости:
	\begin{equation}
	\label{eq:lecture01-07}
	f_n(x) \overset{X}{\rightrightarrows}f(x)
	\end{equation}
	Определение \eqref{eq:lecture01-07} на $(\varepsilon-\delta)$-языке:
	\begin{equation}
	\label{eq:lecture01-06fixed}
	\text{для } \forall \; \varepsilon > 0 \; \exists \; \nu = \nu(\varepsilon) \in \mathbb{R}\; | \text{ для } \forall \; fix \; x \in X  \text{ и } \text{для } \forall \; n \geqslant \nu\Rightarrow \abs{f_n(x) - f(x)} \leqslant \varepsilon.
	\end{equation}
\end{col-answer-preambule}

\colquestion{Супремальный критерий равномерной сходимости функциональных последовательностей (ФП) и замечания к нему}
\begin{theorem}[Супремальный критерий равномерной сходимости ФП]
	\begin{equation}
	\label{eq:lecture01-09}
	f_n(x) \overset{X}{\rightrightarrows}f(x) \Leftrightarrow
	r_n = \sup_{x \in X}\abs{f_n(x) - f(x)} \xrightarrow[n \to \infty]{}0.
	\end{equation}
\end{theorem}
\begin{proof}
	\circled{$\Rightarrow$} Если выполнена \eqref{eq:lecture01-07}, то, учитывая, что в \eqref{eq:lecture01-06fixed} используется $\forall \; fix \; x		 \in X$ и $\forall \; n \geqslant \nu(\varepsilon)$, получаем
	\begin{equation*}
	\begin{split}
	&r_n = \sup_{x \in X}\abs{f_n(x) - f(x)} \leqslant \varepsilon, \text{ т.е. }\\
	&\text{для }\forall \; \varepsilon > 0 \; \exists \; \nu = \nu(\varepsilon) \in \mathbb{R} \; \,\vert\, \text{ для }\forall \;
	n \geqslant \nu \Rightarrow 0 \leqslant r_n \leqslant \varepsilon, \text{ т.е. }
	r_n \xrightarrow[n \to \infty]{}0.
	\end{split}
	\end{equation*}\\
	\circled{$\Leftarrow$}
	Пусть выполнена \eqref{eq:lecture01-09}, тогда
	\begin{equation*}
	\begin{split}
	& \text{для }\forall \; \varepsilon > 0 \; \exists \; \nu = \nu(\varepsilon) \in \mathbb{R} \; \,\vert\, \text{ для }\forall \; n
	\geqslant \nu \text{ и для } \forall \; x \in X \Rightarrow \\
	& \Rightarrow \abs{f_n(x) - f(x)} \leqslant \sup\limits_{x \in X}\abs{f_n(x) - f(x)} = r_n \leqslant
	\varepsilon.
	\end{split}
	\end{equation*}
	Таким образом, имеем \eqref{eq:lecture01-06}, где $\nu$ зависит от $\forall \; \varepsilon > 0$ и
	не зависит от конкретного элемента множества $X$.
\end{proof}

\begin{notes}
	\item Если известно, что для $\forall \; n \in \mathbb{N}$ и для $\forall \; x \in X \Rightarrow
	\abs{f_n(x) - f(x)} \leqslant a_n$, где $\left(a_n\right)$ - б.м.п, то тогда имеем \eqref{eq:lecture01-07}.
	Сформулированное утверждение даёт \important{мажоритарный признак} (достаточное условие)
	равномерной сходимости ФП.
	\item Если
	\begin{equation*}
	\exists \; x_n \in X \; \,\vert\,\; g_n(x) = \abs{f_n(x) - f(x)} \Rightarrow g_n(x) \centernot{
		\xrightarrow[n \to \infty]{}} 0,
	\end{equation*}
	то тогда равномерной сходимости нет, т.е. $f_n(x) \centernot{\overset{X}{\rightrightarrows}} f(x)$. Это
	даёт достаточное условие (признак) неравномерной сходимости ФП.
\end{notes}
\newpage
\begin{col-answer-preambule}
	Обозначение равномерной сходимости ФР:
	\begin{equation}
	\label{eq:lecture01-20}
	\sum u_n(x) \overset{X}{\rightrightarrows}.
	\end{equation}
	Критерий Коши равномерной сходимости ФР:
	\begin{equation}
	\label{eq:lecture01-21}
	\eqref{eq:lecture01-20} \Leftrightarrow \text{для } \forall \; \varepsilon > 0 \; \exists \; \nu = \nu (\varepsilon) \in \mathbb{R} \; |
	\; \text{для } \forall \; x \in X \text{ и для } \forall \; n \geqslant \nu \; \text{и для }\forall \; m \in \mathbb{N} \Rightarrow 
	\abs{S_{n+m}(x) - S_n(x)} = \abs{ \sum\limits_{k = n + 1}^{n + m} u_k(x) } \leqslant \varepsilon.
	\end{equation}
	Критерий Коши сходимости ЧР:
	\begin{equation}
	\label{eq:lecture01-temp}
	\sum a_n \text{ сходится} \Leftrightarrow \text{для } \forall \;  \varepsilon>0 \ \exists \; \nu \in \mathbb{R} : \text{для }\forall \; n \geqslant \nu \; \text{ и} \;  \text{для }\forall\; m \in \mathbb{N} \Rightarrow \abs{S_{n+m} - S_n} = \abs{\sum\limits_{k = n + 1}^{n + m} a_k} \leqslant \varepsilon
	\end{equation}
	ЧП $\left(a_n\right)$ является \important{сходящейся числовой мажорантой} для ФР $\sum u_n (x)$, если:
	\begin{equation}
	\label{eq:lecture01-mazh-01}
		\text{1.  ЧР} \sum a_n \text{ сходится},
	\end{equation}
	\begin{equation}
	\label{eq:lecture01-mazh-02}
	\text{2. для }\forall \; n \in \mathbb{N}  \text{ и для } \forall \; x \in X \Rightarrow |u_n(x)| \leqslant a_n.
	\end{equation}
\end{col-answer-preambule}

\colquestion{Мажорантный признак Вейерштрасса равномерной сходимости функционального ряда (ФР) и замечания к нему}
\begin{theorem}[мажорантный признак Вейерштрасса равномерной сходимости ФР] Если ФР имеет на $X$ сходяющуюся числовую мажоранту, то он равномерно сходится на $X$.
\end{theorem}
\begin{plan}
	\item По def \important{сходящейся числовой мажоранты}.
	\subitem Расписать \eqref{eq:lecture01-mazh-01} по \eqref{eq:lecture01-temp}: $\abs{\sum\limits_{k = n+1}^{n+m} a_k} \leqslant \varepsilon$.
	\subitem Подставить \eqref{eq:lecture01-mazh-02} в \eqref{eq:lecture01-21}.
\end{plan}
\begin{proof}
	Доказательство с использованием критерия Коши сходимости ЧП \eqref{eq:lecture01-temp} и критерия Коши равномерной сходимости ФР \eqref{eq:lecture01-21}:

	Т.к. $\sum a_n$ сходится, то
	\begin{equation}
	\label{eq:1_23}
	\text{для } \forall \; \varepsilon > 0 \; \exists \; \nu(\varepsilon) \in \mathbb{R} \; | \; \text{для } \forall \;	n \geqslant \nu \text{ и для } \forall \; m \in \mathbb{N} \Rightarrow \abs{\sum_{k = n+1}^{n+m} a_k} \leqslant \varepsilon.
	\end{equation}

	Если $\text{для } \forall \; n \in \mathbb{N} \text{ и для } \forall \; x \in X \Rightarrow \abs{u_n(x)} \leqslant a_n$, то для частичных сумм ФР $\sum u_n(x)$ имеем: \newline $\abs{S_{m+n} (x) - S_n (x)} = \abs{\sum\limits_{k = n+1}^{n+m} u_k(x)} \leqslant \sum\limits_{k = n + 1}^{n+m} \abs{u_k (x)} \leqslant \sum\limits_{k = n+1}^{n+m} a_k = \abs{\sum\limits_{k = n+1}^{n+m} a_k} \leqslant \varepsilon$, это для $\forall \; n \geqslant \nu = \nu(\varepsilon) \text{ и для } \forall \; m \in \mathbb{N}$, что в силу \eqref{eq:lecture01-21} даёт \eqref{eq:lecture01-20}.
\end{proof}

\begin{notes}
	\item Признак Вейерштрасса является лишь достаточным условием равномерной сходимости ФР. На практике сходящуюся числовую мажоранту $\left( a_n \right)$ либо находят с помощью соответствующих оценок $\abs{u_n(x)}$ сверху, либо берут $a_n = \underset{x \in X}{sup} \abs{u_n(x)}$. В последнем случае получаем наиболее точную мажоранту, но в случае расходимости $\sum a_n$ даже для этой самой точной мажоранты ничего о равномерной сходимости ФР сказать нельзя, т.е. требуются дополнительные исследования.
	\item Обобщая признак Вейерштрасса, где используется сходимость числовой мажоранты - признак равомерной сходимости ФР, используют функцию мажоранты, а именно получаем:
	\begin{equation*}
	 \text{если } \exists \; v_n(x) \geqslant 0 \; : \; \abs{u_n(x)} \leqslant v_n(x) \; \text{для } \forall \; n \in \mathbb{N} \text{ и для } \forall \; x \in X \text{ и } \sum v_n(x)	 \overset{X}{\rightrightarrows},
	 \end{equation*}
	 то тогда для ФР $\sum u_n(x)$ имеем \eqref{eq:lecture01-20}.
\end{notes}

\newpage
\begin{col-answer-preambule}
	\begin{plan}
		\item Формулировка: из названия (как и Дирихле для рядов).
		\item Доказательство:
		\subitem оценка Абеля, взятая c 2-кой для надёжности.
		\subitem оценить $\abs{b_{n+1}}$ и $\abs{b_{n+m}}$ по $\widetilde{\varepsilon} = \dfrac{\varepsilon}{6 \cdot c}$
		\subitem def равномерной сходимости для $\sum a_n(x) b_n(x)$.
		\item Замечания: как и для рядов ($\sum (-1)^n b_n(x)	\overset{X}{\rightrightarrows}$, $\text{Лейбница} \approx \text{единица}$).
	\end{plan}
\end{col-answer-preambule}
\colquestion{Признак Дирихле равномерной сходимости ФР и следствие из него (признак Лейбница равномерной сходимости ФР)}
\begin{theorem}[Признак Дирихле равномерной сходимости ФР] Пусть для ФП $a_n(x)$ частичные суммы $\sum a_n(x)$ ограничены в совокупности (равномерно на $X$), т.е.
	\begin{equation}
	\label{eq:1_24}
	\text{для }\forall \; x \in X \text{ и для } \forall \; n \in \mathbb{N} \Rightarrow \abs{a_1 (x) + a_2(x) + \ldots + a_n(x)} \leqslant c,
	\end{equation}
	где $c = const > 0$, \important{не зависит ни от $n$, ни от $x$}. Если $\forall \; fix \; x \in X \Rightarrow \left( b_n(x) \right)$ - функциональная последовательность является монотонной, то в случае
	\begin{equation}
	\label{eq:1_25}
	b_n(x) \overset{X}{\rightrightarrows} 0,
	\end{equation}
	имеем $\sum a_n(x) b_n(x) \overset{X}{\rightrightarrows}$.
\end{theorem}
\begin{proof}
	Монотонная последовательность $\left( b_n(x) \right) \text{ для } \forall \; fix \; x \in X$ позволяет так же, как и в ЧР, использовать на основе \eqref{eq:1_24} оценку Абеля:
	\begin{equation}
	\label{eq:1_26}
	\abs{\sum_{k = n+1}^{n+m} a_k(x) b_k(x)} \leqslant 2 \text{\important{c}} \left(\abs{b_{n+1}(x)} + 2 \abs{b_{n+m} (x)}\right).
	\end{equation}

	Если выполняется \eqref{eq:1_25}, то тогда имеем:
	\begin{equation*}
	\text{ для } \forall \; \varepsilon > 0 \text{ по числу } \tilde{\varepsilon} = \dfrac{\varepsilon}{6 c} > 0 \; \exists \; \nu (\varepsilon) \in \mathbb{R} \; | \; \text{для } \forall \; n \geqslant \nu(\varepsilon) \text{ и для } \forall \; m \in \mathbb{N} \text{ и для } \forall \; x \in X \Rightarrow \abs{b_{n+1} (x) } \leqslant \tilde{\varepsilon} \text{ и } \abs{b_{n+m} (x)} \leqslant \tilde{\varepsilon}, 
	\end{equation*}
	поэтому для частичных сумм $S_n(x) = \sum\limits_{k=1}^{n} a_k(x) b_k(x)$ в силу \eqref{eq:1_26} $\text{для } \forall \; n \geqslant \nu(\varepsilon) \text{ и для } \forall \; m \in \mathbb{N} \text{ и для } \forall \; x \in X$ имеем: $\abs{S_{n+m} (x)  - S_n(x)} = \abs{\sum\limits_{k = n+1}^{n+m} a_k(m) b_k(x)} \leqslant 2 \cdot c \cdot(\tilde{\varepsilon} 	+ 2 \tilde{\varepsilon} ) = 6\cdot c \cdot  \tilde{\varepsilon} = \varepsilon$. Отсюда по критерию Коши равномерной сходимости ФР следует, что $\sum\limits a_n(x) b_n(x) \overset{X}{\rightrightarrows}$.
\end{proof}
\begin{consequence}[Признак Лейбница равномерной сходимости ФР]
	Если $\forall \; fix \; x \in X$ последовательность $\left(b_n(x)\right)$ является монотонной, то в случае $b_n(x) \overset{X}{\rightrightarrows}
	0 \Rightarrow 	\sum (-1)^n b_n(x)	\overset{X}{\rightrightarrows}$.
\end{consequence}
\begin{proof}
	Следует из того, что в условии теоремы $a_n = (-1)^n$ не зависит от $x$, причём \newline $\abs{\sum\limits_{k=1}^{n} a_k} \leqslant 1 = const, \text{для } \forall \; n \in \mathbb{N}$.
\end{proof}

\newpage
\begin{col-answer-preambule}
	Для обозначения поточечной сходимости ФР $\sum u_n(x)$ на X будем использовать запись:
	\begin{equation}
	\label{eq:lecture01-14}
	\sum u_n(x) \overset{X}{\rightarrow}.
	\end{equation}
	\begin{plan}
		\item Формулировка: \important{оДин}и - один знак, Ди\important{ни} - непрерывны, Ди\important{ни} - непрерывны.
		\item Доказательство:
		\subitem 3 свойства остатка ряда $R_n(x) = S(x) - S_n(x)$: Fun UFO (\textcolor{magenta}{F}u\textcolor{magenta}{n} UFO — функция непрерывна, Fu\textcolor{magenta}{n} \textcolor{magenta}{U}FO — функциональная последовательность убывает, Fun U\textcolor{magenta}{FO} — функция к 0).
		\subitem дм у пво (\textcolor{magenta}{д}е \textcolor{magenta}{М}орган, \textcolor{magenta}{у}прощение, \textcolor{magenta}{п}ринцип \textcolor{magenta}{в}ыбора, $x_{\textcolor{magenta}{0}}$)
		\subitem противоречие с последним свойством остатка.
		\subitem $R_{m} (x_{nk}) \geqslant R_{nk} (x_{nk}) > \varepsilon_0 \Rightarrow [\text{ переходя к пределу }] \Rightarrow R_m (x_0) = \lim\limits_{n_k \to \infty} R_m(x_{nk}) \geqslant \varepsilon_0$, что противоречит последнему из свойств остатка.
		\item Теорема: то же самое, только вместо сохранения одного знака члены ФП будут монотонны.
			\subitem по доказанному признаку, задав ФР как $u_n(x) = f_n(x) - f_{n-1}(x)$.
	\end{plan}
\end{col-answer-preambule}
\colquestion{Признак Дини равномерной сходимости ФР и следствие из него (теорема Дини для ФП)}
\begin{theorem}[Признак Дини равномерной сходимости ФР]
	Пусть
	\begin{enumerate}
		\item Члены ФР $\sum u_n(x)$ непрерывны и сохраняют один и тот же знак на $X = [a, b], \text{ для } \forall \; n \in \mathbb{N}$.
		\item $\sum u_n(x) \overset{X}{\to} S(x)$.
	\end{enumerate}
	Тогда, если $S(x) = \sum\limits_{n=1}^{\infty} u_n (x)$ - непрерывная функция на $[a, b]$, т.е. $S(x) \in C([a, b])$, то $\sum u_n(x) \overset{X}{\rightrightarrows}$.
\end{theorem}
\begin{proof}
	Рассмотрим на $X = [a,b]$ остатки ряда $R_n(x) = u_{n+1}(x) + \ldots = S(x) - S_n(x)$.	Нетрудно видеть, что выполняются следующие свойства:
	\begin{enumerate}
		\item для $\forall \; fix \; n \in \mathbb{N} \Rightarrow R_n (x)$ - непрерывная функция на $[a,b]$ как разность двух непрерывных функций.
		\item для $\forall \; fix \; x \in X \Rightarrow$ $\text{ФП}$ $(R_n(x))$ убывает в случае, когда $\forall \; u_n (x) > 0$, т.к. \newline $R_n(x) = u_n(x) + R_{n+1}(x) \geqslant R_{n+1}(x), \text{ для } \forall \; n \in \mathbb{N}$.
		\item Т.к. имеет место \eqref{eq:lecture01-14}, то для $\forall \; fix \; x \in X \Rightarrow R_n(x) \overset{X}{\to} 0$.
	\end{enumerate}
	Докажем от противного. Предположим, что рассматриваемая положительная поточечная сходимость на $X$ ФР не является равномерной сходимостью на $X$.

	Тогда по правилу де Моргана имеем: $\exists \; \varepsilon_0 > 0 \; | \; \text{для } \forall \; \nu \in \mathbb{R} \; \exists \; n (\nu) \geqslant 0 \text{ и } \exists \; x (\nu) \in X \; | \; R_{n \nu} (x_\nu) > \varepsilon_0$. Для простоты будем считать, что $\exists \; x_n \in X \; | \; R_n (x_n) > \varepsilon_0$. По принципу выбора из ограниченной последовательности $x_n$ можно выбрать сходящуюся подпоследовательность, т.е. $x_{n_k} \underset{n_k \to \infty}{\longrightarrow} x_0$, при этом в силу использования $X = [a,b]$ - компакт, получаем, что $x_0 \in X$. Если зафиксируем $m \in \mathbb{N}$, то для $\forall \; n_k \geqslant m \Rightarrow R_{n_k} (x_{n_k}) > \varepsilon_0$, по свойствам остаткам будем иметь, что $R_{m} (x_{n_k}) \geqslant R_{n_k} (x_{n_k}) > \varepsilon_0$. В неравенстве $R_m (x_{n_k}) > \varepsilon_0$, переходя к пределу при $n_k \to \infty \; \text{для } \forall \; m \in \mathbb{N}$, получаем в силу непрерывности $R_n(x): R_m (x_0) = \lim\limits_{n_k \to \infty} R_m(x_{n_k}) \geqslant \varepsilon_0$, что противоречит последнему из свойств остатка, а именно $R_m(x_0) \overset{X}{\longrightarrow} 0$ при $m \to \infty$, поэтому из нашего предположения следует, что выполняется $R_m(x_0) \not\to 0$, противоречие, т.е. выполняется $\sum u_n(x) \overset{X}{\rightrightarrows}$.
\end{proof}
\begin{consequence}[Теорема Дини для ФП]
	Если для ФП $f_n(x), n \in \mathbb{N}$ на $X = [a,b]$ выполняются свойства:
	\begin{enumerate}
		\item для $\forall \; f_n(x) \in C([a,b])$ и для	 $\forall \; fix \; x \in X \Rightarrow f_n(x)$ монотонна.
		\item $f_n(x) 	\overset{X}{\longrightarrow}f(x)$. Тогда, если $f(x) \in C([a,b])$, то $f_n(x) \overset{X}{\rightrightarrows}$.
	\end{enumerate}
\end{consequence}
\begin{proof}
	следует из того, что члены рассматриваемой ФП $f_n(x)$ можно рассматривать как частичные суммы соответствующего ФР с общим членом
	\begin{equation}
	\label{eq:1_27}
	\begin{cases}
	u_n(x) = f_n(x) - f_{n-1}(x),\\
	f_0(x) = 0.
	\end{cases}
	\end{equation}
	Действительно, $S_n(x) = f_n(x) - f_0(x) = f_n(x), \text{ для } \forall \; n \in \mathbb{N}$.

	А далее к соответствующему ФР применима теорема Дини равномерной сходимости ФР.
\end{proof}

\newpage
\begin{col-answer-preambule}
Пусть $x_0$ - предельная точка множества сходимости $X \subset \mathbb{R}$ для ФР $\sum u_n(x)$.
Будем говорить, что в $\sum u_n(x)$ \important{возможен почленный предельный переход} $x \to x_0$, если
\begin{equation}
\label{eq:lecture01-28}
\exists \; lim \sum_{n = 1}^{\infty} u_n(x) = \sum_{n = 1}^{\infty} lim u_n(x),
\end{equation}
причём получившийся в левой части \eqref{eq:lecture01-28} ЧР является сходящимся.

В частности, если $x_0 \in X$ и $\forall \; u_n(x)$ непрерывен в некоторой окрестности точки $x_0$, и значит, для $\forall \; n \in \mathbb{N} \; \exists \lim\limits_{x \to x_0} u_n(x) = u_n(x_0)$, то в случае выполнения \eqref{eq:lecture01-28} для суммы $S(x)$ ФР $\sum u_n(x)$ при $x \to x_0$ имеем:
\begin{equation}
\label{eq:lecture01-29}
\exists \; \lim\limits_{x \to x_0} S(x) = \lim\limits_{x \to x_0} \sum_{n=1}^{\infty} u_n(x) = \sum_{n=1}^{\infty} \lim\limits_{x \to x_0} u_n(x) = \sum_{n=1}^{\infty} u_n(x_0) = S(x),
\end{equation}
что соответствует непрерывности $S(x)$ в точке $x_0 \in X$.
\end{col-answer-preambule}

\colquestion{Теорема о непрерывности суммы равномерно сходящегося ФР и замечания к ней}
\begin{theorem}[о непрерывности суммы равномерно сходящегося ФР]
	Если все члены $u_n(x), n \in \mathbb{N}$, ФР $\sum u_n(x)$ непрерывны на $X = [a,b]$, то в случае равномерной сходимости этого ряда на $[a,b]$ его сумма $S(x)$ будет непрерывной функцией на $[a,b]$.
\end{theorem}
\begin{proof}
	Требуется обосновать \eqref{eq:lecture01-29} для $\forall \; x_0 \in [a,b]$, причём в случае концевых значений $x_0 = a, \; x_0 = b$ будем использовать соответствующие односторонние пределы, т.е. рассматривать одностороннюю непрерывность.

	Для $fix \; x_0 \in [a,b]$ придадим произвольные приращения $\Delta x \in \mathbb{R} \; | \; (x_0 + \Delta x) \in [a,b]$ и рассмотрим соответствующие приращения суммы ФР $\sum u_n(x)$:
    \begin{equation*}
        \Delta S(x_0) = S(x_0 + \Delta x) - S(x_0).
    \end{equation*}

    Из равномерной сходимости ФР $\sum u_n(x)$ на
    $X = [a,b] \Rightarrow \text{для } \forall \; \varepsilon > 0 $
    ${ \exists \; \nu = \nu(\varepsilon) \in \mathbb{R} \; | \; \text{для } \forall \; n \geqslant \nu }$,
    $ \text{ и для } \forall \; x \in [a,b]$ для частичных сумм $S_n(x) = u_1(x) + u_2(x) + \ldots + u_n(x)$ ряда $\sum u_n(x)$ имеем: $\abs{S_n(x) - S(x)} \leqslant \varepsilon$.

	Отсюда, в частности, для $x = x_0 \in X $ и $ x = x_0 + \Delta x \in X \Rightarrow$
	\begin{equation}
	\label{eq:1_30}
	\begin{cases}
	\abs{S_n(x_0) - S(x_0)} \leqslant \varepsilon, \\
	\abs{S_n(x_0 + \Delta x) - S(x_0 + \Delta x)} \leqslant \varepsilon.
	\end{cases}
	\end{equation}

	Далее из непрерывности $\forall \; u_n(x)$ в $x_0 \in [a,b]$ следует непрерывность частичных сумм в $x_0$ (как конечных сумм непрерывных функций).

	В силу этого, для $ \forall \; \varepsilon \; \exists \; \delta > 0 \; : \; \text{для } \forall \; \abs{\Delta x} \leq \delta \Rightarrow$
	\begin{equation}
    	\label{eq:1_31}
    	\Rightarrow \abs{S_n(x_0 + \Delta x) - S_n(x_0)} \leqslant \varepsilon.
	\end{equation}

	Таким образом,  в силу \eqref{eq:1_30}, \eqref{eq:1_31} имеем: для $\forall \varepsilon > 0$, выбирая $n \geqslant \nu$ и рассматривая $\forall \abs{\Delta x}	 \leqslant \delta$, имеем:

	$\abs{\Delta S(x_0)} = \abs{S_n(x_0) - S(x_0) + S_n(x_0 + \Delta x) - S_n(x_0) + S (x_0 + \Delta x) - S_n(x_0 + \Delta x)} \leqslant \\  \leqslant \abs{S_n(x_0) - S(x_0)} + \abs{S_n(x_0 + \Delta x) - S_n(x_0)} + \abs{S(x_0 + \Delta x	) - S_n(x_0 + \Delta x)} \leqslant \varepsilon + \varepsilon + \varepsilon = 3 \cdot \varepsilon$.

    Поэтому получаем:
	для	$\forall \; \varepsilon \; \exists \; \delta > 0 : \; \text{для } \forall \; \abs{\Delta x} \leqslant \delta \Rightarrow \abs{\Delta S(x_0)} \leqslant M \cdot \varepsilon, M = const = 3 > 0$.

	Отсюда по М-лемме для Ф1П следует, что $\Delta S(x_0) \underset{\Delta x \to 0}{\to} 0$, что на языке приращений равносильно \eqref{eq:lecture01-29}. При этом, т.к. из равномерной сходимости следует поточечная сходимость ЧР в правой части \eqref{eq:lecture01-29} будет сходящимся.
\end{proof}

\begin{notes}
	\item Доказанную теорему часто называют теоремой Стокса-Зейделя или теоремой Стокса-Зайделя.
	\item В условии доказанной теоремы равномерную сходимость можно заменить для произвольного множества $ X \subset \mathbb{R}$ на локальную равномерную сходимость. Будем говорить, что ФР $\sum u_n(x)$ сходится локально равномерно на $X \subset \mathbb{R}$, если для $\forall \; [a,b] \subset X \Rightarrow \sum u_n(x) (x) \overset{[a,b]}{\rightrightarrows}$. У $\sum u_n(x)$ может быть локальная равномерная сходимость на $X$, но может не быть полной (???) равномерной сходимости на $X$. В случае локальной равномерной сходимости $\sum u_n(x)$ на $X$ берём $\forall x_0 \in X$ и заключаём её в некоторый отрезок $x_0 \in [a,b] \subset X$. Т.к. есть равномерная сходимость для $\sum u_n(x)$ на этом отрезке, то по доказанной теореме сумма $S(x)$ в случае непрерывности $\forall \; u_n(x)$ на $X$ будет непрерывна на $[a,b] \subset X$ и, в частности, непрерывна в $x_0 \in X$, а т.к. это можно сделать для $\forall \; x_0 \in X$, то тем самым получаем непрерывность $S(x)$ на $X \subset \mathbb{R}$ даже в случае, когда нет равномерной сходимости ФР на $X$.
\end{notes}

\newpage
\colquestion{Теорема о почленном интегрировании равномерно сходящегося ФР}
\begin{plan}
\item Очевидно, что $S(x)$ - непрерывна, поэтому интегрируема
\item Рассмотрим частичные суммы $T_n = \suml_{k=1}^{n}\int u_k(x)dx$.
\item Рассмотрим разницу $\abs{T_n - \intl_a^bS(x)}$ и т.к. $\abs{S(x) - S_n(x)} \leqslant \varepsilon$ получим $\intl_a^b (S(x) - S_n(x)) \leqslant M \varepsilon$
\item Доказываем по М-лемме о сходимости ЧП.
\end{plan}
\begin{theorem}[о почленном интегрировании равномерно сходящихся ФР]
	Если $\forall \; u_n(x) \in C([a,b]), $ \\ для $n \in \mathbb{N}$, то в случае, когда $\sum u_n(x) \overset{[a,b]}{\rightrightarrows}$, возможно почленное интегрирование этого ряда на $[a,b]$, т.е.
	\begin{equation}
	\label{eq:lecture01-32}
	\exists \dint\limits_a^b S(x)dx = \dint\limits_a^b \left(\sum_{n=1}^{\infty}u_n(x)\right)dx = \sum_{n=1}^{\infty} \dint\limits_a^b u_n(x)dx.
	\end{equation}
\end{theorem}
\begin{proof}
	На основании теоремы о непрерывности суммы равномерно сходящихся ФР получим, что сумма ряда $S(x) = \sum\limits_{n=1}^{\infty}u_n(x)$ будет непрерывна на $[a,b]$, а значит, интегрируема на $[a,b]$.

	Используя частичные суммы для $\sum u_n(x)$, рассмотрим частичные суммы $T_n = \dint\limits_a^b S_n(x)dx  =$ \\$= 	\dint\limits_a^b \sum_{k=1}^{n} u_k(x)dx = \sum\limits_{k=1}^{n} \dint\limits_a^b u_k(x)dx$ для ЧР правой части \eqref{eq:lecture01-32}.

	Требуется доказать, что $\lim\limits_{n \to \infty} T_n = \dint\limits_a^b S(x)dx$.

    Из равномерной сходимости $\sum u_n(x)$ на $[a,b]$ получим, что для $\forall \; \varepsilon > 0 \; \exists \; \nu = \nu(\varepsilon) \; | \; \text{для } \forall \; n \geqslant \nu $ и для $ \forall \; x \in [a,b] \Rightarrow$
	\begin{equation}
	\label{eq:1_33}
	 	\abs{S(x) - S_n(x)} = \abs{\sum_{k = n+1}^{\infty} u_k(x)} \leqslant \varepsilon
	\end{equation}

	Отсюда получаем, что $\abs{\dint\limits_a^b S(x)dx - T_n} = \abs{\dint\limits_a^b S(x)dx - \dint\limits_a^b S_n(x)dx} =  \abs{\dint\limits_a^b (S(x) - S_n(x))dx} \leqslant $ \\ $\leqslant \dint\limits_a^b \abs{S(x) - S_n(x)}dx \leqslant \dint\limits_a^b \varepsilon dx = M \varepsilon$, где $M = b - a = const \geqslant 0$.

	Таким образом, для $\forall \; \varepsilon > 0 \; \exists \; \nu = \nu(\varepsilon) \; | \; \text{ для } \forall \; n \geqslant \nu \Rightarrow \abs{\dint\limits_a^b S(x) dx - T_n} \leqslant M \varepsilon$, поэтому по М-лемме сходимости ЧП следует, что
	\begin{equation*}
    	\exists \lim\limits_{n \to \infty} T_n = \dint\limits_a^b S(x)dx = \dint\limits_a^b \left(\sum\limits_{k=1}^{\infty} u_k(x)\right) dx,
	\end{equation*}
	что равносильно \eqref{eq:lecture01-32}.
\end{proof}

\newpage
\colquestion{Теорема о почленном дифференцировании ФР}

\begin{theorem}[о почленном дифференцировании ФР]
	Пусть ФР $\sum u_n(x)$ на $X = [a,b]$ удовлетворяет условиям:
	\begin{enumerate}
		\item $\sum u_n(x) \overset{X}{\rightarrow}$,
		\item $\exists \; u_n^{'}(x)$, непрерывная для $\forall \; n \in \mathbb{N}, x \in X$.
	\end{enumerate}
	Тогда, если
    \begin{equation}
        \label{eq:1_34}
        \sum u_n^{'}(x) \overset{X}{\rightrightarrows}
    \end{equation}
    то рассматриваемый ФР $\sum u_n(x)$ можно почленно дифференцировать на $[a,b]$, т.е.
	\begin{equation}
	\label{eq:1_35}
	\exists \left( \sum_{n=1}^{\infty} u_n (x) \right)^{'} = \sum_{k=1}^{\infty} u_k^{'}(x), \text{для }\forall \; x \in X.
	\end{equation}
\end{theorem}
\begin{proof}
	В силу \eqref{eq:1_34}, по условию 2 рассматриваемой теоремы получаем, что по теореме об интегрировании ФР $\sum u_n^{'}(t)$ можно почленно интегрировать на
    	$\forall \; [a,x] \subset [a,b]$, \nolinebreak т.е.

    \begin{equation*}
        \exists \dint\limits_a^x \left(\sum\limits_{n=1}^{\infty} u_n^{'}(t)\right)dt = \sum_{n=1}^{\infty} \dint\limits_a^x u_n^{'}(t)dt = \sum_{n=1}^{\infty} [u_n]^{t = x}_{t = a} = \sum\limits_{n=1}^{\infty}\left(u_n(x) - u_n(a)\right).
    \end{equation*}

	Отсюда в силу условия 1 (поточечная сходимость для $\sum u_n(x)$) получаем, что
    \begin{equation*}
            \exists \; S(x) = \sum\limits_{n=1}^{\infty} u_n(x) = \sum\limits_{n=1}^{\infty}u_n(a) + \dint\limits_a^x \sum\limits_{n=1}^{\infty}u_n^{'}(t)dt.
    \end{equation*}

	Используя далее \important{теорему Барроу} о дифференцировании интеграла с переменным верхним пределом от непрерывной подынтегральной функции, получаем:\\
	\begin{equation*}
	\exists \; S^{'}(x) = (const)^{'} + \left(\dint\limits_a^x \left( \sum\limits_{n=1}^{\infty} u_n^{'} (t) \right)dt \right)^{'}_x = \sum\limits_{n=1}^{\infty}u_n^{'} (x),
	\end{equation*} что соответствует \eqref{eq:1_35}.
\end{proof}

\newpage
\begin{col-answer-preambule}
Под \important{степенным рядом} будем подразумевать ФР вида
\begin{equation}
\label{2.01}
a_0 + a_1(x-x_0) + a_2(x-x_0)^2 + \ldots +  a_n(x-x_0)^n + \ldots
= \sum_{n=0}^{\infty} a_n(x-x_0)^n,
\end{equation}
где $ \fix x_0 \in \R{} $ - центр для СтР, а $ \forall \;	 a_n \in \R{} $ - соответствующая числовая последовательность (\important{коэффициенты СтР}).
\end{col-answer-preambule}

\colquestion{Теорема Абеля о сходимости степенного ряда (СтР) и замечание к ней.}
\begin{plan}
\item \textcolor{magenta}{А}беля - сходится \textcolor{magenta}{а}бсолютно, признак сравнения ЧР.
\item Сходящаяся ЧП является ограниченной (т.е. ограничен каждый её член)
\item Рассматриваем это условие для $x_1$, получаем верхнюю границу для $a_n$.
\item Затем аналогично рассматриваем условие для $x$, ограничивая сверху $M q^n$.
\end{plan}
\begin{statementDotted}{Теорема Абеля}[о сходимости степенных рядов]

	Если СтР \eqref{2.01} сходится при $ x = x_1 \neq x_0 $, то он будет сходится абсолютно для любого $ x $, где
	\begin{equation}
	\label{2.02}
	\abs{x-x_0} < \abs{x_1 - x_0}.
	\end{equation}

\end{statementDotted}
\begin{proof}
	Из сходимости при $x = x_1$, т.е. ряда $ \sumnzi a_n(x_1-x_0)^n $ следует в силу необходимого условия сходимости ЧР, что $ a_n(x_1-x_0)^n \xrightarrow[n \to \infty]{} 0$ ,
	а т.к. $\forall$ сходящаяся ЧП является ограниченной, то \newline
	$ \exists \; M  = \const > 0 :
	\abs{a_n(x_1-x_0)^n} \leq M, \text{ для } \forall \; n \in \mathbb{N}$, т. е.
	\begin{equation}
	\label{2.03}
	\abs{a_n} \leq \dfrac{M}{\abs{x_1-x_0}^n}.
	\end{equation}

	Для $\forall \; x $, удовлетворяющего \eqref{2.02}, в силу \eqref{2.03} получаем:
	\begin{equation*}
	\abs{a_n (x - x_0)^n} = \abs{a_n} \abs{x - x_0}^n \overset{\eqref{2.03}}{\leq}
	\dfrac{M \abs{x - x_0}^n}{\abs{x_1 - x_0}^n} = Mq^n,
	\text{ где } q = \dfrac{\abs{x - x_0}}{\abs{x_1 - x_0}} \in [0;1[.
	\end{equation*}

	Таким образом, мы получили сходящуюся мажоранту, ибо ряд $ \sumnzi Mq^n = M \sumnzi q^n $ сходится при $ q \in [0;1[ $.

	По признаку сравнения сходимости ЧР имеем, что для $ \forall \; x $, удовлетворяющего \eqref{2.02}, ряд \eqref{2.01} будет сходиться.
\end{proof}
\begin{note}
	Из полученных выше результатов следует, что если рассмотреть множество $ X_0 $ всех $ x $, удовлетворяющих \eqref{2.02}, то имеем, что $ X_0 \subset X $, т.е. $X_0$ - некоторое подмножество множества $X$ сходимости для \eqref{2.01}.
\end{note}

\newpage
\begin{col-answer-preambule}
\end{col-answer-preambule}

\colquestion{Формула Даламбера для вычисления радиуса сходимости СтР.}
\begin{plan}
\item Рассматриваем $x \in \interval]{-R + x_0; x_0 + R}[ (x \neq x_0)$ .
  \item Подставляем в теорему Даламбера для ЧР ($a_{n+1} / a_n$).
    \item Рассматриваем два случая: $d < 1$ и $d > 1$.
\end{plan}
\begin{theorem}[формула Даламбера для вычисления радиуса сходимости СтР]
	Если существует конечный или бесконечный предел
	\begin{equation}
	\label{2.05}
	\limninf \abs{\dfrac{a_n}{a_{n+1}}},
	\end{equation}
	то для радиуса сходимости ряда \eqref{2.01} имеем:
	\begin{equation}
	\label{2.06}
	R = \limninf \abs{\dfrac{a_n}{a_{n+1}}}.
	\end{equation}
\end{theorem}
\begin{proof}$  $

	Без ограничения общности будем считать, что в \eqref{2.01} $ \forall \; a_n \neq 0 $.
	Т.к. СтР \eqref{2.01} сходится при $ x=x_0 $, то рассмотрим случай $ x \ne x_0 $.

	Если $ x \in I = \; \interval]{ \nullFrac x_0-R \;;\; x_0 + R \nullFrac}[ $, где $ R \geq 0 $, то по признаку Даламбера сходимости ЧР для \eqref{2.01} имеем:
	\begin{equation*}
	\exists \; d = \limninf \dfrac{\abs{a_{n+1} (x-x_0)^{n+1}}}{\abs{a_{n} (x-x_0)^{n}}} =
	\limninf \abs{\dfrac{a_{n+1}}{a_n}} \abs{x-x_0} \overset{\eqref{2.06}}{=} \dfrac{\abs{x-x_0}}{R}.
	\end{equation*}

	В силу того, что $ x \in I $ и, значит, $ \abs{x-x_0} < R$, получаем, что   $ d < 1 $ и СтР \eqref{2.01} будет сходящимся.
	Если $ d > 1 $, т.е. $ \abs{x-x_0} > R $, то \eqref{2.01} расходится.
	Таким образом, \eqref{2.06} будет радиусом сходимости для \eqref{2.01}.
\end{proof}

\newpage
\begin{col-answer-preambule}
\end{col-answer-preambule}

\colquestion{Формула Коши для вычисления радиуса сходимости СтР и замечания к ней.}
\begin{plan}
\item Рассмотрим  $x \neq x_0$
\item Применяем теорему Коши для ЧР
\item Рассматриваем два случая: $k < 1$ и $k > 1$.
\end{plan}
\begin{theorem}[формула Коши для вычисления радиуса сходимости СтР]
	Если существует конечный или бесконечный предел
	\begin{equation}
	\label{2.07}
	\limninf \sqrt[n]{\abs{a_n}},
	\end{equation}
	то для радиуса сходимости ряда \eqref{2.01} имеем:
	\begin{equation}
	\label{2.08}
	R = \dfrac{1}{\limninf \sqrt[n]{\abs{a_n}}}.
	\end{equation}
\end{theorem}
\begin{proofUndotted} проведём по той же схеме, что и в предыдущей теореме.

	Т.к. случай $ x=x_0 $ тривиален (в данной точке ряд всегда сходится), то рассмотрим случай $ x \ne x_0 $.

	По признаку Коши сходимости ЧР для \eqref{2.01} получаем:
	\begin{equation*}
	\exists \; k = \limninf \sqrt[n]{\abs{a_n (x-x_0)^n}} =
	\abs{x-x_0} \limninf \sqrt[n]{\abs{a_n}} \overset{\eqref{2.08}}{=} \dfrac{\abs{x-x_0}}{R}.
	\end{equation*}

	Если $ k < 1 $, т. е. $ \abs{x - x_0} < R $, то СтР \eqref{2.01} сходится.

	Если $ k > 1 $, т. е. $ \abs{x - x_0} > R $, то СтР \eqref{2.01} расходится.


	Таким образом, в силу определения, величина \eqref{2.08} будет радиусом сходимости для \eqref{2.01}.
\end{proofUndotted}

\begin{notes}
	\item В силу связи между признаками Даламбера и Коши сходимости ЧР, в случае, когда предел \eqref{2.06} не существует (ни конечный, ни бесконечный),
	предел \eqref{2.08} может существовать, и в этом смысле формула Коши \eqref{2.08} предпочтительнее, чем \eqref{2.06}.

	\item Можно показать, что в случае, когда в \eqref{2.08} нет ни конечного, ни бесконечного предела, радиус сходимости для \eqref{2.01} всегда можно вычислить по
	\textbf{формуле Коши-Адамара}, использующей понятие верхнего предела последовательности:
	\begin{equation}
	\label{2.09}
	R = \dfrac{1}{\;\; \overline{\limninf} \sqrt[n]{\abs{a_n}} \;\;}.
	\end{equation}
	Под верхним пределом последовательности подразумевается верхняя грань (supremum) множества конечных пределов всех сходящихся подпоследовательностей рассматриваемой последовательности.
\end{notes}

\newpage
\begin{col-answer-preambule}
\end{col-answer-preambule}

\colquestion{Теорема о локальной равномерной сходимости СтР, замечания к ней и следствие из неё (о равенстве степенных рядов).}
\begin{plan}
\item Рассматриваем произвольный отрезок из интеравала сходимости.
\item Делаем отрезок симметричным относительно $x_0$.
\item Ограничиваем члены СтР сверху: $a_n r^n$.
\item Применяем ообобщённый признак Коши (супремум пределов)
\end{plan}
Следствие о равенстве СтР:

\begin{plan}
\item Приравниваем сумму двух рядов
\item Подставляем $x_0$, получаем равенство $a_0 = b_0$
\item Делим остаток на $(x - x_0)$
\item Предел $x \to x_0$, получаем равенство $a_1 = b_1$. Goto 2.
\end{plan}
\begin{theorem}[о локальной равномерной сходимости СтР]
	Если СтР \eqref{2.01} имеет ненулевой радиус сходимости, то этот ряд \eqref{2.01}  сходится равномерно на любом отрезке из интервала сходимости данного ряда.
\end{theorem}
\begin{proof}$  $

	Рассмотрим $ \forall \; [a, b] \subset I = \interval]{ \nullFrac x_0-R \;;\; x_0 + R \nullFrac}[$, где $ R > 0 $ - радиус сходимости СтР \eqref{2.01}. Имеем:
	\begin{equation}
	\label{2.10}
	x_0-R < a < b < x+R \Rightarrow -R < a-x_0 < b-x_0 < R \Rightarrow
	\begin{cases}
	\abs{a-x_0} < R, \\
	\abs{b-x_0} < R.
	\end{cases}
	\end{equation}
	Полагая $ r = \max \left\{\nullFrac \abs{a-x_0}, \; \abs{b-x_0} \nullFrac \right\} $, в силу \eqref{2.10} получаем:
	\begin{equation}
	\label{2.11}
	0 \leq r < R.
	\end{equation}
	Отсюда для $ \forall \; x \in [a, b] $ получаем:
	\begin{equation*}
	\abs{x - x_0} \leq \max \left\{\nullFrac \abs{a-x_0}, \; \abs{b-x_0} \nullFrac \right\} = r,
	\end{equation*}
	поэтому для $ \forall \; n \in \mathbb{N}_0 $ имеем:
	\begin{equation*}
	\abs{a_n (x-x_0)^n} = \abs{a_n} \abs{x-x_0}^n \leq \abs{a_n} r^n = c_n
	\text{ - мажоранта.}
	\end{equation*}

	Применяя к ряду $ c_n $ обобщённый признак Коши сходимости ЧР, получаем:
	\begin{equation*}
	\exists \; \overline{\limninf} \sqrt[n]{\abs{c_n}} =
	\overline{\limninf} \sqrt[n]{\abs{a_n} r^n} =
	r \cdot \underbrace{\overline{\limninf} \sqrt[n]{\abs{a_n}}}_{\frac{1}{R}}
	\overset{\eqref{2.09}}{=} \dfrac{r}{R} \overset{\eqref{2.11}}{<} 1,
	\end{equation*}
	а значит, ряд $ \sum c_n $ сходится.\\

	Таким образом, мы получили равномерно сходящуюся числовую мажоранту, и поэтому, по мажорантному признаку Вейерштрасса для ФР, рассматриваемый СтР \eqref{2.01} будет равномерно сходиться на
	$ \forall \; [a, b] \subset I$.
\end{proof}

\begin{notes}
	\item Из доказанной теоремы следует, что любой СтР сходится локально равномерно на интервале своей сходимости.

	\item Применяя теорему Стокса-Зейделя для ФР и учитывая, что в \eqref{2.01} все слагаемые являются непрерывными функциями на $ I $,
	в силу локальной равномерной сходимости \eqref{2.01} на $ I $, внутри интервала сходимости сумма любого СтР \eqref{2.01}  будет являться непрерывной функцией.
\end{notes}
\begin{consequence}[о равенстве СтР]
	Если для СтР \eqref{2.01} с непрерывной суммой $ S_n(x) $ есть степенной ряд
	$ \sumnzi b_n (x-x_0)^n $ с соответствующей суммой $ T(x) $, причём $ T(x) = S(x) $ в некоторой окрестности центра разложения \nolinebreak$ x_0 $, то тогда и сами СтР совпадают,
	т.е. $ a_n = b_n, \; \text{для } \; \forall \; n \in \mathbb{N}_0 $.
\end{consequence}

\begin{proof} Пусть имеем, что
	\begin{equation*}
	S(x) = a_0 + a_1 (x-x_0) + \ldots = b_0 + b_1 (x-x_0) + \ldots = T(x).
	\end{equation*}

	В силу непрерывности $ S(x) $ и $ T(x) $ в соответствующей окрестности точки $ x_0 $ при
	$ x \to x_0 $, получаем:
	\begin{equation*}
	\begin{split}
	& a_0 = \lim\limits_{x\to x_0} S(x) = \lim\limits_{x\to x_0} T(x) = b_0, \text{ отсюда }\\
	& a_1 (x-x_0) + a_2(x-x_0)^2 + \ldots = b_1 (x-x_0) + b_2(x-x_0)^2 + \ldots.
	\end{split}
	\end{equation*}
	Таким образом, для $ \forall \; x \neq x_0 $ имеем:
	\begin{equation*}
	a_1 + a_2(x-x_0)  + \ldots =  b_1 + b_2(x-x_0) + \ldots.
	\end{equation*}

	Используя опять соответствующую окрестность точки $ x_0 $, при $ x \to x_0 $, получим, что
	$ a_1 = b_1 $ и так далее (по ММИ).
\end{proof}

\newpage
\begin{col-answer-preambule}
\end{col-answer-preambule}

\colquestion{Теорема о дифференцировании СтР, замечания и следствие из неё.}
\begin{plan}
\item Слагаемые - непрерывно дифференцируемы + имеем поточечную сходимость СтР, поэтому сумма СтР будет непрерывно дифференцируемой.
\item Считаем радиус по обобщённой теореме Коши (формула Коши-Адамара)
\end{plan}
Следствие

\begin{plan}
\item. Просто дифференцируем и замечаем схожесть с рядом Тейлора.
\end{plan}
\begin{theorem}[о дифференцировании СтР]
	Сумма СтР \eqref{2.01} внутри его интервала сходимости является непрерывной дифференцируемой функцией, причём у продифференцированного СтР
	будет тот же радиус (а, значит, и интервал) сходимости, что и у исходного ряда \eqref{2.01}.
\end{theorem}
\begin{proof}
	По теореме о почленном дифференцировании ФР и замечанию к ней достаточно показать, что возможно почленное дифференцирование \eqref{2.01} на $\forall$ отрезке
	$ [a,b] \subset  I = \interval]{ \; x_0-R \;;\; x_0 + R \;}[ $.
	\begin{enumerate}
		\item В \eqref{2.01} слагаемые $ u_n(x) = a_n (x-x_0)^n $, $ n \in \mathbb{N}_0 $ являются непрерывно дифференцируемыми функциями для $ \forall \; x \in [a; b]$
		т.к. $ \exists \; u_n'(x) = n a_0 (x-x_0)^{n-1} $ непрерывная на $ [a;b] $.
		\item Так как $\forall$ СтР \eqref{2.01} сходится поточечно внутри своего интервала сходимости, то \linebreak
		$ \sumnzi u_n(x) \xrightarrow[]{\text{для } \forall \; {[a;b]} \subset I} S(x) $.
	\end{enumerate}
	Осталось показать, что продифференцированный СтР
	\begin{equation*}
	\sumnzi u_n'(x) = \sumnzi n a_n (x-x_0)^{n-1} = \sum_{n=1}^{\infty} n a_n (x-x_0)^{n-1}
	= \sumnzi (n+1) a_{n+1} (x-x_0)^n \overset{[a; b]}{\rightrightarrows}.
	\end{equation*}
	Используя \important{формулу Коши-Адамара}, имеем:
	\begin{align*}
	& \overset{\sim}{R} =
	\dfrac{1}{\;\; \overline{\limninf} \sqrt[n]{(n+1)\abs{a_{n+1}}} \;\;} =
	\dfrac{1}{\;\; \overline{\limninf} \left(\sqrt[n]{n+1} \sqrt[n]{\abs{a_{n+1}}}\right) \;\;} =
	\begin{sqcases}
	\sqrt[n]{n+1} \xrightarrow[n \to\infty]{} 1, \\
	\sqrt[n]{\abs{a_{n+1}}} = \left(\sqrt[n+1]{\abs{a_{n+1}}}\;\;\right)^{\frac{n+1}{n}}
	\end{sqcases} = \\
	& =  \dfrac{1}{\;\; \overline{\limninf} \left(\sqrt[n+1]{\abs{a_{n+1}}}\;\; \right)^{\frac{n+1}{n}} \;\;}   =
	\begin{sqcases}
	\frac{n+1}{n} \xrightarrow[n \to \infty]{} 1, \\\\
	\overline{\limninf} \sqrt[n+1]{\abs{a_{n+1}}} = \frac{1}{R}
	\end{sqcases}
	= \dfrac{1}{ \;\; \frac{1}{R} \;\;} = R.
	\end{align*}

	Значит, у исходного и продифференцированного рядов один и тот же радиус, а, значит, и интервал, сходимости.
	Тогда, в силу того, что $\forall$ СтР сходится локально равномерно, получаем, что
	$ \sumnzi u_n'(x) \overset{[a; b]}{\rightrightarrows} S'(x) $.


	Причём, в силу непрерывности слагаемых, S(x) будет непрерывно дифференцируема на $ \forall \; [a; b] \subset I$, а, значит, и для $ \forall \; x \in I $.
\end{proof}

$  $

\begin{notes}
	\item Применяя последовательно дифференцирование к СтР \eqref{2.01}, получим по ММИ, что сумма ряда \eqref{2.01} будет бесконечное число раз дифференцируемой функцией.

	\item Можно показать, что дифференцирование СтР хоть и сохраняет интервал сходимости, но в общем случае \important{не улучшает} его множество сходимости в том смысле, что если, например,
	исходный ряд \eqref{2.01} находится на каком-то из концов интервала I $ (x=x_0 \pm R) $,
	то продифференцированный ряд уже может расходиться на этом конце.
\end{notes}

\begin{consequence}
	Если на интервале $ I = \interval]{ \; x_0-R \;;\; x_0 + R \;}[ $  бесконечно дифференцируемая функция $ f(x) $ представляется в виде
	$ f(x) = \sumnzi a_n(x-x_0)^n, \text{для }\forall\; x \in I $,
	то для неё СтР \eqref{2.01} будет являться соответствующим рядом Тейлора в окрестности точки $ x_0 $, т. е. для $ \forall \; a_n = \dfrac{f^{(n)} (x_0)}{n!}$,  $n \in \mathbb{N}_0 $.
\end{consequence}
\begin{proof}
	Действительно, дифференцируя почленно $ n $ раз равенство
	\begin{equation*}
	f(x) = a_0 + a_1 (x-x_0) + \ldots + a_n(x-x_0)^n + \ldots
	\end{equation*}
	в силу доказанной теоремы получим:
	\begin{equation*}
	\exists \; f^{(n)}(x) = n! \cdot a_n + (n+1)! \cdot a_{n+1} (x-x_0) + \ldots
	\end{equation*}
	Отсюда при $ x \to x_0 $ имеем:
	\begin{equation*}
	n! \cdot a_n = \lim\limits_{x \to x_0} f^{(n)} (x) = f^{(n)} (x_0) \;\;\;\;\;
	\Leftrightarrow \;\;\;\;\; a_n =  \dfrac{f^{(n)} (x_0)}{n!},
	\end{equation*}
	т.е. $ \forall \; a_n$ - коэффициент в разложении в ряд Тейлора.
\end{proof}

\newpage
\begin{col-answer-preambule}
\end{col-answer-preambule}

\colquestion{Теорема о замене переменной в несобственных интегралах (НИ) и замечание к ней.}
\begin{plan}
\item Применяем теорему о замене переменных в ОИ на произвольном подотрезке $\interval[{\alpha; \gamma}]$.
\item Переходим к пределу $\gamma \to \beta - 0$.
\end{plan}
\begin{theorem}[о замене переменных в НИ]
	Будем одновременно рассматривать как НИ-1, так и НИ-2.

	Пусть $f(x)$ определена для $ \forall \; x \in [a;b[$, где либо $b = + \infty$ (НИ-1), либо $f(b-0) = \infty$ (НИ-2).

	Если функция $x(t) = \phi (t)$ - \important{непрерывно дифференцируема} для $\forall \; t \in [\alpha; \beta[$ и \important{строго монотонна}, то в случае, когда: $\begin{cases}
	\phi (\alpha) = a, \\
	\phi (\beta - 0) = b.
	\end{cases}$, интеграл $\dint\limits_{a}^{b} f(x)dx$, где $b = + \infty$ (НИ-1) либо $f(b-0) = +\infty$ (НИ-2), сходится тогда и только тогда, когда сходится интеграл
	\begin{equation}
	\label{eq:lecture03-09}
	\dint\limits_{\alpha}^{\beta} f(\phi(t)) \phi^{'}(t)dt.
	\end{equation}
	При этом справедлива формула замены переменных в НИ:
	\begin{equation}
	\label{eq:lecture03-10}
	\dint\limits_{a}^{b} f(x)dx = \begin{sqcases} x = \phi(t) \Rightarrow dx = \phi^{'}(t)dt, \\ \left.x\right|_{a = \phi(\alpha)}^{b = \phi(\beta - 0)}.\end{sqcases} = \dint\limits_{\alpha}^{\beta} f(\phi(t)) \phi^{'}(t)dt,
	\end{equation}
	причём в правой части \eqref{eq:lecture03-10} может стоять как некоторый НИ, так и обычный интеграл Римана.
\end{theorem}
\begin{proof}
	Следует из соответствующей теоремы о замене переменных в ОИ (интеграле Римана).

	Для доказательства, выбирая для $\forall \; \gamma \in [\alpha; \beta[$, в силу строгой монотонности $\phi(t)$, получаем что $c = \phi(\gamma) \in [a;b[$. При этом для $\forall \; c \in [a;b[ \; \exists \; ! \; \gamma \in [\alpha; \beta[$.

	Тогда по \important{теореме о замене переменных в ОИ} имеем:
	\begin{equation*}
	\dint\limits_{a}^{c} f(x)dx = \begin{sqcases} x = \phi(t) \Rightarrow dx = \phi^{'}(t)dt, \\ \left.x\right|_{a = \phi(\alpha)}^{c} \Rightarrow \exists \; ! \; \gamma \in [\alpha; \beta[ \; | \; c = \phi(\gamma) \Rightarrow \left.t\right|_{\alpha}^{\gamma}.\end{sqcases} = \dint\limits_{\alpha}^{\gamma} f(\phi(t)) \phi^{'}(t)dt.
	\end{equation*}

	Отсюда, переходя к пределу и учитывая, что $\gamma \to \beta - 0 \Rightarrow c \to b - 0$, получаем \eqref{eq:lecture03-10}.
\end{proof}
\begin{note}

	Для НИ-2 вида $\dint\limits_{a}^{b-0}f(x)dx$ после замены переменных имеем:
	\begin{equation*}
	t = \left.\dfrac{1}{b-x} \right|_{\frac{1}{b-a	} > 0}^{+ \infty}, \text{ а для } \left.x\right|_{a}^{b-0},
	\end{equation*}
	отсюда получаем: $x = b - \dfrac{1}{t} \Rightarrow dx = \dfrac{dt}{t^2} \Rightarrow \dint\limits_{a}^{b} f(x)dx = \dint\limits_{\frac{1}{b-a}}^{+\infty} \dfrac{f(b-\frac{1}{t})}{t^2}dt$.

	Тем самым мы \important{свели НИ-2 к соответствующему НИ-1}, дальнейшее исследование которого, например, на сходимость, можно проводить с помощью полученных ранее условий сходимости НИ-1.
\end{note}

\newpage
\begin{col-answer-preambule}
		Аналогично, как и теорема о замене переменных в НИ-2, обосновываются формулы двойной подстановки (аналог формулы Ньютона-Лейбница) и метод интегрирования по частям для НИ-2 и НИ-1.
\end{col-answer-preambule}

\colquestion{Формула двойной подстановки для НИ и интегрирование по частям в НИ.}
\begin{plan}
\item Рассматриваем частичную первообразную $F_0(x) = \dintl_{x_0}^xf(t)dt$
\item По теореме Барроу можно продифференцировать интеграл.
\item Рассматриваем произвольную первообразную $F(x)$ и замечаем, что $F(x) = F(x_0) + c_0$
\item $x = a$, $x = b - 0$
\item Выражаем общий интеграл и получаем нужную формулу. При этом проблемным в формуле будет только $F(b-0)$. Т.е. интеграл сходится $\Leftrightarrow$ сходится $F(b-0)$.
\end{plan}
Интегрирование по частям

\begin{plan}
\item По формулам двойной подстановки и интегрирования по чатям для НИ.
\end{plan}
\begin{theorem}[Формула Ньютона-Лейбница для НИ.]
Пусть для $f(x)$, определённой для $\forall \; x \in [a, b[$, где $b = + \infty$ или $f(b - 0) = \infty$ существует дифференцируемая первообразная $F(x)$, т.е. $\exists \; F^{'}(x) = f(x), \text{для }\forall \; x \in [a, b[$. Тогда имеем:
\begin{equation*}
\begin{split}
&\dint\limits_{a}^{b} f(x)dx = \lim\limitslimits{c \to + \infty}{c \to b-0} \dint\limits_{a}^{c} f(x)dx = \lim\limits_{c\to b-0} \begin{sqcases} F(x) \end{sqcases}^{c}_{a} = \\
& =\lim\limits_{c\to b-0} \left(F(c) - F(a)\right) = F(b-0) - F(a) = \begin{sqcases} F(x) \end{sqcases}^{b-0}_{a}.
\end{split}
\end{equation*}
При этом используемый интеграл сходится тогда и только тогда, когда значения $\linebreak F(b-0), F(+\infty)$ конечны.
\end{theorem}
\begin{proof}
	Для $fix \; x_0 \in [a,b[$ рассмотрим $F_0(x) = \dintl_{x_0}^	{x}f(t)dt$ - одну из первообразных для $f(x)$, т.к. по теореме Барроу $\exists \; F_0^{'}(x) = f(x)$. Рассмотрим $\forall \; F(x)$ - первообразную $f(x)$ на $[a,b[$. Тогда $\exists \; c_0 = const \; | \; F(x) = F_0(x) + c_0$, т.е. $F(x) - c_0 = F_0(x) = \dintl_{x_0}^{x}f(t)dt$. Полагая здесь $x := a, x := b-0$, имеем: \newline $\begin{cases}F(a) - c_0 = \dintl_{x_0}^{a} f(t)dt, \\ F(b-0) - c_0 = \dintl_{x_0}^{b} f(t)dt. \end{cases} \Rightarrow \left(F(b-0) - c_0\right) - \left(F(a) - c_0\right) = \dintl_{x_0}^{b-0} f(t)dt - \dintl_{x_0}^{a} f(t)dt = \dintl_{x_0}^{b-0} f(t)dt + \dintl_{a}^{x_0} f(t)dt = \dintl_{a}^{b-0} f(t)dt = \newline = F(b-0) - F(a)$.

	$\dintl_{a}^{b-0} f(t)dt$ сходится $\Leftrightarrow F(b-0)$, т.к. $F(a) = const \in \mathbb{R}$.
\end{proof}
\begin{note}
На практике формулы двойной подстановки используются в том же виде, что и для ОИ: $\dint\limits_{a}^{b} f(x)dx = \begin{sqcases} \dint\limits f(x)dx \end{sqcases}^b_a$.
\end{note}
\begin{theorem}[Интегрирование по частям в НИ.]
Пусть $u = u(x), v = v(x)$ непрерывно дифференцируемы на $\forall \; x \in [a; b[$, где $b = + \infty$ или $f(b-0) = \infty$.

Если существует конечный предел $\lim\limitslimits{x \to b-0}{(x\to +\infty)} u(x) v(x) = u(b-0)v(b-0) \in \mathbb{R}$, то тогда в случае сходимости одного из использованных ниже интегралов, получаем:
\begin{equation*}
\dint\limits_{a}^{b} u(x) v^{'}(x)dx = \begin{sqcases} u(x)v(x)\end{sqcases}^b_a - \dint\limits_{a}^{b} v(x) u^{'}(x)dx.
\end{equation*}
\end{theorem}
\begin{proof}
	По формулам двойной подстановки для НИ и интегрирования по частям в ОИ: \newline $\dintl_a^{b-0} u(x) dv(x) = \begin{sqcases}\dintl u(x) v^{'}(x) dx\end{sqcases}^{b-0}_{a} = \begin{sqcases}u(x)v(x) - \dintl v(x)du(x)\end{sqcases}^{b-0}_{a} = \begin{sqcases}u(b-0) v(b-0) \in \mathbb{R}\end{sqcases} = \newline = \left(u(b-0) v(b-0) - \dintl v(b-0) u^{'}(b-0) db\right) - \left(v(a)u(a) - \dintl v(a) u^{'}(a) da\right) = \newline = \begin{sqcases}u(x)v(x)\end{sqcases}^{b-0}_{a} - \begin{sqcases}\dintl v(x) u^{'}(x) dx\end{sqcases}^{b-0}_{a} = \begin{sqcases}v(x) u(x)\end{sqcases}^{b-0}_{a} - \dintl_{a}^{b-0} v(x) du(x)$.
\end{proof}
\begin{note}
На практике удобнее использовать:
\begin{equation*}
\dint\limits_{a}^{b} udv = \begin{sqcases} uv \end{sqcases}^b_a - \dint\limits_{a}^{b} vdu.
\end{equation*}
\end{note}

\newpage
\begin{col-answer-preambule}
	Функцию $\phi(x)$, определённую на $X$ будем называть \important{равномерным частным пределом} $f(x,y)$ при $y \to y_0$, если
	\begin{equation}
	\label{eq:lecture04-04}
	\text{для } \forall \; \varepsilon > 0 \; \exists \; \delta = \delta (\varepsilon) > 0 \; | \; \text{для } \forall \; x \in X \text{ и для } \forall \; y \in Y \text{ из } 0 < |y-y_0| \leqslant \delta \text{ следует } |f(x,y) - \phi(x)| \leqslant \varepsilon.
	\end{equation}
	В этом случае будем писать
	\begin{equation}
	\label{eq:lecture04-05}
	f(x,y) \overset{X}{\underset{y \to y_0}{\rightrightarrows}} \phi(x).
	\end{equation}
\end{col-answer-preambule}

\colquestion{Признак существования равномерного частного предела для непрерывных Ф2П.}
\begin{plan}
\item Теорема Кантора для ФНП (что-то вроде Коши для ЧР).
\item Хитрая замена нужных x.
\end{plan}
\begin{theorem}[признак равномерной сходимости Ф2П]
	Если функция $f(x,y)$ непрерывна на прямоугольнике $[a,b] \times [c,d]$, являющимся компактом в $\mathbb{R}^2$, и $y_0 \in [c,d]$, то имеем:
	\begin{equation}
	\label{eq:lecture04-08}
	f(x,y) \xrightarrow[y \to y_0, y \in [c,d\text{]}]{[a,b]} f(x, y_0).
	\end{equation}
\end{theorem}
\begin{proof}
	Из \important{теоремы} Кантора для ФНП получаем, что рассматриваемая $f(x,y)$ будет равномерно непрерывна для $\forall \; x \in [a,b]$ и для $\forall \; y \in [c,d]$, т.е.:
	\begin{equation*}
	\begin{split}
	&\text{ для } \forall \; \varepsilon > 0 \; \exists \; \delta = \delta(\varepsilon) > 0 : \text{для }\forall \; \widetilde{x}, \bar{x} \in [a,b] \text{ и для } \forall \; \widetilde{y}, \bar{y} \in Y \\
	& \text{ из } \begin{cases}
	0 < \abs{\bar{x} - \widetilde{x}} \leqslant \delta, \\ 0 < \abs{\bar{y} - \widetilde{y}} \leqslant \delta.
	\end{cases} \Rightarrow \abs{f(\widetilde{x}, \widetilde{y}) - f(\bar{x}, \bar{y})} \leqslant \varepsilon.
	\end{split}
	\end{equation*}

	Полагая здесь: $\begin{cases}
	\widetilde{x} = \bar{x} = x \in [a,b],\\ \widetilde{y} = y \in [c,d], \\ \bar{y} = y_0 \in [c,d].
	\end{cases}$, получаем:
	\begin{equation*}
	\begin{split}
	& \text{ для } \forall \; \varepsilon > 0 \; \exists \; \delta = \delta (\varepsilon) > 0: \text{для } \forall \; y \in [c,d] \text{ из } \abs{y - y_0}  \leqslant \delta(\varepsilon),  \text{ для } \forall \; x \in [a,b] \Rightarrow \\
	& \Rightarrow \abs{f(x, y) - f(x, y_0)} \leqslant \varepsilon.
	\end{split}
	\end{equation*}
	Т.к. здесь $\delta = \delta(\varepsilon) > 0$ не зависит от $x \in [a,b]$, то получаем \eqref{eq:lecture04-05}, где $\phi(x) = f(x, y_0)$, что соответствует \eqref{eq:lecture04-08}.
\end{proof}

\newpage
\begin{col-answer-preambule}
\end{col-answer-preambule}

\colquestion{Критерий Гейне равномерной сходимости Ф2П и замечания к нему.}
\begin{plan}
\item Доказываем в обе стороны!
\item \circled{$\Rightarrow$} по определению.
\item \circled{$\Leftarrow$} Из равномерной сходимости $f(x, y_n)$ и критерия Гейне для Ф1П следует поточечная сходимость $f(x, y)$.
\item Предполагаем, что нету равномерной сходимости и применяем правило Де Моргана.
\item Для каждого $\delta = \dfrac{1}{n}$ выбираем $x_n = x(\delta)$ и $y_n = y(\delta)$.
\item Подставляем $x_n$ в определение поточечной сходимости.
\item Докидываем туда же $y_n$, получаем противоречие т.к. одновременно должно выполняться $blabla \leqslant \varepsilon_0$ и $blabla > \varepsilon_0$.
\end{plan}
\begin{theorem}[критерий Гейне равномерной 	сходимости Ф2П]
	$f(x,y) \underset{y \to y_0}{\overset{X}{\rightrightarrows}} \phi(x) \Leftrightarrow$ для $\forall \; y_n \in Y, y_n \to y_0, y_n \ne y_0$, где $y_0$ - предельная точка для множества $Y$, выполнялось:
	\begin{equation}
	\label{eq:lecture04-090}
	g_n(x) = f(x, y_n) \underset{n \to \infty}{\overset{X}{\rightrightarrows}} \phi(x)
	\end{equation}
\end{theorem}
\begin{proof}
	\circled{$\Rightarrow$}. Пусть выполняется \eqref{eq:lecture04-05}, тогда для $\forall \; \varepsilon > 0 \; \exists \; \delta > 0:$ для $\forall \; y \in Y \text{ из } 0 < \abs{y - y_0} \leqslant \delta$, для $\forall \; x \in X \Rightarrow \abs{f(x,y) - \phi(x)} \leqslant \varepsilon$.

	Рассматривая $\forall \; \left(y_n\right) \in Y$, в пределах точки $y_0$ по найденному ранее  $\delta > 0 \; \exists \; \nu \in \mathbb{R}$ такое, что для $\forall \; n \geqslant \nu \Rightarrow \abs{y_n - y_0} \leqslant \delta$.

	Окончательно получаем: для $\forall \; \varepsilon > 0 \; \exists \; \nu \in \mathbb{R}$ такое, что для $\forall \; n \geqslant \nu$, для $\forall \; x \in X \Rightarrow \linebreak \Rightarrow \abs{y_n - y_0} \leqslant \delta \Rightarrow \abs{f(x,y_n) - \phi(x)} \leqslant \varepsilon$, т.е. имеем \eqref{eq:lecture04-090}.

	\circled{$\Leftarrow$}. Пусть для $\forall \; \left(y_n\right) \in Y$ в предельной точке выполнено \eqref{eq:lecture04-090}. Тогда в силу того, что из равномерной сходимости $g_n(x) = f(x, y_n)$ следует поточечная сходимость ФП $g_n(x)$, получаем, что $g_n(x) \xrightarrow[n \to \infty]{X} \phi(x)$.

	Поэтому в силу критерия Гейне существования предела Ф1П получаем, что:
	\begin{equation*}
	f(x,y_0) = g_n(x) \xrightarrow[n \to \infty]{X} \phi(x) \Rightarrow f(x,y) \xrightarrow[y \to y_0]{X} \phi(x).
	\end{equation*}

	Предположим, что имеем поточечную сходимость, но равномерной сходимости нет, т.е. получаем:
	\begin{equation*}
	f(x,y) \underset{y \to y_0}{\overset{X}{\rightrightarrows}} \phi(x).
	\end{equation*}

	Тогда по \important{правилу де Моргана}, имеем:
	\begin{equation*}
	\exists \; \varepsilon_0 > 0 \text{ такое, что для } \forall \; \delta > 0 \; \exists \; y(\delta) \in Y, \exists	\; x(\delta) \in X  \text{ такое, что из } 0 < \abs{y(\delta) - y_0} \leqslant \delta \Rightarrow
	\end{equation*}
	\begin{equation}
	\label{eq:lecture04-10}
	\Rightarrow |f\left(x(\delta), y(\delta)\right) - \phi(x(\delta))| > \varepsilon_0.
	\end{equation}

	Выбирая для простоты $\delta = \dfrac{1}{n} \xrightarrow[n \to \infty]{}+0$, получаем, что $\begin{cases} \exists \; x_n = x\left(\dfrac{1}{n}\right) \in X, \\ \exists \; y_n = y\left(\dfrac{1}{n}\right) \in Y.
	\end{cases} \text{ такие, что из} \linebreak 0 < \abs{y_n - y_0} \leqslant \delta \Rightarrow \abs{f(x_n, y_n) - \phi(x_n)} > \varepsilon_0$.

	Используя условие $f(x_n, y) \xrightarrow[y \to y_0]{X} \phi(x_n)$, для найденного $\varepsilon_0 > 0$ получаем:
	\begin{equation*}
	  \exists \; \delta_0 > 0 \text{ такая, что для } \forall \; y \in Y \text{ из } 0 < \abs{y - y_0} \leqslant \delta_0 \Rightarrow \abs{f(x_n,y) - \phi(x_n)} \leqslant \varepsilon_0.
	\end{equation*}

	Подставляя $y = y_n$, получаем $0<\abs{y_n - y_0} \leqslant \delta_0 \Rightarrow \abs{f(x_n, y_n) - \phi(x_n)} \leqslant \varepsilon_0$.

	Выбирая теперь $\nu = \dfrac{1}{\delta_0} \in \mathbb{R}, \text{ для } \forall \; n \geqslant \nu \Rightarrow 0 < \abs{y_n - y_0} \leqslant \dfrac{1}{n} \leqslant \dfrac{1}{\nu}$. Отсюда в силу \eqref{eq:lecture04-10} при $\delta = \dfrac{1}{n} > 0$ получаем, что для $\forall \; n \geqslant \nu$ выполняется $\abs{f(x_n,y_n) - \phi(x_n)} > \varepsilon_0$. Противоречие.
\end{proof}

\begin{notes}
	\item Доказанная теорема позволяет из соответствующих свойств ФП получить аналогичные свойства для равномерно сходящихся Ф2П, в том числе сформулированный ранее супремальный критерий равномерной сходимости Ф2П и критерий Коши для Ф2П. Кроме того, в силу теоремы Дини	для ФП имеем соответствующую теорему Дини для равномерной сходимости Ф2П.

	\begin{theorem}[Дини для равномерной сходимости Ф2П]
		Пусть для $\forall \; fix \; y \in Y, f(x,y)$ непрерывна по $x \in [a,b] = X$, причём при монотонной сходимости \newline $y \to y_0 \; (y \uparrow y_0 \text{ либо } y \downarrow y_0)$ соответственно получаем	$f(x,y)$ монотонно сходится к $\phi(x)$ $\left(f(x,y) \uparrow \downarrow \phi(x)\right)$. Тогда, если предельная функция $\phi(x) = \lim\limits_{y \to y_0} f(x,y)$ непрерывна на $X = [a,b]$, то кроме поточечной сходимости будем иметь равномерную сходимость \eqref{eq:lecture04-05}.
	\end{theorem}
	\item 	Аналогично получаем теорему	Стокса-Зейделя для Ф2П.

	\begin{theorem}[Стокса-Зейделя]
		Пусть для $\forall \; fix \; y \in Y, f(x,y)$ непрерывна по $x \in [a,b] = X$.
		Тогда, если $ f(x,y) \overset{[a,b]}{\underset{y \to y_0}{\rightrightarrows}}  \phi(x)$, где $y_0$ - предельная точка для $Y$, то предельная функция будет непрерывной на $[a,b]$.
	\end{theorem}
\end{notes}

\newpage
\begin{col-answer-preambule}
	Предположим, что $f(x,y)$ определена для $\forall \; y \in Y$ и для $\forall \; x \in [a,b]$, причём при $\forall \; fix \; y \in Y$ $f(x,y)$ интегрируема по $x \in [a,b]$. В этом случае:
	\begin{equation}
	\label{eq:lecture04-11}
	\exists \; F(y) = \dintl_a^b f(x,y)dx, y \in Y.
	\end{equation}
	
	\eqref{eq:lecture04-11} - \important{интеграл Римана (собственный)}, зависящий от параметра $y \in Y$.
	
	В дальнейшем интеграл вида \eqref{eq:lecture04-11} будем кратко называть \important{СИЗОП}.
\end{col-answer-preambule}

\colquestion{Теорема о предельном переходе в собственных интегралах, зависящих от параметра (СИЗОП) и замечания к ней.}
\begin{theorem}[о предельном переходе в СИЗОП]
	Пусть определён СИЗОП \eqref{eq:lecture04-11}. Тогда, в случае $f(x,y) \underset{y \to y_0}{\overset{[a,b]}{\rightrightarrows}} \phi(x)$, где, как и в определении СИЗОП \eqref{eq:lecture04-11}, предполагая интегрируемость $f(x,y)$ по $x$, получаем:
	\begin{equation}
	\label{eq:lecture04-12}
	\exists \; \lim\limits_{y \to y_0}\; \dintl_a^b f(x,y)dx = \dintl_a^b \phi(x)dx = \dintl_a^b \lim\limits_{y \to y_0}\; f(x,y)dx.
	\end{equation}
\end{theorem}
\begin{proof}
	В силу \eqref{eq:lecture04-05} имеем \eqref{eq:lecture04-04}, откуда для $I = \dintl_a^b \phi(x)dx$, получаем:
	\begin{equation*}
	\abs{F(y) - I} \overset{\eqref{eq:lecture04-11}}{=} \abs{\dintl_a^b \left(f(x,y) - \phi(x)\right) dx} \leqslant \dintl_a^b \abs{f(x,y) - \phi(x)}dx \overset{\eqref{eq:lecture04-04}}{\leqslant} \dintl_a^b \varepsilon dx = \varepsilon(b-a).
	\end{equation*}
	
	Таким образом, получаем, что $\exists \; M = b-a = const > 0$ такое, что для $\forall \; \varepsilon > 0 \; \exists \; \delta > 0$ такая, что \newline для $\forall \; y \in Y \text{ из } 0 < \abs{y - y_0} \leqslant \delta \Rightarrow \abs{F(y) - I} \leqslant M \varepsilon$.
	
	Откуда по $M$-лемме для сходимости Ф1П, получаем: $	F(y) \xrightarrow[y \to y_0]{} I$, т.е. имеем \eqref{eq:lecture04-12}.
\end{proof}
\begin{notes}
	\item При доказательстве теоремы неявно предполагалось, что $\phi(x) \in \mathbb{R}([a,b])$. Это условие выполняется в силу критерия Гейне существования равномерного частного предела и соответствующего условия интегрируемости Ф1П.
	\item Используя теорему Дини для Ф2П, в силу доказанной теоремы, получаем, что если для $\forall \; fix \; y \in Y \Rightarrow f(x,y)$ непрерывна на $X = [a,b]$, то в случае, когда $f(x,y)$ монотонна по $y$ на $Y = [c,d]$ получаем, что при выполнении условия поточечной сходимости:
	\begin{equation*}
	f(x,y) \underset{y \to y_0}{\overset{[a,b]}{\rightrightarrows}} \phi(x),
	\end{equation*}
	то имеем для $\forall \; y_0 \in [c,d] \Rightarrow \eqref{eq:lecture04-12}$.
	\item Если $f(x,y)$ непрерывна для $\forall \; x \in [a,b]$ и для $\forall \; y \in [c,d]$, тогда справедливо \eqref{eq:lecture04-12}, где $\phi(x) = f(x, y_0)$, для $\forall \; fix \; y_0 \in [c,d]$.
	
	В частности, при указанных условиях СИЗОП \eqref{eq:lecture04-11} является непрерывной функцией на $Y \in [c,d]$, т.к.
	\begin{equation*}
	\exists \; \lim\limits_{y \to y_0} F(y) = \lim\limits_{y \to y_0} \dintl_a^b f(x,y)dx = \dintl_a^b \lim\limits_{y \to y_0} f(x,y)dx= \dintl_a^b f(x,y_0)dx = F(y_0),
	\end{equation*}
	что равносильно непрерывности \eqref{eq:lecture04-11} в любой точке $y_0 \in [c,d]$, причём на концах отрезка рассматривается односторонняя непрерывность.
\end{notes}

\newpage
\begin{col-answer-preambule}
\end{col-answer-preambule}

\colquestion{Теорема о почленном дифференцировании СИЗОП.}
\begin{plan}
\item Рассматриваем $G(y)$ = интеграл от $f'_y(x, y)$ на $[a; b]$.
\item $G(y)$ - непрерывно дифференцируема, а значит интегрируема. Берём интеграл на $[c; y]$.
\item Меняем порядок интегрирования, и берём интеграл, получаем первообразную.
\item По теореме Барроу берём производную.
\end{plan}
\begin{theorem}[о почленном дифференцировании СИЗОП]
	Пусть $f(x,y)$ непрерывна на $[a,b] \times [c,d]$ и для неё:
	\begin{equation*}
	\exists \; \dfrac{\partial f(x,y)}{\partial y} - \text{непрерывна на $[a,b] \times [c,d]$}.
	\end{equation*}

	Тогда СИЗОП \eqref{eq:lecture04-11} будет непрерывно дифференцируемой функцией на $[c,d]$, для которой производная вычисляется по правилу Лейбница:
	\begin{equation}
	\label{eq:lecture04-14}
	F^{'}(y) = \left(\dintl_a^b f(x,y) dx\right)^{'}_y = \dintl_a^b f^{'}_{y}(x,y)dx = \dintl_a^b \dfrac{\partial f(x,y)}{\partial y} dx.
	\end{equation}
\end{theorem}
\begin{proof}
	Для доказательства воспользуемся теоремой об интегрируемости СИЗОП. Рассмотрим функцию
	\begin{equation}
	\label{eq:lecture04-15}
	G(y) = \dintl_a^b \dfrac{\partial f(x,y)}{\partial y} dx.
	\end{equation}

	В силу полученных ранее результатов, СИЗОП \eqref{eq:lecture04-15} корректно определён и является непрерывно дифференцируемой функцией на $[c,d]$. Поэтому функция $G(y)$ для $\forall \; fix \; y \in \; ]c,d[$ будет интегрируемой на $[c,y]$. А значит, получаем:
	\begin{equation*}
	\exists \dintl_c^y G(t)dt \overset{\eqref{eq:lecture04-15}}{=} \dintl_c^y \left(\dintl_a^b  \dfrac{\partial f(x,t)}{\partial t} dx \right) dt.
	\end{equation*}

	Отсюда, меняя порядок интегрирования, в силу теоремы о почленном интегрировании СИЗОП, имеем:
	\begin{equation*}
	\dintl_c^y G(t)dt = \dintl_a^b \left(\dintl_c^y \dfrac{\partial f(x,t)}{\partial t} dt\right) dx = \dintl_a^b \begin{sqcases} f(x,t) \end{sqcases}_{t = c}^{t = y}dx = \dintl_a^b \left(f(x,y) - f(x,c)\right) dx \overset{\eqref{eq:lecture04-11}}{=} F(y) - c_0,
	\end{equation*}
	где $c_0 = \dintl_a^b f(x,c)dx = const$.

	Отсюда получаем, что $F(y) = c_0 + \dintl_c^y G(t)dt$.

	Используя теорему Барроу о дифференцировании интегралов с переменным верхним пределом, получаем: \newline	 $\exists \; F^{'}(y) = \left(c_0\right)^{'}_{y} + \left(\dintl_c^y G(t) dt\right)^{'}_{y} = 0 + G(y) \overset{\eqref{eq:lecture04-15}}{=} \dintl_a^b \dfrac{\partial f(x,y)}{\partial y} dx$, что даёт \eqref{eq:lecture04-14}.
\end{proof}

\newpage
\begin{col-answer-preambule}
	\begin{enumerate}
	\item Пусть $f(x, y)$ определена для $\forall x \in \interval[{a; +\infty}[$ и $\forall y \in Y \subset \R{}$. Если $\forall \fix y \in Y \Rightarrow$
	\begin{equation}
	\label{eq:lecture04-17}
	\dintl_a^{+\infty}f(x, y) = dx \xrightarrow[]{y}.
	\end{equation}
	
	Тогда будет корректно определена функция:
	\begin{equation}
	\label{eq:lecture04-18}
	F(y) = \dintl_a^{+\infty}f(x,y)dx, y \in Y.
	\end{equation}

	\item Пусть НИЗОП \eqref{eq:lecture04-18} сходится на $Y \subset \mathbb{R}$. Если $y_0$ -
	предельная точка $Y$ и выполняется
	\begin{equation*}
	f(x, y) \xrightarrow[y \to y_0]{\interval[{a; +\infty}[} \phi(x),
	\end{equation*}
	то будем говорить, что в данном НИЗОП \important{допустим предельный переход}, если
	\begin{equation}
	\label{eq:lecture04-25}
	\exists \lim\limits_{y \to y_0} \intl_a^{+\infty}f(x, y)dx = \intl_a^{+\infty}\lim\limits_  {y \to y_0}f(x, y)dx =
	\intl_a^{+\infty}\phi(x)dx.
	\end{equation}
\end{enumerate}
\end{col-answer-preambule}

\colquestion{Теорема о предельном переходе в несобственных интегралах, зависящих от параметра (НИЗОП), следствие из неё и замечание к ней.}
    \begin{theorem}[О предельном преходе в НИЗОП]
    	Пусть для $\forall \; \fix y \in Y \Rightarrow f(x, y)$ непрерывна для $\forall \; x \geqslant a$ и для предельной точки $y \in Y$
    	имеем
    	\begin{equation}
    	\label{eq:lecture04-26}
    	f(x, y) \overset{\forall \; \interval[{a; A}]}{\underset{y \to y_0}{\rightrightarrows}} \phi(x), \text{ где } \forall \; A > a.
    	\end{equation}
    	Если $\dintl_a^{+\infty}f(x, y)dx \overset{y}{\rightrightarrows}$, то тогда возможен предельный переход
    	\eqref{eq:lecture04-25}.
    \end{theorem}
    \begin{proof}
    	Воспользуемся \important{теоремой о предельном переходе в функциональном ряду}, для чего, беря
    	произвольную последовательность $(A_n) \uparrow +\infty$, по критерию Гейне существования
    	конечного предела функции для \eqref{eq:lecture04-18} получаем
    	\begin{align*}
    	&\exists \lim\limits_{y \to y_0} F(y) = \lim\limits_{y \to y_0}\dintl_a^{+\infty}f(x, y)dx =
    	\sqcase{\dintl_a^{+\infty} = \lim\limits_{An \to +\infty}\parenthesis{\dintl_{A_0}^{A_1} + \dintl_{A_1}^{A_2} + \ldots
    			+ \dintl_{A_{n - 1}}^{A_n}} = \sum\limits_{n = 1}^{\infty}\dintl_{A_{n - 1}}^{A_n}f(x, y)dx}=\\
    	&=\sum\limits_{n = 1}^{\infty}\lim\limits_{y \to y_0}\intl_{A_{n - 1}}^{A_n}f(x, y)dx =
    	\sqcase{\text{По теореме о предельном переходе в СИЗОП}} =\\
    	&=\sum\limits_{n = 1}^{\infty}\intl_{A_{n - 1}}^{A_n}\lim\limits_{y \to y_0}f(x, y)dx =
    	\sum\limits_{n = 1}^{\infty}\dintl_{A_{n - 1}}^{A_n}\phi(x)dx =
    	\lim\limits_{An \to +\infty}\parenthesis{\dintl_{A_0 = a}^{A_1} + \intl_{A_1}^{A_2} + \ldots
    		+ \intl_{A_{n - 1}}^{A_n}} =\\
    	&=\lim\limits_{An \to +\infty}\dintl_a^{A_n}\phi(x)dx = \dintl_a^{+\infty}\phi(x)dx,
    	\end{align*}
    	т.е. имеем \eqref{eq:lecture04-25}.
    \end{proof}
    
    \begin{consequence}[О непрерывности НИЗОП]
    	Пусть $f(x, y)$ непрерывная для $\forall \; x \in \interval[{a; +\infty}[$ и для
    	$\forall \; y \in \interval[{c; d}]$. Если интеграл
    	\begin{equation*}
    	\intl_a^{+\infty}f(x, y)dx \overset{\interval[{c; d}]}{\rightrightarrows}
    	\end{equation*}
    	то НИЗОП \eqref{eq:lecture04-18} - непрерывная функция на $\interval[{c; d}]$, т.е.
    	\begin{equation*}
    	\text{для } \forall \; y \in \interval[{c; d}] \Rightarrow \exists \lim\limits_{y \to y_0}F(y) =
    	\lim\limits_{y \to y_0}\intl_a^{+\infty}f(x, y)dx = \intl_a^{+\infty}f(x, y)dx = F(y_0).
    	\end{equation*}
    \end{consequence}
    \begin{proof}
    	Из непрерывности $f(x, y)$ на $\interval[{a; +\infty}[\times\interval[{c; d}]$ следует, что
    	\begin{equation*}
    	\text{для } \forall \; \fix A \geqslant a \Rightarrow f(x, y)
    	\overset{\interval[{a; A}]}{\underset{y \to y_0}{\rightrightarrows}} f(x, y) =
    	\phi(x) (\text{для } \forall \;  \fix y_0 \in \interval[{c; d}])
    	\end{equation*}
    	Далее, используя доказательство теоремы в силу \eqref{eq:lecture04-25}
    	\begin{equation*}
    	\exists \lim\limits_{y \to y_0}F(y) = \intl_a^A\phi(x)dx \intl_a^{+\infty}f(x, y_0)dx =
    	F(y_0),
    	\end{equation*}
    	что и требовалось доказать.
    \end{proof}
    \begin{note}
    	Доказанная теорема и следствие справедливы и в отсутствии равномерной сходимости для
    	рассматриваемого НИЗОП, если он сходится локально равномерно на $Y$,
    	\begin{equation*}
    	\text{для }\forall \interval[{\alpha; \beta}] \subset Y \Rightarrow \intl_a^{+\infty}f(x, y)dx
    	\overset{\interval[{\alpha; \beta}]}{\rightrightarrows}
    	\end{equation*}
    	Это связано с тем, что свойство непрерывности функции на множестве определено в любой точке из
    	этого множества. Поэтому, выбирая $\forall \; \fix y_0 \in Y$ и заключая его в соответствующий
    	отрезок $y_0 \in \interval[{\alpha; \beta}] \subset Y$,  в случае локальной равномерной
    	сходимости получаем, например, что \eqref{eq:lecture04-18} будет непрерывна на
    	$\interval[{\alpha; \beta}]$, а значит, в точке $y_0$. А исходя из этого, получаем непрерывность
    	\eqref{eq:lecture04-18} на всём $Y$.
    \end{note}
\newpage
\begin{col-answer-preambule}
\end{col-answer-preambule}

\colquestion{Теорема об интегрировании НИЗОП и замечания к ней.}
\begin{plan}
\item Рассматриваем последовательность $(A_n) \uparrow$
\end{plan}
    \begin{theorem}[Об интегрировании НИЗОП]
    	Пусть $f(x, y)$ непрерывная на декартовом произведении
    	$\interval[{a; +\infty}[\times\interval[{c; d}]$. Если интеграл
    	\begin{equation*}
    	\intl_a^{+\infty}f(x, y)dx \overset{\interval[{c; d}]}{\rightrightarrows},
    	\end{equation*}
    	то тогда НИЗОП \eqref{eq:lecture04-18} является интегрируемой на $\interval[{c; d}]$
    	функцией, для которой
    	\begin{equation}
    	\label{eq:lecture04-27}
    	\intl_c^dF(y)dy = \intl_c^ddy\intl_a^{+\infty}f(x, y)dx = \intl_a^{+\infty}dx\intl_c^df(x, y)dy
    	\end{equation}
    \end{theorem}
    \begin{proof}
    	По той же схема, что и в предыдущей теореме, рассмотрим произвольную последовательность
    	$(A_n) \uparrow +\infty (A_0 = a)$ и используем критерий Гейне на основании теоремы о
    	почленном интегрировании СИЗОП, получаем:
    	\begin{align*}
    	&\exists \intl_c^dF(y)dy = \intl_c^d\parenthesis{\suml_{n = 1}^{\infty}\intl_{A_{n - 1}}^{A_n}f(x,y)dx}dy=
    	\sqcase{&u_n(y) = \intl_{A_{n - 1}}^{A_n}f(x, y)dx \text{ непрерывна на } \interval[{c; d}]\\
    		&\suml_{n = 1}^{\infty}u_n(y) = \intl_a^{+\infty}f(x, y)
    		\overset{\interval[{c; d}]}\rightrightarrows} =\\
    	&=\intl_c^d\suml_{n = 1}^{\infty}u_n(y)dy = \suml_{n = 1}^{\infty}\intl_c^du_n(y)dy =
    	\suml_{n = 1}^{\infty}\intl_c^d\parenthesis{\intl_{A_{n - 1}}^{A_n}f(x, y)dx}dy =
    	\suml_{n = 1}^{\infty}\intl_{A_{n - 1}}^{A_n}\parenthesis{\intl_c^d f(x, y)dx}dy =\\
    	&= \liml_{A_n \to +\infty}\parenthesis{\intl_{A_0 = a}^{A_1} + \intl_{A_1}^{A_2} + \ldots +
    		\intl_{A_{n - 1}}^{A_n}}= \liml_{A_n \to +\infty} \intl_{a}^{A_n}\parenthesis{
    		\intl_c^d f(x, y)dx}dy = \intl_a^{+\infty}dx\intl_c^df(x ,y)dy
    	\end{align*}
    \end{proof}
    \begin{notes}
    	\item Доказанная теорема справедлива не только для случае $x \in \interval[{a; +\infty}[$,
    	$y \in \interval[{c; d}]$, но и для случая $x \in \interval ]{a; +\infty}[$,
    	$y \in \interval[{c; d}]$, при условии, что дополнительно ко всем условиям указанной
    	теоремы выполняется, что точка $x = a$ не является точкой разрыва второго рода для
    	$f(x, y)$, т.е.
    	\begin{equation*}
    	\exists \liml_{x \to a+0}f(x, y) \in \R{}
    	\end{equation*}
    	В этом случае, доопределяя функцию $f(x, y)$ в точке $x = a$, т.е. рассматривая функцию
    	\begin{equation*}
    	g(x, y) = \begin{cases}
    	&f(x, y), x > a, y \in \interval[{c; d}]\\
    	&\liml_{x \to a+0}f(x, y), y \in \interval[{c; d}]
    	\end{cases}
    	\end{equation*}
    	Получаем её непрерывность в точке $x = a$ справа. А далее, учитывая, что рассмотренные
    	интегралы от $f(x, y)$ и $g(x ,y)$ совпадают используя доказанную теорему.
    	\item Можно показать, что наряду с интегрируемым НИЗОП по конечному промежутку возможно
    	его почленное интегрирование по бесконечному промежутку $\interval[{c; +\infty}[$, если
    	\begin{enumerate}
    		\item $f(x, y)$ непрерывна на $\interval [{a; +\infty}[\times\interval[{c; +\infty}[$
    		\item  $\dintl_a^{+\infty}f(x, y)dx \overset{\interval[{c; +\infty}[}
    		{\rightrightarrows}, \dintl_a^{+\infty}f(x, y)dx
    		\overset{\interval[{a; +\infty}[}{\rightrightarrows}$
    	
    	\item Существует один из интегралов: $\exists \dintl_c^{+\infty}dy\dintl_a^{+\infty}f(x, y)dx$ или $ \dintl_a^{+\infty}dx
    	\dintl_c^{+\infty}f(x, y)dy)$
	    \end{enumerate}
	    При выполнении всех этих условий имеем:
	    $\exists \dintl_c^{+\infty}dy\dintl_a^{+\infty}f(x, y)dx = \dintl_a^{+\infty}dx
	    \dintl_c^{+\infty}f(x, y)dy)$
	    
	\end{notes}
    
    \begin{theorem}[об интегрировании НИЗОП-2]
    	Если $f(x,y)$ непрерывна на $[a, b[ \times [c,d]$ и $\dintl_a^b f(x,y)dx = \overset{\interval[{c; d}[}{\rightrightarrows}$, то тогда для НИЗОП-2:
    	\begin{equation*}
    	F(y) = \dintl_a^{b-0} f(x,y)dx,
    	\end{equation*}
    	имеем:
    	\begin{equation*}
    	\exists \; \dintl_c^d F(y)dy =  \dintl_c^d dy \dintl_a^b f(x,y)dx = \dintl_a^b dx \dintl_c^d f(x,y)dy 
    	\end{equation*}
    \end{theorem}

\newpage
\begin{col-answer-preambule}
\end{col-answer-preambule}

\colquestion{Теорема о почленном дифференцировании НИЗОП и замечание к ней.}
    \begin{theorem}[О почленном дифференцировании НИЗОП]
    	Пусть $f(x, y)$ - непрерывна на $\interval[{a; +\infty}[\times\interval[{c; d}]$,
    	$\exists \; f'_y(x, y)$ - непрерывная на $\interval[{a; +\infty}[\times\interval[{c; d}]$.
    	Тогда если
    	\begin{enumerate}
    		\item $\dintl_a^{+\infty}f(x, y)dx \xrightarrow[]{\interval[{a; +\infty}[}$
    		\item $\dintl_a^{+\infty}f'_y(x, y)dx \overset{\interval[{a; +\infty}[}
    		{\rightrightarrows}$,
    	\end{enumerate}
    	то тогда НИЗОП \eqref{eq:lecture04-18} - функция почленно дифференцируема на $\interval[{a; +\infty}[$, и её производная
    	вычисляется по правилу Лейбница:
    	\begin{equation*}
    	\label{eq:lecture04-28}
    	\exists \; F'(y) \overset{\eqref{eq:lecture04-18}}{=}
    	\parenthesis{\intl_a^{+\infty}f(x, y)dx}'_y = \intl_a^{+\infty}f'_y(x, y)dx
    	\end{equation*}
    \end{theorem}
    \begin{proof}
    	Для $\forall \; \fix y \in \interval[{c; d}]$ корректно определяем СИЗОП
    	\begin{equation*}
    	\Phi(y) = \intl_c^y\parenthesis{\intl_a^{+\infty}f'_y(x, t)dx}dt
    	\end{equation*}
    	В силу выполнения всех условий почленного интегрирования СИЗОП можем изменить порядок
    	интегрирования
    	\begin{align*}
    	&\Phi(y) = \intl_a^{+\infty}\parenthesis{\intl_c^yf'_y(x, t)}dx =
    	\intl_a^{+\infty}\sqcase{f(x, t)}_{t = c}^{t = y}dx =
    	\intl_a^{+\infty}\parenthesis{\intl_c^yf(x, y) - f(x, c)}dx =\\
    	&=\intl_a^{+\infty}f(x, y)dx - \intl_a^{+\infty}f(x, c)dx = F(y) - F(c)
    	\end{align*}
    	Отсюда, используя теорему Барроу о дифференцировании интеграла с переменным верхним пределом
    	имеем
    	\begin{align*}
    	&\exists F'(y) = \parenthesis{\Phi(y) + F(c)}'_y =
    	\parenthesis{\intl_c^y\parenthesis{\intl_a^{+\infty}f'_y(x, t)dt}dx}'_y =\\
    	&=\sqcase{\intl_0^{+\infty}f'_y(x, t)dx}_{t = y} = \intl_a^{+\infty}f'_y(x, y)dx \Leftrightarrow
    	\eqref{eq:lecture04-28}
    	\end{align*}
    \end{proof}
    \begin{note}
    	Так же, как и в условии непрерывности НИЗОП в доказательстве теоремы о почленном
    	дифференцировании вместо равномерной сходимости рассмотрим НИЗОП используя локальную
    	равномерную сходимость соответствующего НИЗОП.
    \end{note}
\colquestion{Вычисление интеграла Дирихле и его обобщения.}
\colquestion{Лемма Фруллани.}
\colquestion{Первая теорема Фруллани.}
\colquestion{Вторая теорема Фруллани.}
\colquestion{Третья теорема Фруллани.}
\end{document}
