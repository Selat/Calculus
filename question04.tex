\begin{col-answer-preambule}
	Для обозначения поточечной сходимости ФР $\sum u_n(x)$ на X будем использовать запись:
	\begin{equation}
	\label{eq:lecture01-14}
	\sum u_n(x) \overset{X}{\rightarrow}.
	\end{equation}
	\begin{plan}
		\item Формулировка: \important{оДин}и - один знак, Ди\important{ни} - непрерывны, Ди\important{ни} - непрерывны.
		\item Доказательство:
		\subitem 3 свойства остатка ряда $R_n(x) = S(x) - S_n(x)$: Fun UFO (\textcolor{magenta}{F}u\textcolor{magenta}{n} UFO — функция непрерывна, \textcolor{magenta}{F}u\textcolor{magenta}{n} \textcolor{magenta}{U}FO — функциональная последовательность убывает, Fun U\textcolor{magenta}{FO} — функция к 0).
		\subitem дм у пво (\textcolor{magenta}{д}е \textcolor{magenta}{М}орган, \textcolor{magenta}{у}прощение, \textcolor{magenta}{п}ринцип \textcolor{magenta}{в}ыбора, $x_{\textcolor{magenta}{0}}$)
		\subitem противоречие с последним свойством остатка.
		\subitem $R_{m} (x_{nk}) \geqslant R_{nk} (x_{nk}) > \varepsilon_0 \Rightarrow [\text{ переходя к пределу }] \Rightarrow R_m (x_0) = \lim\limits_{n_k \to \infty} R_m(x_{nk}) \geqslant \varepsilon_0$, что противоречит последнему из свойств остатка.
		\item Теорема: то же самое, только вместо сохранения одного знака члены ФП будут монотонны.
			\subitem по доказанному признаку, задав ФР как $u_n(x) = f_n(x) - f_{n-1}(x)$.
	\end{plan}
\end{col-answer-preambule}
\colquestion{Признак Дини равномерной сходимости ФР и следствие из него (теорема Дини для ФП)}
\begin{theorem}[Признак Дини равномерной сходимости ФР]
	Пусть
	\begin{enumerate}
		\item Члены ФР $\sum u_n(x)$ непрерывны и сохраняют один и тот же знак на $X = [a, b], \text{ для } \forall \; n \in \mathbb{N}$.
		\item $\sum u_n(x) \overset{X}{\to} S(x)$.
	\end{enumerate}
	Тогда, если $S(x) = \sum\limits_{n=1}^{\infty} u_n (x)$ - непрерывная функция на $[a, b]$, т.е. $S(x) \in C([a, b])$, то $\sum u_n(x) \overset{X}{\rightrightarrows}$.
\end{theorem}
\begin{proof}
	Рассмотрим на $X = [a,b]$ остатки ряда $R_n(x) = u_{n+1}(x) + \ldots + \ldots = S(x) - S_n(x)$.	Нетрудно видеть, что выполняются следующие свойства:
	\begin{enumerate}
		\item для $\forall \; fix \; n \in \mathbb{N} \Rightarrow R_n (x)$ - непрерывная функция на $[a,b]$ как разность двух непрерывных функций.
		\item для $\forall \; fix \; x \in X \Rightarrow$ $\text{ФП}$ $(R_n(x))$ убывает в случае, когда $\forall \; u_n (x) > 0$, т.к. \newline $R_n(x) = u_n(x) + R_{n+1}(x) \geqslant R_{n+1}(x), \text{ для } \forall \; n \in \mathbb{N}$.
		\item Т.к. имеет место \eqref{eq:lecture01-14}, то для $\forall \; fix \; x \in X \Rightarrow R_n(x) \overset{X}{\to} 0$.
	\end{enumerate}
	Предположим, что рассматриваемая положительная поточечная сходимость на $X$ ФР не является равномерной сходимостью на $X$.

	Тогда по правилу де Моргана имеем: $\exists \; \varepsilon_0 > 0 \; | \; \text{для } \forall \; \nu \in \mathbb{R} \; \exists \; n (\nu) \geqslant 0 \text{ и } \exists \; x (\nu) \in X \; | \; R_{n \nu} (x_\nu) > \varepsilon_0$. Для простоты будем считать, что $\exists \; x_n \in X \; | \; R_n (x_n) > \varepsilon_0$. По принципу выбора из ограниченной последовательности $x_n$ можно выбрать сходящуюся подпоследовательность, т.е. $x_{nk} \underset{n_k \to \infty}{\longrightarrow} x_0$, при этом в силу использования $X = [a,b]$ - компакт, получаем, что $x_0 \in X$. Если зафиксируем $m \in \mathbb{N}$, то для $\forall \; n_k \geqslant m \Rightarrow R_{nk} 	(x_{nk}) > \varepsilon_0$, по свойствам остаткам будем иметь, что $R_{m} (x_{nk}) \geqslant R_{nk} (x_{nk}) > \varepsilon_0$. В неравенстве $R_m (x_{nk}) > \varepsilon_0$, переходя к пределу при $n_k \to \infty \; \text{для } \forall \; m \in \mathbb{N}$, получаем в силу непрерывности $R_n(x): R_m (x_0) = \lim\limits_{n_k \to \infty} R_m(x_{nk}) \geqslant \varepsilon_0$, что противоречит последнему из свойств остатка, а именно $R_m(x_0) \overset{X}{\longrightarrow} 0$ при $m \to \infty$, поэтому из нашего предположения следует, что выполняется $R_m(x_0) \not\to 0$, противоречие, т.е. выполняется $\sum u_n(x) \overset{X}{\rightrightarrows}$.
\end{proof}
\begin{consequence}[Теорема Дини для ФП]
	Если для ФП $f_n(x), n \in \mathbb{N}$ на $X = [a,b]$ выполняются свойства:
	\begin{enumerate}
		\item для $\forall \; f_n(x) \in C([a,b])$ и для	 $\forall \; fix \; x \in X \Rightarrow f_n(x)$ монотонна.
		\item $f_n(x) 	\overset{X}{\longrightarrow}f(x)$. Тогда, если $f(x) \in C([a,b])$, то $f_n(x) \overset{X}{\rightrightarrows}$.
	\end{enumerate}
\end{consequence}
\begin{proof}
	следует из того, что члены рассматриваемой ФП $f_n(x)$ можно рассматривать как частичные суммы соответствующего ФР с общим членом
	\begin{equation}
	\label{eq:1_27}
	\begin{cases}
	u_n(x) = f_n(x) - f_{n-1}(x),\\
	f_0(x) = 0.
	\end{cases}
	\end{equation}
	Действительно, $S_n(x) = f_n(x) - f_0(x) = f_n(x), \text{ для } \forall \; n \in \mathbb{N}$.

	А далее к соответствующему ФР применима теорема Дини равномерной сходимости ФР.
\end{proof}
