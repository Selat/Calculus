\colquestion{Комплексная форма И.Ф. Преобразование Фурье и его свойства.}
%% \begin{plan}
%% \item $z = x + iy, \omega = u + iv$. Выражаем $u, v$.
%% \end{plan}

\begin{col-answer-preambule}
\end{col-answer-preambule}

Пусть выполнены для $f(x)$ все условия теоремы о её представлении в виде интеграла Фурье. Тогда
\begin{align*}
  f(x) = \Phi(x) = \dfrac{1}{2\pi}\intl_{-\infty}^{+\infty}dy\intl_{-\infty}^{+\infty}f(t)\cos y(t - x)dt.
\end{align*}
В дальнейшем все соответствующие НИ-1 будем рассматривать в смысле $v.p.$ В этом случае
\begin{align*}
  \forall B > 0 \Rightarrow \intl_{-B}^Bdy\intl_{-\infty}^{+\infty}f(t)\sin y(t - x)dt = 0.
\end{align*}
А отсюда при
\begin{align*}
  B \to +\infty v.p.\intl_{-\infty}^{+\infty}dy\intl_{-\infty}^{+\infty}f(t)\sin y(t - x)dt = 0.
\end{align*}
В дальнейшем $v.p.$ будем опускать. Поэтому, используя мнимую единицу $i (i^2 = -1)$, получаем
\begin{align*}
  &f(x) = \dfrac{1}{2\pi}\intl_{-\infty}^{+\infty}dy\intl_{-\infty}^{+\infty}f(t)\cos y(t - x)dt +
  \dfrac{i}{2\pi}\intl_{-\infty}^{+\infty}dy\intl_{-\infty}^{+\infty}f(t)\sin y(t - x)dt =\\
  &=\dfrac{1}{2\pi}\intl_{-\infty}^{+\infty}dy\intl_{-\infty}^{+\infty}f(t)(\cos y(t - x) + i\sin y(t - x))dt
  = \sqcase{e^{i\phi} = \cos\phi + i\sin\phi} =
\end{align*}
\begin{equation}
  \label{eq:lecture27-14}
  = \dfrac{1}{2\pi}\intl_{-\infty}^{+\infty}dy\intl_{-\infty}^{+\infty}e^{iy(t - x)}f(t)dt.
\end{equation}
\eqref{eq:lecture27-14} даёт комплексную форму ИФ из которой следует
\begin{align*}
  f(x) = \dfrac{1}{2\pi}\intl_{-\infty}^{+\infty}\parenthesis{e^{-ixy}\intl_{-\infty}^{+\infty}e^{iyt}f(t)dt}
  dy =
\end{align*}
\begin{equation}
  \label{eq:lecture27-15}
  \dfrac{1}{\sqrt{2\pi}}\intl_{-\infty}^{+\infty}e^{-ixy}F(y)dy
\end{equation}
где
\begin{equation}
  \label{eq:lecture27-16}
  F(y) = \dfrac{1}{\sqrt{2\pi}}\intl_{-\infty}^{+\infty}e^{iyt}f(t)dt.
\end{equation}
\eqref{eq:lecture27-16} - преобразование Фурье функции $f(t)$, которое может быть не только
действительным, но и комплексно-значным для $t \in \R{}$. Сам \eqref{eq:lecture27-16} подразумевается
в смысле $v.p.$

Функция $F(y)$ в \eqref{eq:lecture27-16} - образ $f(x)$ при преобразовании Фурье, сама $f(x)$ -
первообразная для \eqref{eq:lecture27-16}. Его можно восстановить по формуле \eqref{eq:lecture27-15},
где интеграл также поразумевается в смысле $v.p.$. \eqref{eq:lecture27-15} - обратное преобразование
Фурье.

Можно показать, что преобразование Фурье \eqref{eq:lecture27-16} обладает следующими свойствами
\begin{enumerate}
\item Линейность: Если $f(x)$ и $g(x)$ имеют преобразования Фурье $F(y)$ и $G(y)$, то тогда
  \begin{align*}
    \forall \mu,\lambda \in \R{} \Rightarrow h(x) = \lambda f(x) + \mu g(x)
  \end{align*}
  имеет преобразовение Фурье $H(y) = \lambda F(y) + \mu G(y)$.

  По ММИ это свойство обобщается на любое число слогаемых.
\item $F(y) \xrightarrow[y \to \infty]{} 0$. Доказательство следует из теоремы Римана-Лебега.
\item Если $f(x)$ непрерывна для $\forall x \in \R{}$, то $F(y)$ также будет непрерывна.
\item Если для $f(x)$ наряду с её прФ $F(y)$ существует прФ для $xf(x) - S(y)$, то тогда в случае
  выполнения \eqref{eq:lecture27-01} и сходимости $\intl_{-\infty}^{+\infty}\abs{xf(x)}dx$ при
  выполнении условий сходимости ИФ получим, что $F'(y)$ будет прФ функции $(ixf(x))$. Доказательство
  следует из теоремы о почленном дифференцировании НИЗОП и правила Лейбница дифференцирования
  НИЗОП, т.к. для интеграла $\intl_{-\infty}^{+\infty}itf(t)e^{iyt}d$ имеем сходящуюся мажоранту
  \begin{align*}
    \intl_{-\infty}^{+\infty}\abs{itf(t)e^{iyt}}dt = \sqcase{\abs{e^{iyt}} = 1} =
    \intl_{-\infty}^{+\infty}\abs{tf(t)}dt \text{ - сходится,}
  \end{align*}
  а поэтому будет сходится равномерно, а тогда
  \begin{align*}
    \exists F'(y) = \dfrac{1}{\sqrt{2\pi}}\parenthesis{\intl_{-\infty}^{+\infty}f(t)e^{iyt}dt}'_y=
    \dfrac{1}{\sqrt{2\pi}}\intl_{-\infty}^{+\infty}\parenthesis{f(t)e^{iyt}}'_ydt =
    \dfrac{1}{\sqrt{2\pi}}\intl_{-\infty}^{+\infty}itf(t)e^{iyt}dt -
  \end{align*}
  преобразование Фурье функции $ixf(x)$.

  При соответствующих условиях это свойство по ММИ обобщается: $\forall k \in \mathbb{N} \Rightarrow
  F^{(k)}(y)$ - прФ функции $(ix)^kf(x)$. На основании свойств можно использовать прФ для вычисления
  соответствующих интегралов и решения дифференциальных и интегральных уравнений и их систем.
\end{enumerate}
