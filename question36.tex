\colquestion{Теорема о поточечной сходимости И.Ф. Следствие из неё и замечание к ней.}
%% \begin{plan}
%% \item $z = x + iy, \omega = u + iv$. Выражаем $u, v$.
%% \end{plan}

\begin{col-answer-preambule}
\end{col-answer-preambule}

\begin{equation}
  \label{eq:lecture27-02}
  a(y) = \dfrac{1}{\pi}\intl_{-\infty}^{+\infty}f(t)\cos ytdt,
\end{equation}
\begin{equation}
  \label{eq:lecture27-03}
  b(y) = \dfrac{1}{\pi}\intl_{-\infty}^{+\infty}f(t)\sin ytdt,
\end{equation}

Из теоремы Римана-Лебега: для кусочно-непрерывной функции $g(t)$ и абсолютно интегрируемой на
$\interval[{a; +\infty}[$ имеем
\begin{equation}
  \label{eq:lecture27-10}
  \liml_{A \to +\infty}\intl_a^{+\infty}g(t)\sin At = 0
\end{equation}

\begin{theorem}[О поточечной сходимости ИФ]
  Если $f(x)$ - абсолютно интегрируема на $\R{}$, т.е. сходится \eqref{eq:lecture27-01}, то в случае,
  когда для
  \begin{equation}
    \label{eq:lecture27-12}
    x_0 \in \R{} \Rightarrow S_0 \in \R{} \vert \exists \liml_{t \to +0}
    \parenthesis{\dfrac{f(x_0 + t)}{t} + \dfrac{f(x_0 - t) - 2s_0}{t}} \in \R{},
  \end{equation}
  то тогда в точке $x_0$ для ИФ имеем $\Phi(x_0) = S_0$
\end{theorem}
\begin{proof}
  Для $A > 0$ воспользуемся обобщением интеграла Дирихле из котором следует
  \begin{equation}
    \label{eq:lecture27-13}
    \dfrac{2}{\pi}\intl_0^{+\infty}\dfrac{\sin At}{t}dt = 1.
  \end{equation}
  Из \eqref{eq:lecture27-13} для \eqref{eq:lecture27-06} в точке $x = x_0$ получаем
  \begin{align*}
    &F(A, x_0) - S_0 = \dfrac{1}{\pi}\intl_0^{+\infty}\dfrac{f(x_0 - u) + f(x_0 + u)}{u}\sin Audu -
    \dfrac{2S_0}{\pi}\intl_0^{+\infty}\dfrac{\sin Au}{u}du =\\
    &=\dfrac{1}{\pi}\intl_0^{+\infty}\dfrac{f(x_0 - u) + f(x_0 + u) - 2S_0}{u}\sin Audu =
    \sqcase{g(u) = \dfrac{f(x_0 - u) + f(x_0 + u) - 2S_0}{u};\\ \exists g(+0) = \liml_{u \to +0}g(u) \in
      \R{}} =\\
    &=\intl_0^{+\infty}g(u)\sin Audu = \sqcase{\text{Теорема Римана-Лебега}}
    \xrightarrow[A \to +\infty]{}0,
  \end{align*}
  поэтому
  \begin{align*}
    F(A, x_0) \xrightarrow[A \to +\infty]{}S_0 \Rightarrow \Phi(x_0) =
    \liml_{A \to +\infty}F(A, x_0) = S_0.
  \end{align*}
\end{proof}

\begin{consequence}
  Для кусочно-непрерывной абсолютно интегрируемой на $\R{}$ функции $f(x)$, имеем кусочно-непрерывную
  производную на любом конечном промежутке, её интеграл Фурье в любой точке $x_0 \in \R{}$ сходится к
  \begin{align*}
    \Phi(x_0) = \dfrac{f(x_0 - 0) + f(x_0 + 0)}{2}.
  \end{align*}
\end{consequence}
\begin{proof}
  Для доказательства достаточно взять
  \begin{align*}
    S_0 = \dfrac{f(x_0 - 0) + f(x_0 + 0)}{2},
  \end{align*}
  а тогда
  \begin{align*}
    \dfrac{f(x_0 - t) + f(x_0 + t) - 2S_0}{t} = \dfrac{f(x_0 -t) - f(x_0 - 0)}{t} +
    \dfrac{f(x_0 + t) - f(x_0 + 0)}{t} \xrightarrow[t \to +0]{} f'_+(x_0 + 0) - f'_-(x_0 - 0) \in
    \R{},
  \end{align*}
  т.е. выполняется условие \eqref{eq:lecture27-10}, а тогда по доказанной теореме
  \begin{align*}
    \Phi(x_0) = S_0 = \dfrac{f(x_0 - 0) + f(x_0 + 0)}{2}.
  \end{align*}
\end{proof}

\begin{note}
  Если $x \in \R{}$ - точка непрерывности для $f$, т.е. $f(x_0 - 0) = f(x) = f(x_0 + 0)$, то тогда
  при выполнении остальных соответствующих условий доказанной теоремы получаем для интеграла Фурье
  \begin{align*}
    &\Phi(x) = \dfrac{f(x_0 - 0) + f(x_0 + 0)}{2} = f(x), \text{ т.е.}\\
    &f(x) = \dfrac{1}{\pi}\intl_0^{+\infty}(a(y)\cos xy + b(y)\sin xy)dy,
  \end{align*}
  где $a(y)$, $b(y)$ вычисляется по формулам \eqref{eq:lecture27-02}, \eqref{eq:lecture27-03}.
\end{note}
