\colquestion{Сходящаяся и непрерывная ФКП. Критерий непрерывности ФКП.}
%% \begin{plan}
%% \item $z = x + iy, \omega = u + iv$. Выражаем $u, v$.
%% \end{plan}

\begin{col-answer-preambule}
\end{col-answer-preambule}
Рассмотрим ФКП $\omega - f(z)$ с областью определения $D \subset \mathbb{C}$. Пусть точка $z_0$ -
предельная для $D$, т.е. либо внутренняя, либо граничная. Число $p \in \mathbb{C}$ называется
пределом $f(z)$ в точке $z_0$, если
\begin{equation}
  \label{eq:lecture16-01}
  \forall \varepsilon > 0 \exists \delta_{\varepsilon} > 0 \vert
  \forall z \in D, 0 < \abs{z - z_0} < \delta_{\varepsilon} \Rightarrow \abs{f(z) - p} < \varepsilon
\end{equation}
В этом случае говорят, что функция $f(z)$ сходится к $p \in \mathbb{C}, z \to z_0$ и пишут
$f(z) \xrightarrow[z \to z_0]{} p$, или $\liml_{z \to z_0}f(z) = p$.

На языке окрестностей имеем
\begin{align*}
  f(z) \xrightarrow[z \to z_0]{}p \in \mathbb{C} \Leftrightarrow \forall \varepsilon > 0 \exists
  \delta > 0 \vert \forall z \in D \cap \overline{B_{\delta}}(z_0) \Rightarrow f(z) \in
  \overline{B_{\varepsilon}}(p)
\end{align*}

\begin{theorem}[Критерий сходимости ФКП]
  Если для ФКП $\omega = f(z), z = x + iy, x = \operatorname{Re}z, y = \operatorname{Im}z$, или
  $\omega = u + iv, u = u(x, y) = \operatorname{Re}f(z), v = v(x, y) = \operatorname{Im}f(z)$, то
  \begin{equation}
    \label{eq:lecture16-02}
    f(z) \xrightarrow[z \to z_0]{}p \in \mathbb{C} \Leftrightarrow
    \begin{cases}
      \exists\liml_{\begin{matrix}
          x \to x_0\\
          y \to y_0
      \end{matrix}}u(x,y) = u_0 = \operatorname{Re}p,\\
      \exists\liml_{\begin{matrix}
          x \to x_0\\
          y \to y_0
      \end{matrix}}v(x,y) = v_0 = \operatorname{Im}p,
    \end{cases}
  \end{equation}
\end{theorem}
\begin{proof}
  По той же схеме, как и в критерии сходимости КП.
\end{proof}

Пусть $z_0$ - внетренняя точка для $D = D(f) \subset \mathbb{C}$. Говорят, что $\omega = f(z)$
непрерывна в точке $z_0 \in D$, если $\exists \liml_{z \to z_0}f(z) = f(z_0)$. Аналогично определяется
нерерывность в граничной точке $z_0 \in D$ (односторонняя непрерывность).

На языке окрестности имеем:
\begin{align*}
  &f(z) \text{ непрерывна в } z_0 \in D \Leftrightarrow\\
  &\Leftrightarrow \forall \varepsilon > 0 \exists \delta > 0 \vert \forall z \in
  \overline{B}_{\delta}(z_0) \Rightarrow f(z) \in \overline{B}_{\varepsilon}(f(z_0)).
\end{align*}

\begin{theorem}[Критерий непрерывности ФКП]
  $\omega = f(z)$ непрерывная в точке $z_0 \in D \Leftrightarrow$
  $u(x, y) = \operatorname{Re}f(z), v(x, y) = \operatorname{Im}f(z)$
  непрерывны в $M_0(x_0, y_0)$, где $x_0 = \operatorname{Re}z_0, y_0 = \operatorname{Im}z_0$.
\end{theorem}
\begin{proof}
  Следует их критерия сходимости ФКП.
\end{proof}
