\begin{col-answer-preambule}
	Обозначение равномерной сходимости:
	\begin{equation}
	\label{eq:lecture01-20}
	\sum u_n(x) \overset{X}{\rightrightarrows}.
	\end{equation}
	Критерий Коши равномерной сходимости ФР:
	\begin{equation}
	\label{eq:lecture01-21}
	\eqref{eq:lecture01-20} \Leftrightarrow \text{для } \forall \; \varepsilon > 0 \; \exists \; \nu = \nu (\varepsilon) \in \mathbb{R} \; |
	\; \text{для } \forall \; x \in X \text{ и для } \forall \; n \geqslant \nu \; \text{и для }\forall \; m \in \mathbb{N} \Rightarrow 
	\abs{S_{n+m}(x) - S_n(x)} = \abs{ \sum\limits_{k = n + 1}^{n + m} u_k(x) } \leqslant \varepsilon.
	\end{equation}
	Критерий Коши сходимости ЧР:
	\begin{equation}
	\label{eq:lecture01-temp}
	\sum a_n \text{ сходится} \Leftrightarrow \text{для } \forall \;  \varepsilon>0 \ \exists \; \nu \in \mathbb{R} : \text{для }\forall \; n \geqslant \nu \; \text{ и} \;  \text{для }\forall\; m \in \mathbb{N} \Rightarrow \abs{S_{n+m} - S_n} = \abs{\sum\limits_{k = n + 1}^{n + m} a_k} \leqslant \varepsilon
	\end{equation}
	Будем говорить, что ЧП $\left(a_n\right)$ является \important{сходящейся числовой мажорантой} для ФР $\sum u_n (x)$, если:
	\begin{equation}
	\label{eq:lecture01-mazh-01}
		\text{1.  ЧР} \sum a_n \text{ сходится},
	\end{equation}
	\begin{equation}
	\label{eq:lecture01-mazh-02}
	\text{2. для }\forall \; n \in \mathbb{N}  \text{ и для } \forall \; x \in X \Rightarrow |u_n(x)| \leqslant a_n.
	\end{equation}
\end{col-answer-preambule}

\colquestion{Мажорантный признак Вейерштрасса равномерной сходимости функционального ряда (ФР) и замечания к нему}
\begin{theorem}[мажорантный признак Вейерштрасса равномерной сходимости ФР] Если ФР имеет на $X$ сходяющуюся числовую мажоранту, то он равномерно сходится на $X$.
\end{theorem}
\begin{plan}
	\item По def \important{сходящейся числовой мажоранты}.
	\subitem Расписать \eqref{eq:lecture01-mazh-01} по \eqref{eq:lecture01-temp}: $\abs{\sum\limits_{k = n+1}^{n+m} a_k} \leqslant \varepsilon$.
	\subitem Подставить \eqref{eq:lecture01-mazh-02} в \eqref{eq:lecture01-21}.
\end{plan}
\begin{proof}
	Доказательство с использованием критерия Коши сходимости ЧП \eqref{eq:lecture01-temp} и критерия Коши равномерной сходимости ФР \eqref{eq:lecture01-21}:

	Т.к. $\sum a_n$ сходится, то
	\begin{equation}
	\label{eq:1_23}
	\text{для } \forall \; \varepsilon > 0 \; \exists \; \nu(\varepsilon) \in \mathbb{R} \; | \; \text{для } \forall \;	n \geqslant \nu \text{ и для } \forall \; m \in \mathbb{N} \Rightarrow \abs{\sum_{k = n+1}^{n+m} a_k} \leqslant \varepsilon.
	\end{equation}

	Если $\text{для } \forall \; n \in \mathbb{N} \text{ и для } \forall \; x \in X \Rightarrow \abs{u_n(x)} \leqslant a_n$, то для частичных сумм ФР $\sum u_n(x)$ имеем: \newline $\abs{S_{m+n} (x) - S_n (x)} = \abs{\sum\limits_{k = n+1}^{n+m} u_k(x)} \leqslant \sum\limits_{k = n + 1}^{n+m} \abs{u_k (x)} \leqslant \sum\limits_{k = n+1}^{n+m} a_k = \abs{\sum\limits_{k = n+1}^{n+m} a_k} \leqslant \varepsilon$, это для $\forall \; n \geqslant \nu = \nu(\varepsilon) \text{ и для } \forall \; m \in \mathbb{N}$, что в силу \eqref{eq:lecture01-21} даёт \eqref{eq:lecture01-20}.
\end{proof}

\begin{notes}
	\item Принцип Вейерштрасса является лишь достаточным условием равномерной сходимости ФР. На практике сходящуюся числовую мажоранту $\left( a_n \right)$ либо находят с помощью соответствующих оценок $\abs{u_n(x)}$ сверху, либо берут $a_n = \underset{x \in X}{sup} \abs{u_n(x)}$. В последнем случае получаем наиболее точную мажоранту, но в случае расходимости $\sum a_n$ даже для этой самой точной мажоранты ничего о равномерной сходимости ФР сказать нельзя, т.е. требуются дополнительные исследования.
	\item Обобщая признак Вейерштрасса, где используется сходимость числовой мажоранты - признак равомерной сходимости ФР, используют функцию мажоранты, а именно получаем:
	\begin{equation*}
	 \text{если } \exists \; v_n(x) \geqslant 0 \; : \; \abs{u_n(x)} \leqslant v_n(x) \; \text{для } \forall \; n \in \mathbb{N} \text{ и для } \forall \; x \in X \text{ и } \sum v_n(x)	 \overset{X}{\rightrightarrows},
	 \end{equation*}
	 то тогда для ФР $\sum u_n(x)$ имеем \eqref{eq:lecture01-20}.
\end{notes}
