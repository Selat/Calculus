\colquestion{Мажорантный признак Вейерштрасса равномерной сходимости функционального ряда (ФР) и замечания к нему}
\begin{theorem}[мажорантный признак Вейерштрасса равномерной сходимости ФР] Если ФР имеет на $X$ сходяющуюся числовую мажоранту, то он равномерно сходится на $X$.
\end{theorem}
\begin{proof}
	Доказательство с использованием критерия Коши сходимости числовых последовательностей и критерия Коши сходимости ФР:

	Т.к. $\sum a_n$ сходится, то
	\begin{equation}
	\label{eq:1_23}
	\forall \varepsilon > 0 \; \exists \; \nu(\varepsilon) \in \mathbb{R} \; | \; \forall	n \geqslant \nu, \forall 	m \in \mathbb{N} \Rightarrow \abs{\sum_{k = n+1}^{n+m} a_k} \leqslant \varepsilon.
	\end{equation}

	Если выполняется неравенство $\abs{u_n(x)} \leqslant a_n, \forall n \in \mathbb{N}, \forall x \in X$, то для частичных сумм \eqref{eq:1_11} ФР \eqref{eq:1_10} имеем: $\abs{S_{m+n} (x) - S_n (x)} = \abs{\sum\limits_{k = n+1}^{n+m} u_k(x)} \leqslant \sum\limits_{k = n + 1}^{n+m} \abs{u_k (x)} \leqslant \sum\limits_{k = n+1}^{n+m} a_k = \abs{\sum\limits_{k = n+1}^{n+m} a_k} \leqslant \varepsilon$, это $\forall n \geqslant \nu = \nu(\varepsilon), \forall m \in \mathbb{N}$, что в силу \eqref{eq:1_21} даёт \eqref{eq:1_20}.
\end{proof}

\begin{notes}
	\item Принцип Вейерштрасса является лишь достаточным условием равномерной сходимости ФР. На практике сходимость числовой мажоранты $\left( a_n \right)$ либо находится с помощью соответствующих оценок $\abs{u_n(x)}$ сверху, либо берут $a_n = \underset{x \in X}{sup} \abs{u_n(x)}$. В последнем случае получаем наиболее точную мажоранту, но в случае расходимости $\sum a_n$ даже для этой самой точной мажоранты ничего о равномерной сходимости ФР сказать нельзя, т.е. требуются дополнительные исследования.
	\item Обобщая признак Вейерштрасса, где используется сходимость числовой мажоранты - признак равомерной сходимости ФР, используя функцию мажоранты, а именно, если $\exists \; v_n(x) \geqslant 0 \; | \; \abs{u_n(x)} \leqslant v_n(x) \; \forall n \in \mathbb{N} $ и $ 	\forall x \in X$ и $\sum v_n(x)	 \overset{X}{\rightrightarrows}$, то тогда для \eqref{eq:1_10} имеем \eqref{eq:1_19}, \eqref{eq:1_20}.
\end{notes}
